\chapter[Bibliographies, Indexes, and Glossaries]%
        {Bibliographies,\\ Indexes, and\\Glossaries} 
\RCSID$Id: ch12.tex,v 1.1 2002/08/23 14:58:46 nwalsh Exp $
\label{chap:bibtex}

\ifincludechapter\else\endinput\fi

Bibliographies\index{bibliographies}\index{tex@\TeX!bibliographies} and 
indexes\index{indexes}\index{tex@\TeX!indexes} are typically difficult to 
incorporate into
a document.  Bibliographies have 
stringent, but varying, presentation requirements (the MLA wants
bibliographies to look one way, the Association of Computing Machinery
wants them another, etc.).
Indexes don't vary that much, but they are tedious to put together.

\BibTeX\ provides a powerful mechanism for handling bibliographies
and citations.  \program{Tib} is another bibliography package for \TeX. 
The \program{MakeIndex} program helps manage the construction
of one or more indexes for a document.  Glossaries are also constructed
with the \program{MakeIndex} 
program\index{glossaries}\index{tex@\TeX!glossaries}.

\section{\protect\BibTeX}

\index{bibtex@\BibTeX}What's wrong with doing 
bibliographies by hand?  Two things.  First,
it is tedious to typeset each bibliography entry according to the strict
requirements of the publisher.  Chances are, you'll have to look up the
requirements each time, and you're bound to make mistakes.  Second, no 
matter what field you work in, it's likely that you'll cite some of the
same articles and books in more than one publication.   Computers are 
supposed to reduce effort, not replicate it.

\BibTeX\ provides a powerful mechanism for dealing with bibliographies
in a mechanical way, considerably reducing effort on your part.  Rather
than formatting each bibliography entry, you build a database of 
bibliography information\index{database (bibliographic)}.  Each 
time you want to make a citation, you
simply use \BibTeX\ to build the bibliography from your database
into your document.  That's easy to
do, as you'll see in a few minutes.

The idea of a bibliography database is introduced in Appendix~B of
\textit{\LaTeX: A Document Preparation System}~\cite{ll:latexbook} and
is described in more detail in \textit{\BibTeX{}ing}~\cite{op:btxdoc} 
and \textit{Designing \BibTeX\ Styles}~\cite{op:btxhak}.  
The purpose of this chapter is to
familiarize you with the concepts of a bibliography database and to describe
many of the freely available tools for manipulating databases.  It is
not intended to replace any of the preceding documents.

\subsection{How \protect\BibTeX\ Works}

The \LaTeX\ command \verb|\cite| inserts 
citations\index{citations!in BibTeX@in \BibTeX} into your
document.\footnote{For Plain \TeX\ and other formats derived from Plain,
the \filename{btxmac.tex} macros provide these commands.}  
You use a short \textit{key}\index{key for bibliographic document} to 
identify the publication
you cite.  For example, in this book, when I want to
cite \textit{The \TeX{}book}~\cite{kn:texbook}, I use the command
\verb|\cite{kn:texbook}|.  The string \verb|kn:texbook| is the
key.  When you build your bibliography database, you assign a key
to each document in the database.  

When \LaTeX\ processes your document, it stores information about the
documents that you cite (including the key for each document) in
the \ext{AUX} file\index{aux files@.aux files}.  The following commands
identify how the bibliography should be formatted and what bibliography
databases contain the publications you cite.  \LaTeX\ also
writes this information to the \ext{AUX} file:

\begin{shortexample}
\bibliographystyle{abbrv}
\bibliography{texpubs}
\end{shortexample}

\BibTeX\ examines the \ext{AUX} file and extracts
the appropriate entries from the bibliography database.  It then formats
those entries according to the {\em bibliography style\/} that you specify.
The bibliography style (stored in a plain text file with the extension
\ext{BST})\index{bst files@.bst files} tells \BibTeX\ exactly 
how each entry should be formatted.
After formatting the entries, \BibTeX\ writes 
a \ext{BBL} file\index{bbl files@.bbl files} that
\LaTeX\ incorporates into your document (at the place where the 
\verb|\bibliography| command occurs) the next time 
you process it.

Sometimes it seems confusing to use \LaTeX\ to create a bibliography.
If \LaTeX\ keeps warning you about ``unknown citations'' for documents
that you {\em know\/} are in the database, try the following: run
\LaTeX, then \BibTeX, then \LaTeX, and then \LaTeX\ again.

The first time you run \LaTeX, it writes the citation keys to the 
\ext{AUX} file (but it doesn't know what publications they refer to).
\BibTeX\ writes the \ext{BBL} file, which includes the printable,
formatted bibliography and information about
what publication corresponds to each citation.  The second time
you run \LaTeX, it still doesn't know what publication
each citation refers to because it hasn't seen the \ext{BBL} file
yet (bibliographies are usually at the end of a document).  During
the second pass, \LaTeX\ writes the citation referents to the \ext{AUX}
file.  Finally, on the third \LaTeX\ pass, it knows what
each citation refers to, so it can typeset the citations as well
as the bibliography!
\vfill

\subsection{Building a Bibliography Database}

A bibliography 
database\index{database (bibliographic)}\index{bibliographies!building  a database} is 
a plain text file that contains information
about a collection of publications.  Bibliography databases generally
have the extension \filename{.bib}.  Example~\ref{ex:bib:bibexample}
is an example of a single database entry.  This entry is from a database
of \TeX-related publications that I put together while writing this book.

\begin{example}{ex:bib:bibexample}{A Sample \protect\BibTeX\ entry}
@book{kn:texbook,
  author =  "Donald E. Knuth",
  title =   "The {\TeX}book",
  publisher =   "Addison-Wesley",
  year =  1989,
  edition =   "Fifteenth",
}
\end{example}

The entry in Example~\ref{ex:bib:bibexample} describes {\it The {\TeX}book\/}
by \ixx{Donald Knuth}{Knuth, Donald}.  The key for 
this entry is \verb|kn:texbook|.

\begin{note}{NOTE}
All of the keys in a \BibTeX\ database must be unique.  If you use
multiple databases for a single document, all of the keys in all of the
databases must be unique.
\end{note}

In database jargon, the bibliography database contains a collection
of records describing publications.  There may be several
types of records in the same database.  Each record contains
several fields, some of which are required and some optional.  The
required fields vary according to the type of record.  

In English, this means that each entry in the database describes a
specific type of publication (book, article, technical report,
etc.).  Every publication is described by its characteristics.  For
example, books have a title, an author or editor, a publisher, and a
year of publication.  Some books also have a publisher's address, a
volume or number, a series, edition, or month of publication (or some
combination of these elements).  These characteristics are called
\textit{fields}, and they are identified by their name.

\subsubsection{Database entries}

This is the general 
structure\index{database (bibliographic)!structure}\index{bibliographies!entries} of 
an entry in a bibliography database:

\begin{ttindent}
\verb|@|\textit{type}\verb|{|\textit{key},
    \textit{field1} = "\textit{value1}",
    \textit{field2} = "\textit{value2}",
    . . .
    \textit{fieldn} = "\textit{valuen}"
\verb|}|
\end{ttindent}

\newpage
Always enter complete bibliography information in mixed
case.  Never abbreviate or set field values in all upper or all lowercase, 
even if the bibliography style that you most frequently use
specifies, for example, that book titles appear in uppercase or that only
the author's first initial should appear.  \BibTeX\ will take care
of formatting the entry according to the style.  If you store incomplete
information in the database, \BibTeX\ can't work correctly if you change
styles.

\subsubsection{Entry types}

Table~\ref{tab:bib:entries} shows the required and optional fields for
article and book entries.  Similar lists exist for the other standard
entry types: booklet, conference, inbook, incollection, inproceedings,
manual, mastersthesis, phdthesis, proceedings, techreport, unpublished,
and a catch-all miscellaneous type.

\begin{xtable}{l|l|l}
  \caption{Types of Entries with Required and Optional 
    Fields\label{tab:bib:entries}}\\
  \bf Type  & \bf Required Fields & \bf Optional Fields \\[2pt]
  \hline
  \tstrut
  article      & author          & month \\
                   & journal         & number \\
                   & title           & pages \\
                   & year            & volume \\
                   &                 & note \\[4pt]
  book         & author or editor & address \\
                   & publisher        & edition \\
                   & title            & month \\
                   & year             & volume or number \\
                   &                  & series \\
                   &                  & note \\
  \hline
\end{xtable}

Fields that are neither required nor optional are ignored.  Therefore,
you can and should associate arbitrary information about a
publication in its entry.  Abstracts and
keywords, for example, are two additional pieces of information that
you might keep for some publications.  They can be stored in 
\verb|abstract| and \verb|keyword| fields in each entry, even
though it is unlikely that they will ever occur in a bibliography.

\begin{note}{NOTE}
The types of records that are valid, and the required and optional
fields they contain are determined solely by the bibliography
style.  There is nothing in the \BibTeX\ program that makes 
\textit{book} and \textit{article} entries more legitimate than
\textit{reptile} or \textit{cartoon} entries.  
\end{note}

\newpage
\subsubsection{Abbreviations}

\index{abbreviations in bibliographic databases}The 
database entry structure that I've shown isn't entirely accurate.  
In my example, every field has a quoted string value.  The truth is, every
value is either a quoted string or an abbreviation or a number.  An
abbreviation is created with the \verb|@string| command.  Typically,
\verb|@string| commands are placed at the top of the bibliography
database.  For
example, the following command defines \verb|ora| to be an 
abbreviation for O'Reilly \& Associates.

\begin{ttindent}
\verb|@string{ora = "O'Reilly & Associates, Inc."}|
\end{ttindent}

The months of the year should always be specified with abbreviations
so that bibliography styles can redefine how they appear in the
bibliography (for this reason, three-letter abbreviations
for the months are defined in the standard styles---you don't have to
define them yourself).  The names of journals that you cite frequently are
also obvious candidates for abbreviation.

\subsubsection{Preamble}

\index{preambles!in bibliographic databases}Sometimes 
it is helpful to define \TeX\ control sequences in a
bibliography database.  \BibTeX\ provides a \verb|@preamble| entry for
this purpose.

Consider the following example, paraphrased
from \textit{\BibTeX{}ing}~\cite{op:btxdoc}: You have a database which contains entries
for each volume of a two-volume set by the same author.  It happens that
Volume One has been reprinted, so it has a more recent date than Volume Two.
The standard styles sort by author and then date, so as it stands, the
bibliography would list Volume Two before Volume One.

To correct this problem, you could specify the dates for Volume One as:

\begin{shortexample}
year = "{\noopsort{1990a}}1992"
\end{shortexample}

and for Volume Two:

\begin{shortexample}
year = "{\noopsort{1990b}}1990"
\end{shortexample}

\BibTeX\ will sort ``1990a'' before ``1990b,'' so they
will appear in the correct order, and the following definition for
\verb|\noopsort| will simply discard its argument, so nothing extra
will appear in the bibliography:\footnote{This example uses Plain \TeX\
syntax rather than a \LaTeX\ \texttt{\bs newcommand}, because \BibTeX\
databases can be used from Plain \TeX\ as well as \LaTeX.}

\begin{shortexample}
\def\noopsort#1{}
\end{shortexample}

The best place to put this definition is in the database that uses it
(so that it will always be present when that database is used).  The
\texttt{@preamble} command simply copies its argument to the top of the
\ext{BBL} file, so this definition at the top of the database will
do exactly what we want:

\begin{shortexample}
@preamble{"\def\noopsort#1{}"}
\end{shortexample}

\newpage
\subsubsection{Comments}

\index{comments!in bibliographic databases}Anything that 
does not appear inside a \verb|@|\texttt{\textit{type}}\verb|{}|
command is a comment. However, many programs that manipulate
bibliography databases will misplace comments appearing before or
after an entry if the entries are reordered.

\BibTeX\ includes a \verb|@comment| entry for backwards compatibility
with older systems.
Unlike \TeX, the percent sign (\%) is {\em not} a
comment character in \BibTeX.

\subsubsection{Special characters}

Inserting \TeX\ control sequences\index{characters!special!in bibliographic databases}\index{special characters!in bibliographic databases}\index{control sequences!in bibliographic databases} (to 
form accented characters, for example)
into a bibliography entry requires special care.  Some styles specify
that entries should be shifted to upper or lowercase, and shifting
the case control sequence names would make them different.

\BibTeX\ is aware of the case sensitivity of \TeX\ control sequence names
and will not change them.  To specify accents or other special characters,
always enclose them in \verb|{| and \verb|}| braces.  The same
treatment should be given to portions of an author's name that should
remain in lowercase even if the rest of the name is shifted to uppercase.

\subsection{Bibliography Styles}

\BibTeX\ styles\index{bibliographies!styles} are really 
programs written in a simple but powerful
stack-based language and interpreted by \BibTeX.\footnote{A complete
description of \BibTeX's programming language can be found in
\textit{Designing \BibTeX\ Styles}~\cite{op:btxhak}.}  Don't confuse
bibliography styles (\ext{BST} files)\index{bst files@.bst files} 
with \LaTeX\ styles (\ext{STY} files\index{STY files}); they are unrelated.  
Although a complete description of the \BibTeX\ language is not presented
here, a short example will help give you a sense of
the language.  

Each \ext{BST} file defines a number of functions. The highest
level functions determine how each entry is formatted: when \BibTeX\
needs to format a ``book'' entry, it executes the \verb|book| function,
which must be defined by the \ext{BST} file in use or an error will
result.

Let's consider part of the task of formatting a book entry.  These code
fragments are from the standard bibliography style \filename{plain.bst}.
When the \verb|book| function is ready to output the book title, it calls
the \verb|format.btitle| function, shown here:

\begin{shortexample}
FUNCTION {format.btitle}
{ title emphasize
}
\end{shortexample}

This function places the \textit{title} field on
top of the stack and calls \verb|emphasize|:

\begin{shortexample}
FUNCTION {emphasize}
{ duplicate$ empty$
    { pop$ "" }
    { "{\em " swap$ * "}" * }
  if$
}
\end{shortexample}

\newpage
This is what \verb|emphasize| will do when the \textit{title}
is not an empty field:

\begin{enumerate}
  \item Duplicate what is on the top of the stack.
  \item Test the value on the top of the stack.  The \verb|empty$| function
        removes the top value from the stack and places a boolean ``true'' 
        value there if what it removed was an empty field, 
        and a ``false'' value otherwise.
  \item Push \verb|{ pop$ "" }| onto the stack.
  \item Push \verb|{ "{\em " swap$ * "}" * }| onto the stack.
  \item Test the condition.  The \verb|if$| function takes three values
        from the stack: a boolean value, something to do if that value
        is true, and something to do if that value is false.  In this
        case, the title is not an empty field, so the value is false.
        The \verb|{ pop$ "" }| value is discarded, and 
        \verb|{ "{\em " swap$ * "}" * }| is evaluated.
  \item Push \verb|{\em|\ttspace\ onto the top of the stack.
  \item Swap the top two items on the stack.  Now \verb|\em|\ttspace\ is
        below the book title on the stack.
  \item Concatenate the top two items on the stack.  Now the top of the
        stack holds the value \texttt{\bs em\ttspace\textit{book title}}.
  \item Push \verb|}| onto the top of the stack.
  \item Concatenate the top two items on the stack.  Now the top of the
        stack holds the value \texttt{\ttopenbrace\bs em\ttspace\textit{book
        title}\ttclosebrace}.
\end{enumerate}

The resulting value left on top of the stack when \verb|emphasize| 
is finished is the \TeX\ code required to print the book title with
emphasis.

\subsubsection{Special-purpose styles}

A number of styles\index{bibliographies!special-purpose styles} have 
been written to allow you to develop special-purpose
bibliographies.  A few of them are listed here:

\begin{itemize}
  \item \filename{bibunits} supports multiple bibliographies in 
          different sections of a single document.
  \item \filename{chapterbib} supports separate bibliographies 
          in each chapter of a single document.
  \item \filename{makebst} asks a number of 
          questions about the bibliography style you need and constructs an
          appropriate \BibTeX\ style.
\end{itemize}

\subsection{Bibliography Database Tools}

\index{bibliographies!database tools}\index{database (bibliographic)!tools}There 
are a lot of programs designed to help you extract information
from bibliography databases, sort the entries,
build subset-bibliographies that contain only some of the entries
from a larger bibliography, and enter or edit information in an existing
or new database.
A lot of these programs are written in \Unix\ shell\index{shell!UNIX} script 
languages and
rely on existing text-processing tools like \program{awk},
\program{sed}, \program{sort}, and \program{grep}.  Of
course, many of these text-processing tools have been ported to other
operating systems.  Most of the utilities written in shell script
languages can be modified to work in other environments by an
ambitious individual.

\subsubsection{bibsort}

The \ixx{\program{bibsort}}{bibsort} shell script reorders the 
entries in a bibliography
database into alphabetical order by entry key name.  \texttt{@string} commands
are reordered by macro name.
This program cannot deal correctly with
comments that appear outside of an entry.  These comments are always
associated with the preceding entry, which is frequently incorrect.

Consider carefully before you reorder the entries in a database.  \BibTeX\
places some restrictions on the available orderings.  Cross references,
for example, must appear before the entry to which they refer.

\subsubsection{biblook}

\ixx{\program{biblook}}{biblook} is an interactive program that
searches rapidly through a bibliography database for key words in specified
fields.  Compound conditions (using ``and'' and ``or'') can be specified.
The entries located by a search can be saved into a separate file.

To use \program{biblook}, you must preprocess the database
with \ixx{\program{bibindex}}{bibindex}, which builds a binary index of
the entries.  Differences in case are removed, \TeX\
control sequences are stripped out, and non-alphanumeric characters
are removed.  This increases the likelihood of correct matches.

\subsubsection{bibclean}

\ixx{\program{bibclean}}{bibclean} is a syntax checker for bibliography
databases.  Running \program{bibclean} before using \program{bibsort},
\program{biblook}, or any other programs described here can eliminate a lot of
the problems these programs may encounter.  Although bibliography databases
are plain text files with a very loose structure, some tools are more easily
confused than others.

\program{bibclean} identifies possible problems in the database and
pretty-prints it in a standard way.  The following formatting changes
are made by \program{bibclean}:

\begin{itemize}
  \item The structure of each database entry is made consistent with 
        respect to the following criteria: the ``@'' sign that begins
        the entry type is moved into column 1; each line in the entry
        is changed to contain exactly one ``field = value'' pair; and
        the closing right brace is placed in column 1 on a line of
        its own.  Additionally, outer parentheses are converted into
        braces; tabs are expanded into blanks; hyphens in a sequence
        of pages are converted into en-dashes; and month names are converted
        into standard abbreviations.

  \item Long string values are split at a blank and continued on the following
        line.  The continuation lines are indented.

  \item Individual names in the author and editor fields are 
        normalized.  A single space is placed after periods that separate
        initials, and all entries are converted into ``first-name middle-name
        last-name'' form (instead of ``last-name, first-name'' form, for 
        example).

  \item The checksums of ISBN and ISSN numbers are verified.

\newpage
  \item Uppercase letters that appear outside of braces are enclosed in
        braces to prevent them from being erroneously shifted to lowercase
        by some bibliography styles.

  \item Text outside of entries is not changed.  Entries are separated by
        a single blank line.
\end{itemize}

\subsubsection{citetags}

\index{citetags}When 
documents must be transmitted in ``source'' form (meaning that the 
actual \TeX\ files will be shipped around), it's generally
unnecessary to include entire bibliography databases.  In these
cases, it would be more convenient to send only the bibliography
entries that are actually used.

The \program{citetags} program extracts the citations from an \ext{AUX}
file.  This list of citations can be passed to the \program{citefind}
program to build a small bibliography database of just the required entries.

\subsubsection{citefind}

\ixx{\program{citefind}}{citefind} processes a list of citations and a list of
bibliography databases and writes out a new database containing just the
entries required to match the citations present.  This provides a minimal
database that can be shipped with the document if the \TeX\ files must be
processed on another computer.

\subsubsection{bibextract}
\index{bibextract}

Given a list of fields and values (specified as regular expressions),
\program{bibextract} creates a new database containing only the entries
that match one or more of the specified values in one or more of the
specified fields.  The necessary \verb|@string| and \verb|@preamble| commands
are also included in the new database.

\subsubsection{lookbibtex}

\ixx{\program{lookbibtex}}{lookbibtex} offers the same features as
\program{bibextract}.  However, \program{lookbibtex} is written in
\program{Perl} rather than a shell script language.

\subsubsection{bibdestringify}

As the name implies, \ixx{\program{bibdestringify}}{bibdestringify} replaces
all \verb|@string| macros in a database by their textual expansions.

\subsubsection{edb}

\ixx{\program{edb}}{edb}, or the ``Emacs Database'' is a powerful database
programming system built on top of \ixx{GNU Emacs Lisp}{GNU Emacs!List for bibliographies}.  \program{edb} has been used to write a database-editing mode
for \BibTeX\ databases.  It can be extended to handle new entry types.

\subsubsection{Emacs \protect\BibTeX\ mode}

\BibTeX\ mode\index{bibtex@\BibTeX!mode in GNU Emacs}\index{GNU Emacs!BibTeX mode@\BibTeX\ mode} is a 
mode for editing \BibTeX\ databases in emacs.  It provides some template
expansion and alignment features, but is essentially a text-editing mode
(as opposed to \program{edb}, which is specifically a database-entry editor).

\newpage
\subsubsection{bibview}

\program{bibview}\index{bibview} is an X Window-based program for editing
bibliography databases.  Figure~\ref{fig:bib:bibview} shows an example of
\program{bibview} editing a database entry.

\epsimage{fig:bib:bibview}{Editing an article entry with \protect\program{bibview}}%
      
Unlike the \program{xbibtex} program, \program{bibview} can handle optional
and ignored fields in bibliography entries.
It does not handle new entry types.

\subsubsection{xbibtex}

\ixx{\program{xbibtex}}{xbibtex} is an X Window-based program for creating
bibliography databases.  Figure~\ref{fig:bib:xbibtex} shows an example of
\program{xbibtex} creating a bibliography entry.

\epsimage{fig:bib:xbibtex}{Editing an article entry with \protect\program{xbibtex}}

It does not appear that \program{xbibtex} can edit existing entries.
It also does not handle unexpected fields or
new entry types.

\subsubsection{bibdb}

The \ixx{\program{bibdb}}{bibdb} program is an MS-DOS-based bibliography
database editor.  The screen capture in Figure~\ref{fig:bib:bibdb} shows an
example of \program{bibdb} editing a bibliography entry.  \program{bibdb}
displays only fields that have values.

\epsimage{fig:bib:bibdb}{Editing an article entry with \protect\program{bibdb}}

Although it does not handle arbitrary fields, \program{bibdb} has a large
selection of optional fields.
It does not handle new entry types.

\section{Tib}

\ixx{\program{Tib}}{Tib} is another tool for maintaining bibliography databases.  The
format of a \program{Tib} database is the same as the format for the
\program{troff} \program{refer}
processor.\footnote{\textit{refer} databases can reportedly be
converted to and from \BibTeX\ format.}  An example entry for
\textit{The \TeX{}book} is shown in Example~\ref{ex:tibdb}.

\begin{example}{ex:tibdb}{A Tib style database entry}
%A Donald E. Knuth
%T The \TeX{}book
%I Addison-Wesley
%D 1989
\end{example}

Unlike \BibTeX, which relies on the citation macros to 
format citations correctly, \program{Tib} actually replaces the 
citations\index{citations!in Tib@in \Tib} with 
the appropriate information.  In other words, \program{Tib}-style citations
are not control sequences; they are just text strings.  The \program{Tib} 
processor creates an entirely new document file that should be passed
to \TeX.

The general format of a \program{Tib} citation is
\verb|[.|\texttt{\textit{citation key(s)}}\verb|.]|.
\program{Tib} databases do not contain a key field; instead,
the citation keys can come from any fields in the
database entry (you can exclude some fields
if you wish).
The punctuation around citations in square brackets is adjusted by
\program{Tib}.  An alternate form of citation using angle brackets inserts
a citation without adjusting any of the surrounding punctuation.

The bibliography is inserted into your document wherever the string
``\verb|.[]|'' occurs at the beginning of a line.  Analogous to \BibTeX,
the format of the reference list is controlled by a \program{Tib} style.
Styles for roughly fifteen technical journals are provided.

\section{Making Indexes}

Constructing an index\index{indexes}\index{tex@\TeX!indexes} for 
a \TeX\ document is relatively straightforward.
First, you must identify each occurrence of each word or concept that you want
indexed.  To do this, you must insert \verb|\index| entries into your
document.\footnote{The format you use must also provide support for
indexing.  In \LaTeX\index{latex@\LaTeX!indexes}, most of the 
support is built in, but you must use the
\idx{makeidx style option}.  In Plain \TeX\ and other Plain-derived
formats, input the \filename{idxmac.tex} macros at the top of your document.
The rest of this section assumes that you use \LaTeX, although the
principles hold for any macro package that supports indexing.}  When you
format your document containing the 
index entries, an \ext{IDX} file\index{IDX files} is
created that contains all of the entries along with the page number where each
entry occurred.

The \program{MakeIndex}\index{MakeIndex!for indexes} 
program\footnote{Frequently called \filename{makeidx} 
on MS-DOS systems.} reads the \ext{IDX} file, sorts and collates the entries,
and writes an \ext{IND} file\index{IND files}.  The \program{MakeIndex} 
program can also 
load an index style file, discussed in the next section.

When \LaTeX\ processes your document, it inserts
the contents of the \ext{IND} file into your document at the point where you
use the \verb|\printindex| control sequence.

\subsection{Index Entries}

Index entries\index{indexes!entries}\index{special characters!in indexes}\index{characters!special!in indexes} use special 
characters to identify different types of entries
(simple entries, multiple-level entries, see-also's, etc.).  The exact
characters used are controlled by the index style file, and it may be
convenient to select an alternate style.  For example,
\program{MakeIndex} uses the double-quote character by default to identify literal
characters in the index entry, but the German\index{German} Babel style makes the literal
double quote character a shortcut for the umlaut accent; this use makes it
unavailable as the quotation character in index entries.

Another point to consider when coding index entries is that the arguments
to the index macros are expanded by \TeX.  This makes it difficult to insert
some control sequences\index{control sequences!in indexes} in 
an index entry (for example \verb|\index{TeX@\TeX}|).
In \LaTeX, you can combat this problem by using the \verb|\protect| control
sequence: \verb|\index{TeX@\protect\TeX}|.

A complete list of the different kinds of index entries that can be
created is included in \textit{MakeIndex: An Index Processor for 
\LaTeX}~\cite{ll:makeindex}.

%\subsubsection{Simple index entries}
%
%The simplest index entries identify a single top-level term.  This form
%of entry is simply:
%
%\begin{ttindent}
%\bs index\{\textit{frogs}\}
%\end{ttindent}
%
%\subsubsection{Multi-level entries}
%
%To mark second and third level indexing terms, separate the levels with
%an exclamation point:
%
%\begin{ttindent}
%\bs index\{\textit{amphibians}!\textit{frogs}\}
%\end{ttindent}
%
%This will create a top level heading called ``pets'' with a subheading
%for ``dogs''.   To create deeper nesting, use more exclamation points:
%
%\begin{ttindent}
%\bs index\{\textit{pets}!\textit{amphibians}!\textit{frogs}\}
%\end{ttindent}
%
%\subsubsection{Page ranges}
%
%To identify an entry that spans a range of pages, simply mark the ends of
%the range.  To mark the beginning of the range, use:
%
%\begin{ttindent}
%\bs index\{\textit{amphibians}|(\}
%\end{ttindent}
%
%and to mark the end:
%
%\begin{ttindent}
%\bs index\{\textit{amphibians}|)\}
%\end{ttindent}
%
%Both ends of the range should fall on pages with the same numbering
%scheme (i.e. roman, arabic, etc.).
%
%\subsubsection{Cross references}
%
%To make a cross-reference entry, use the 

\subsection{Index Format}
\label{sec:indexeslooklike}

The format of the index is controlled by both the index style (which
specifies whether or not headings should be present, what the delimiters
should be, etc.) and the definition of several control sequences:\footnote{In
\LaTeX, these control sequences have default definitions that you may not
need to change.}

\begin{iplist}[\textnormal]{.25in}
  \ipitem[\bs\texttt{theindex}] Controls what happens just before the index
    is printed.  This sequence should establish a new page, set up 
    running headers and footers, select multiple columns, and do whatever
    other global setup is desired.
  \ipitem[\bs\texttt{item}] Executed just before a first-level entry.
  \ipitem[\bs\texttt{subitem}] Executed just before a second-level entry.
  \ipitem[\bs\texttt{subsubitem}] Executed just before a third-level entry.
  \ipitem[\bs\texttt{indexspace}] Executed between alphabetical sections.
  \ipitem[\bs\texttt{endtheindex}] Controls what happens just after the index
   is printed.  This sequence should undo anything that was started by
   \verb|\theindex|.
\end{iplist}

\subsection{Special-purpose styles}

A number of styles
%\indexes{special-purpose styles}
have been written 
to allow you to develop special-purpose
indexes.  \program{multind} and \program{index} are described here.

\filename{multind} supports multiple indexes in a single document.

\filename{index} is a reimplementation of the \LaTeX\ indexing
  commands.  It
  stores index entries in the \ext{AUX} files so that portions of 
  a document can be reformatted without losing the entire index.
  It also provides support for multiple indexes and replaces the
  \textit{makeidx} style option.

\section{Making Glossaries} 

\index{glossaries}\index{tex@\TeX!glossaries}The \LaTeX\ 
command \verb|\glossary| can be used to accumulate words
and concepts for a document glossary.  The output, stored in a
\ext{GLO} file\index{GLO files}, is very similar 
to an \ext{IDX} file\index{IDX files}.  Like an index,
the glossary file can be processed by \program{MakeIndex}\index{MakeIndex!for glossaries} to produce
a sorted list of terms.  To produce a glossary, you must 
create an index style for the glossary.  Here is a minimal 
\filename{glossary.ist}:

\begin{shortexample}
keyword "\\glossaryentry"
preamble "\n\\begin{theglossary}\n"
postamble "\n\\end{theglossary}\n"
\end{shortexample}

The output file that \program{MakeIndex} creates will have to be edited
by hand, naturally, in order to incorporate the definitions of the entries.
You will also have to define an appropriate glossary environment by 
defining the control sequences\index{control sequences!for glossaries} used to make the glossary:

\begin{iplist}[\textnormal]{.25in}
  \ipitem[\bs\texttt{theglossary}] Controls what happens just before the glossary
    is printed.  This sequence should establish a new page, set up 
    running headers and footers, and do whatever
    other global setup is desired.
  \ipitem[\bs\texttt{item}] Executed just before a glossary item.
  \ipitem[\bs\texttt{endtheglossary}] Controls what happens just after the glossary
   is printed.  This sequence should undo anything that was started by
   \verb|\theglossary|.
\end{iplist}

Finally, you will have to \verb|\input| or \verb|\include| the glossary in
your document.

