\chapter[Online Documentation]{Online\\Documentation}
\RCSID$Id: ch10.tex,v 1.1 2002/08/23 14:58:46 nwalsh Exp $
\label{chap:online}

\ifincludechapter\else\endinput\fi

Some common kinds of documentation---for example, manuals for computer 
programs---present a unique challenge to the author.  In many cases, it 
would be nice to be able to provide online documentation\index{online documentation}\index{documentation (online)} in addition to a typeset manual.

One option is to maintain two different documents: one for publication
and one for online access.  This is difficult to maintain and is prone
to error.  As the documentation evolves, it is almost inevitable that
some changes to one document will not be implemented in the other.

Another option is to include only very limited formatting information
in the document designed for publication so that it is easy
to ``strip out'' the formatting commands and produce an online manual.
The unfortunate side effect of this approach is that the resulting
typeset documentation doesn't have a very professional appearance.

With care, \TeX\ can be the basis for a middle-ground approach to this 
problem.\footnote{Structured
markup languages like SGML also address this problem, but they introduce
their own set of difficulties.  Regardless of whatever advantages they
may hold, I'm not going to discuss them here.}  
If you are starting a new documentation project,
\ixx{\TeXinfo}{texinfo@\TeXinfo} and \ixx{Lame\TeX}{Lametex@Lame\TeX} provide two alternatives for the production of
typeset and online documentation from the same source.  \ixx{\LaTeX2HTML}{latex2HTML@\LaTeX2HTML} and
\ixx{\LaTeX2hy}{latex2hy} provide alternatives that may be suitable for existing 
documentation.

You'll find that the best results occur when you plan ahead: if you know
that you need both typeset and ASCII documentation, try to use tools
that will make the task easier.  But, even if you try to plan ahead,
it's not uncommon to find out after the fact that you need or want
ASCII documentation.  If you don't have the \TeX\ sources, you'll just
have to take the best results you can get with one of the tools described
later in this chapter
and do whatever hand-editing is required.  

If you have the \TeX\ sources for a document, 
here are some guidelines that can help improve the quality of
the conversion to plain ASCII: 

\begin{itemize}
  \item Redefine all font commands to use \cs{tt}.  This makes \TeX\
        work with a fixed-width font.  You will probably get many, many
        overfull and underfull box messages.  This can't be helped.  Setting
        \cs{tolerance} and \cs{hbadness} to large values
        (10000 with \TeX\ 2.x or 100000 with \TeX\ 3.x, for example) will
        reduce the number of warnings.

        Similarly, redefine all the commands that change font size to select
        the same size (probably 10pt or 12pt will work best, but larger
        values may be better if you have wide margins).

  \item Don't use any special fonts---no \verb|picture| environments in \LaTeX,
        for example.  Take out rules, too.

  \item Use \cs{raggedright}.  There's no point in trying to line
        up the right margin.

  \item Remove or redefine all mathematics to avoid the use of math-mode.
        It won't work; don't ask \TeX\ to try.

  \item Remove all tables (\cs{halign} in Plain \TeX, \verb|tabular|
        environments in \LaTeX).
 
  \item Remove floating environments; this may help.

  \item Depending on your level of expertise and the number of documents
        that you have to convert, redefine footnotes and other environments
        to give you more marginal improvements.
\end{itemize}

The \filename{ascii.sty} style\index{ascii.sty style files} for 
\LaTeX\ encapsulates many of these rules for you.

\section{Something Is Lost}

Many things that can easily be represented
on paper cannot be represented in plain ASCII.  One reason for this
is that plain ASCII output is not proportionally spaced.  Also, in ASCII
you can move up or down only by
rows, and left or right only by columns; you can't move down ``3 points'' 
on a terminal to typeset a subscript, for example.

These differences combine to make many things impossible.  For all of
its marvelous sophistication, \TeX\ cannot help you 
typeset mathematics\index{mathematics!in ASCII} in 
plain ASCII.  It just can't be done.  Most tables can't be done in
plain ASCII either (at least not if you want lines that are only 80 characters
or so long).

The following sections describe tools that may help you achieve the
goal of online and typeset documentation from the same sources.  Each
tool has its own advantages and disadvantages. The ones that work best
for you will depend on the type of documentation you are producing
and the amount of work you are willing to do.

\section{\protect\TeXinfo}

\ixx{\TeXinfo}{texinfo@\TeXinfo} is the document formatting system adopted by the
\idx{Free Software Foundation (FSF)} and the \idx{GNU Project}.  It is 
a special \TeX\ format
that is very different from standard \TeX.\footnote{A \LaTeX\
implementation, called \LaTeX{}info, is also available.}  The goal of
\TeXinfo\ is to devise an input format that can be processed by
\TeX\ to produce typeset output and then be processed by another program to
produce \emph{hypertext} output.  (The other program in this case is
\ixx{\program{MakeInfo}}{MakeInfo}.)

\TeXinfo\ supports ordinary text, sectioning commands, itemize and
enumeration environments, footnotes, cross-references, tables of
contents, lists of figures and tables, and multiple indexes.

The \TeXinfo\ example from Chapter~\ref{chap:macpack},
{\it \nameref{chap:macpack}}, is reproduced in Example~\ref{ex:on:infinp}
(the \TeXinfo\ input), Figure~\ref{fig:on:infex} (the typeset page), 
and Figure~\ref{fig:on:infinf} (the resulting online documentation).

\exampleinput{perf-inf}{ex:on:infinp}{\protect\TeX{}info Commands}

\epspage{perf-inf.eps}{3.25in}{.49,3.075}{\protect\TeX{}info sample page}{fig:on:infex}

\begin{figure}
  \begin{center}
    \FramedVerbInput{perf-inf.inf}
    \caption{Online documentation produced by MakeInfo}
    \label{fig:on:infinf}
  \end{center}
\end{figure}

The output from \program{MakeInfo} is very nearly pure ASCII.  The
motivation for {hypertext} output is to make cross 
references dynamic\index{cross references!dynamic} when 
the ``info'' version of the document is used for online
reference.\footnote{``Info'' is the name of both the output format and a
program for displaying the text in a hypertext fashion.  Another common
way to access info files is with the GNU emacs online help system.}  
The result is close enough to pure ASCII that converting
it to pure ASCII is not (usually) too difficult.  For example, the
\ixx{\textit{comp.fonts} newsgroup}{comp.fonts newsgroup}'s Frequently Asked Questions list is 
maintained as a \TeXinfo\ document. It is 
posted as an info version that has been processed
by a Perl script to ``flatten'' the hypertext.

The \TeXinfo\ format is well documented, so a brief description here
will suffice.  First, the backslash, which is ordinarily
used to introduce control sequences in \TeX, is not special.  Instead
the ``@''-sign is used.  Second, a
\TeXinfo\ document is divided into ``\idx{nodes}.''  A node corresponds
roughly to a chapter or a large section of a chapter.  In the online
documentation, it is easy to jump between related nodes. 

Because the info version of the document is ASCII text, many
of the special typesetting features of \TeX\ aren't applicable.
To support them in the typeset document, \TeXinfo\ allows you to
specify that some portions of the input should be seen only by \TeX\
and some should be seen only by \program{MakeInfo}.
Example~\ref{ex:on:infinp} uses this feature to typeset
mathematics using the best features of both \TeX\ and \program{MakeInfo}.

\section{\protect\LaTeX2HTML}

\ixx{\LaTeX2HTML}{latex2HTML@\LaTeX2HTML} attempts 
to convert \LaTeX\ documents into \idx{HTML}, the
document structuring language used by the \idx{World Wide Web (WWW)} project.
HTML stands for HyperText Markup 
Language\index{HyperText Markup Language (HTML)}; it is a way of describing
documents in terms of their structure (headings, paragraphs, lists,
etc.).  
\idx{SGML}, the Standard Generalized Markup Language,
provides
a framework for developing structured documentation; HTML is one specific
SGML document type.
HTML documents are displayed by special programs called
\idx{browsers} that interpret the markup and present the information in a
consistent manner.  Because an HTML document is described in
terms of its structure and not its appearance, HTML documents can be
effectively displayed by browsers in non-graphical environments.

\newpage
One of the most important features of HTML documents is the ability to
form hypertext links\index{links!hypertext}\index{hypertext links} between 
documents.  Hypertext links allow you to
build dynamic relationships between documents.  For example, selecting
a marked word or phrase in the current document displays more
information about the topic, or a list of related topics.  

\LaTeX2HTML preserves many of the features of a \LaTeX\ document in 
HTML.  Elements that are too complex to represent in HTML, such
as mathematical equations and logos like ``\TeX,'' are converted into
graphic images that can be displayed online by graphical browsers.
All types of cross referencing elements (including footnotes) are preserved as
hypertext links.

When installed, \LaTeX2HTML understands many basic \LaTeX\ commands, but
it can be customized to handle other styles.  \LaTeX2HTML is written in
\program{Perl}.

All in all, \LaTeX2HTML is one of the easiest and most effective tools
for translating typeset documentation into a format suitable for online
presentation.  In a graphical environment like X11, Microsoft Windows,
or the Macintosh, HTML documents offer very good support for online
documentation.

\section{Lame\protect\TeX}

\ixx{Lame\TeX}{Lametex@Lame\TeX} is a \ixx{PostScript translator}{PostScript!translator} for 
a (very limited) subset of
\LaTeX.  One of its original design goals, the inclusion of sophisticated
PostScript commands directly in a \LaTeX\ document, has been superseded
by the \ixx{\PSTricks}{PSTricks@\PSTricks} package.
However, one of the side effects of a special-purpose translator for
\LaTeX\ is the ability of that translator to produce different kinds of
output, including plain ASCII.

The primary advantage of this method is that it does not require learning
an entirely foreign macro package like \TeXinfo.  The disadvantage
is that it understands only a very small subset of \LaTeX.
This subset includes \emph{only} the following commands: 

\begin{xtable}{lll}
\cs{\#} & \cs{footnotesize}     & \cs{ref} \\
\cs{\$} & \cs{hspace}   & \cs{rm} \\
\cs{\%} & \cs{hspace*}  & \cs{sc} \\
\cs{\&} & \cs{huge}             & \cs{scriptsize} \\
\cs{Huge}       & \cs{include}  & \cs{section} \\
\cs{LARGE}      & \cs{input}            & \cs{section*} \\
\cs{Large}      & \cs{it}               & \cs{setlength} \\
\cs{\_} & \cs{item}             & \cs{sf} \\
\cs{addtolength}        & \cs{itemize} & \cs{sl} \\
\cs{backslash}  & \cs{label}    & \cs{small}\cs{smallskip} \\
\cs{begin}      & \cs{large}            & \cs{subparagraph} \\
\cs{bf} & \cs{ldots}            & \cs{subparagraph*} \\
\cs{bigskip}    & \cs{medskip}  & \cs{subsection} \\
\cs{center}     & \cs{newlength}        & \cs{subsection*} \\
\cs{chapter}    & \cs{newline}  & \cs{subsubsection} \\
\cs{chapter*}   & \cs{normalsize}       & \cs{subsubsection*} \\
\cs{clearpage}  & \cs{par}              & \cs{tiny} \\
\cs{description}        & \cs{paragraph}        & \cs{today} \\
\cs{document}   & \cs{paragraph*}       & \cs{tt} \\
\cs{documentstyle}      & \cs{part}             & \cs{verbatim} \\
\cs{em}         & \cs{part*}            & \cs{verbatim*} \\
\cs{end}                & \cs{quotation}        & \cs{verse} \\
\cs{enumerate}  & \cs{quote}            & \cs{vspace} \\
\cs{flushleft}  & \cs{raggedleft}       & \cs{vspace*} \\
\cs{flushright} & \cs{raggedright}      & \\
\end{xtable}

In addition, unlike \LaTeX, Lame\TeX\ doesn't
understand any Plain \TeX\ commands (other than the ones listed).

\section{latex2hy}

\ixx{\program{latex2hy}}{latex2hy} is 
a \LaTeX-to-ASCII converter\index{latex@\LaTeX!converter to ASCII}.  It has several
options for controlling the input and output character sets.  In
addition, \program{latex2hy} has a number of options for improving the
quality of both ASCII and printed documentation.  For example, input
documents can contain both \TeX\ and ASCII representations for complex
objects (like mathematical formulae).  The printed documentation uses
the \TeX\ version while \program{latex2hy} uses the ASCII version.
Provision is also made for ``fixups,'' which allow character sequences
from the input text to be translated into different sequences on
output.  For example, ``$9.81\frac{m}{s^2}$'' can be automatically
translated into \verb|9.81m/s^2|.  By adding specific fixups to each
document that you translate, you can obtain successively better 
approximations automatically.

\program{latex2hy} gives particular attention to 
cross references\index{cross references!in latex2hy}.
Cross references are translated into ``\idx{links}''
between topics in \emph{hypertext} output formats.  Currently, only the
\ixx{\emph{TurboVision}}{TurboVision} hypertext format is supported, although several
other formats are being considered.

\section{detex}

\ixx{\program{detex}}{detex} is a simple program that does little more than strip
control sequences and other \TeX{}ish character sequences from your document.
Doing this makes the document more amenable to other kinds of processing (like
spellchecking) but does a poor job of producing ``online documentation.''
Still, it's an option.

\section{dvispell}

The \ixx{\program{dvispell}}{dvispell} program produces plain text output from a \ext{DVI}
file.  It is not a spellchecker, but it was designed to extract the
words from a \TeX\ document which were then fed to a spellchecking
program that was unable to ignore \TeX\ control sequences.

The \program{dvispell} program is part of the \ixx{\emTeX}{emTeX@\emTeX} package, and it is
remarkably sophisticated.  There are many ways in which \program{dvispell}
can be programmed to perform complex manipulations.  One special strength
of \program{dvispell} is its ability to handle conversion of accented 
characters and Greek symbols.

\program{dvispell} differs from the other programs described here
because it works with \TeX\ output, the \ext{DVI} file.  On the one
hand, this provides \program{dvispell} with more information
(character positions, line breaks, floating bodies, etc).  On the
other hand, all of the formatting commands are missing, so it isn't
easy to determine what the user had in mind. (It's a bit like
reverse-engineering a piece of software---without the source code,
it's not always easy to tell why things look the way they do.)  But
\program{dvispell} gives you access to most of the information that is
present in the \ext{DVI} file.
