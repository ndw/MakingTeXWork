\chapnumkern=-1pc
\chapter[Introducing \protect\MF]{%
        Introducing\\ \protect\MF} 
\RCSID$Id: ch11.tex,v 1.1 2002/08/23 14:58:46 nwalsh Exp $
\label{chap:mf}

\ifincludechapter\else\endinput\fi

\TeX\ builds pages out of \idx{boxes} and \idx{glue}.  The smallest boxes 
are usually
individual characters.  \ixx{\MF}{metafont@\MF} is a tool that creates the actual characters.

In the last few years, a real explosion has taken place in the number 
of readily available fonts\index{fonts!building with metafont@building with \MF}\index{characters!as boxes}\index{metafont@\MF!building fonts}.  
Today, many people have access to a large number
of high quality typefaces.  Not too long ago (before \idx{Adobe Type Manager (ATM)} 
was available for the Macintosh and Microsoft Windows), fonts were too expensive
to be generally available.  In those days, a font building tool like \MF\ was
essential if \TeX\ was going to have a number of typefaces and a large number 
of mathematical symbols.  Today, the role of \MF\ is diminishing.  Many people
choose to use PostScript fonts\index{fonts!PostScript} almost 
entirely.  In fact, with PostScript
alternatives to the \idx{Computer Modern Math fonts}\index{fonts!Computer Modern} now available, it's possible to
use \TeX\ without using \MF\ fonts at all.

On the other hand, lots of people do still use the Computer Modern
fonts, and there are many other \MF\ fonts (for non-English languages
like Arabic and Japanese, for example), so \MF\ is still a useful tool.
Also, it can be used to make nice portable diagrams.

This chapter cannot teach you how to harness all of \MF's power.  The only
practical way to learn how to design your own images with \MF\ is to 
read {\it The \MF{}book}~\cite{kn:mfbook}  (even if your images
are nowhere near as complex as an entire font).

\newpage
What you will learn from this chapter is how to run \MF\ to create 
\TeX\ \ixx{\ext{PK} fonts}{PK fonts@\ext{PK} fonts!made with metafont@made with \MF} 
and \ixx{\ext{TFM} files}{TFM files!making with metafont@making with \MF} from 
existing \MF\ programs.  
The Computer
Modern fonts, the \AmS\ fonts, and the DC Fonts are all created
with \MF.  It is not unreasonable to imagine that you might someday
want to create one of these fonts at a non-standard size or unusual
magnification.  

Knowing how to run \MF\ will also help you set up a system for
performing automatic font generation.  This can save a lot
of disk space.  See the section called ``\nameref{sec:autofont}'' in
Chapter~\ref{chap:fonts}, {\it \nameref{chap:fonts}}, for more information
about setting up automatic font generation.

From many perspectives, \TeX\ and \MF\ behave in analogous ways.  
Where \TeX\ reads a plain text \ext{TEX} document, processes it, and
produces a \ext{DVI} file, \MF\ reads a
plain text \ext{MF} program, processes it, and produces a generic
bitmap font file as output. 

Because you probably aren't going to be writing your own font programs for \MF,
you have to get them from somewhere.  Many distributions of \TeX\
include \MF\ and the font files for the Computer Modern fonts.
This chapter will concentrate on just the Computer Modern font programs
because those are the most likely to be available.  

Many \MF\ fonts can be retrieved from the CTAN archives.  
Table~\ref{tab:mf:mfdistrib}
lists some common fonts and their location in 
the CTAN archives.  A collection of font samples is shown in
Appendix~\ref{app:fonts}, {\it\nameref{app:fonts}}.

\begin{xtable}{l|l}
  \caption{Some Popular \protect\MF\ Fonts on the CTAN 
    Archives\label{tab:mf:mfdistrib}}\\
  \bf Fonts                   & \bf Location \\[2pt]
  \hline
  \tstrut
  Standard Computer Modern    & \it fonts/cm \\
  Standard \LaTeX\            & \it macros/latex/distribs/base/fonts \\
  \AmS\ Fonts                 & \it fonts/ams \\
  DC Fonts                    & \it fonts/dc \\
  Concrete                    & \it fonts/concrete \\
  Pandora                     & \it fonts/pandora \\
  Ralph Smith's Script Font   & \it fonts/rsfs \\
  St. Mary's Road             & \it fonts/stmary \\[2pt]
  \hline
\end{xtable}

\section{What to Run?}

Like \TeX\ itself, the actual \MF\ file you have to execute
\index{metafont@\MF!running} varies between
platforms and implementations.  If you have built and/or installed
\TeX, you probably already know what program to run.  
You'll have to ask your system administrator for help if you cannot
figure out what the name of the \MF\ executable is on your computer.
In the rest
\linebreak
\newpage
of this chapter, I'll assume that the command \textit{mf}
runs \MF.  You should substitute the name of the executable
program on your system for \textit{mf} in the examples that follow.

\MF\ comes in big and small versions just like \TeX.  You will
need the big \MF\ if you are building fonts for very high-resolution
devices or at very large sizes.  Most fonts can be built with a small
\MF.

\section{What Files Does \protect\MF\ Need?}

In addition to the \MF\ programs, \MF\index{metafont@\MF!files} must 
be able to find several
other files.  The files that are needed are normally created during the
installation process.  The sections that follow describe each of the
files that \MF\ needs.

\subsection{Pool Files}

The pool file\index{pool files!for metafont@for \MF} for each 
version of \MF\ has the same purpose as the
pool file created for \TeX.  It is created when \MF\ is compiled.
If you don't have a pool file, there's nothing you can do about it.
If you obtained precompiled programs (from the Internet, from a
friend, or commercially) and you don't have the pool file needed by
\MF, you received an incomplete distribution.

If you did not install \MF, and the pool file is missing, contact the
system administrator who performed the installation.  Something was
done incorrectly.

\subsection{Base Files}

Base files\index{base files!for metafont@for \MF} for \MF\ are 
analogous to format files for \TeX.
Unlike \TeX, which has many different format files, there are
relatively few base files for \MF.  

Base files are created by a special version of \MF, usually called
\iniMF.  However, some implementations combine \MF\ and \iniMF\
into one program, in which case you must select \iniMF\ with a special
option when you run the program.  

Like ini\TeX, \iniMF\ interprets all the control sequences in a set of
base macros and builds the in-memory data structures that the \MF\
program needs.  After loading all the files, \iniMF\ writes
the memory image into a base file.  When \MF\
loads the base file, it simply copies it into memory; no
interpretation is necessary.  

Because \MF\ loads the base files without interpretation, they are not
generally portable from one system to another, or even between
different versions of \MF\ on the same system.  Different versions of
\MF\ load differently in memory, and this makes the base files
incompatible.  For example, you need a big \iniMF\ to make
base files for big \MF\ and small \iniMF\ for a small \MF.

The ``\nameref{sec:mf:gettingstarted}'' section later in this chapter 
describes how to build a base file, if you do not already have one.

\subsection{\protect\MF\ Programs}

When you run \MF, you have to tell it what 
file (\MF\ program)\index{metafont@\MF!programs} to
process.  \MF\ programs are stored in files with the extension \filename{.mf}.
If you specify a complete pathname, \MF\ will load the
specific file that you request.  If you specify a simple filename
without a path, \MF\ looks for the file in several user-defined and,
possibly, system-defined locations.  The most common way to specify
user-defined locations is by setting the \verb|MFINPUTS|
environment variable to a list of subdirectories where \MF\ programs
are kept.  The format of the environment variable differs according to
the platform you use.  On \Unix\ systems, it is 
a list of directory names separated by colons.  On MS-DOS and OS/2 systems, 
it is a
list of directory names separated by semicolons.  Consult the
documentation for the particular implementation of \MF\ you use
for more information about system-defined locations where \MF\
looks for input files.

\section{Command-line Options}

In addition to the name of the \MF\ program, \MF\ has very few 
command-line\index{command line!metafont options@\MF\ options}\index{command line!specifying filenames!for metafont@for \MF}
options.  The name of a base file is the only option regularly 
used.
In this section, I'll explain what can go on the \MF\ command
line.

A formal specification of the \MF\ command line looks like this:

\begin{ttindent}
\$ \textbf{mf \textit{switches} \textbf{&}\textit{base} mf-program mf-commands}
\end{ttindent}

After the name of the \MF\ program, the first things that you can specify
on the command line are implementation-dependent switches and options.  For
example, implementations of \MF\ that combine \iniMF\ and \MF\ into a
single program may use the switch \textit{/I} to specify that \iniMF\
processing is desired.  There are no system-independent switches for \MF.
Consult the documentation that comes with your implementation for more
information about system-dependent switches.

After any system-dependent switches, you can specify
the name of the base file to use.  This option,
if present, must come before any other options, and you must put an
ampersand (\&) in front of the base file name.  If you do not specify a
base file, \MF\ will use a default base.

After the base, \MF\ looks for the name of a \MF\ program file.  If \MF\
finds a filename on the command line, it will process the program contained
in that file before looking at any other options that may follow.

Finally, you can insert arbitrary \MF\ commands on the command line by
typing them just as you would in a program.  In fact, this is
frequently done to change the size or magnification of the font being
created.  Exactly how this is accomplished is described in the
section called ``\nameref{sec:buildfont}'' later in this chapter.

Please read the section called ``\nameref{sec:runtex:cmdcautions}''
in Chapter~\ref{chap:running}, {\it \nameref{chap:running}}, for a
detailed description of some common problems with \MF\ commands on the
command line.  They are exactly the same problems that can occur on
the \TeX\ command line.

Also, the behavior of \MF\ when run without any options at all is the 
same as the behavior of \TeX\ as described in the section called
``\nameref{sec:runtex:texwoopts}'' in Chapter~\ref{chap:running}.

\section{Building a Base File}
\label{sec:mf:gettingstarted}

Building a \ixx{base file}{base files!building} for \MF\ is 
exactly analogous to building
a format file for \TeX.  Fortunately, there are far fewer base files.
The two most common base files are the \textit{plain}
\index{plain base files} base and the 
\textit{cmbase}\index{cmbase base files} (the
base used for the Computer Modern fonts).  Use the \iniMF\ program\index{iniMF building base file} to build a base file.  For 
example, to build the plain base,
use:

\begin{ttindent}
\$ \textbf{inimf plain}
\end{ttindent}

Move the resulting files, \filename{plain.base}\index{plain.base files!for metafont@for \MF} and \filename{plain.log}\index{plainlog files@plain.log files!for metafont@for \MF},
into the directory where \MF\ looks for base files (typically in a 
\filename{bases} or \filename{formats} subdirectory under the standard
\TeX\ directory).

The \textit{cmbase} is actually an extension of the plain base.  To build
it, use the command:

\begin{ttindent}
\$ \textbf{inimf &plain cmbase}
\end{ttindent}

Remember to use appropriate quotation or shell escape characters to prevent
the ampersand from being misinterpreted.

\section{Running \protect\MF}

\index{metafont@\MF!running}\index{running metafont@running \MF}The sections that 
follow describe how to use \MF\ to make a font from
an existing \ext{MF} file.  You should reread the sections
``\nameref{sec:issueofsize}'' from Chapter~\ref{chap:fonts},
{\it \nameref{chap:fonts}}, and ``\nameref{sec:bitfonts}'' from
Chapter~\ref{chap:printing}, {\it \nameref{chap:printing}\/},
if you are unfamiliar with the notions of
bitmap and device resolution and magnification and design size in
\TeX\ fonts.

\subsection{Picking a Mode}

The input to \MF\index{metafont@\MF!picking a mode} is 
an ``outline'' description of a font.  The output is
a set of bitmaps that realize the outline on a particular device
at a particular size.  

Output from \MF\ is naturally device-dependent because different
printers have different resolutions.  But even two printers with
the same resolution sometimes produce different looking results from the 
same bit patterns.
For example, some laser
printers are ``write white'' printers, and some are ``write black''
printers---depending on whether the printer fills in the black (text)
areas or the white (non-text) areas of the page.  A small, intricate bitmap
(like a character from a font) designed for a write-black printer
may not look as good on a write-white printer and vice versa.

There are several internal parameters that can be modified to produce
optimal output on any given device.  These parameters are grouped together
in a \emph{mode}.  Whenever you run \MF\ to create a new set of bitmaps, 
you must select a mode. 

There are no rules for determining what parameters produce the best
mode for a particular printer; it's just done by trial and error (set
some values, print a font, see if it looks good, adjust some values,
print another font, etc.).

Luckily, a large collection of predefined \MF\ modes is available in the 
CTAN archives. The file \filename{fonts/modes/modes.mf} contains 
appropriate \MF\ modes for many printers.

Each mode has a name\index{metafont@\MF!mode name}. 
For example, \verb|laserwriter| and
\verb|LNOthree| are two modes from \filename{modes.mf}.  See
``\nameref{sec:buildfont},'' later in this chapter for
how to select a mode when you run \MF.

Internally, \MF\ has a variable called \verb|mode| that it uses
to keep track of the current mode.  The examples in this chapter
use the \verb|laserwriter| mode because that is appropriate for my
printer.  

%Two modes, \verb|proofing| mode and \verb|localfont| mode are
%special.  

\subsubsection{Proofing mode}

Proofing mode\index{metafont@\MF!proofing mode} is the 
mode that \MF\ uses if you do not select another
mode.  In proofing mode, \MF\ displays each character on the screen.
This mode does not usually produce a useful font because it
 sets the device resolution to 2601.72 dots per inch (dpi).  That
resolution is chosen so that there are exactly 36 pixels per point
(one point is 1/72.27 inches in \TeX).

Because font designers are going to want to look at a font on the
screen far more often than they will want to print it, \MF\ uses 
proofing mode as the default.

If you find that \MF\ is producing fonts with huge resolutions
(thousands of dots per inch) or fonts without \ext{TFM} files, you're 
probably running in proofing mode.
Remember to use the mode-setting commands on the command line when
you run \MF.  These commands are described in ``\nameref{sec:buildfont},''
later in this chapter.
%\subsubsection{Localfont Mode}
%
%The \verb|localfont| mode is customarily set to be the mode of the
%local printer.  If you have several different printers that have different
%modes (although many laser printers share the same mode), \verb|localfont| mode
%may not be very useful.
%
%If you are unsure of what mode to use, try \verb|localfont| mode
%first and see if it produces good looking fonts.
%
%In some implementations, \verb|localfont| mode is the default mode.
%However, in others, like \emTeX, \verb|proofing| mode is the default.
%In either case, it never hurts to request \verb|localfont|
%mode specifically.

\subsection{Selecting a Size}

The design size\index{metafont@\MF!font design size} of 
the font you are creating is determined
by the \ext{MF} file you use.  For example, the design size of
\filename{cmr10.mf} is 10pt.  The design size of \filename{cmr12.mf}
is 12pt, etc.
To select how large you want the bitmaps to be, you must set
the magnification.  There is no other way to select the
size when you are running \MF.

\newpage
If you know the design size of a font and the size of the bitmaps that
you actually want to produce, it is easy to calculate the magnification
required.  The magnification is simply the ratio of the size
you want to the design size.

For example, suppose you need a 16pt version of Computer Modern Roman.
First, pick the font that has a design size closest to the size you
want---in this case, Computer Modern Roman 17pt
(\filename{cmr17.mf}).  To calculate the magnification, simply
divide 16 by 17; in other words, the desired magnification is 0.9412.
To make a 13pt version of \filename{cmr12}, use a magnification of 1.0833.
Remember, in order to use these fonts in \TeX, you will have to use the
\verb|at| or \verb|scaled| operators (for example, 
\verb|\font\cmrxvi=cmr17 at 16pt|).

\subsection{Making a GF Font}
\label{sec:buildfont}

\index{GL fonts!making with metafont@making with \MF}\index{metafont@\MF!GL fonts}\index{fonts!GL}After you 
have selected a mode and a magnification, running \MF\ is
easy.  In general, you use a command line like the following:

\begin{ttindent}
\$ \textbf{mf {\bs}mode=}\textit{selected-mode}\textbf{; mag=}\textit{desired-magnification}\textbf{; {\bs}input} \textit{mf-file}
\end{ttindent}

For example, on a \Unix\ system, the following command line would make
a GF file for the Computer Modern Roman 12pt font at a size of 13pt:

\begin{ttindent}
\$ \textbf{mf '{\bs}mode=laserwriter; mag=1.0833; {\bs}input cmr12'}
\end{ttindent}

Note the use of quotation marks to avoid shell-expansion of the backslash
characters.

On an MS-DOS or OS/2 system, where the backslash is not a special
character, the 13pt version of \filename{cmr12} is made with the following
command:

\begin{ttindent}
\$ \textbf{mf {\bs}mode=laserwriter; mag=1.0833; {\bs}input cmr12}
\end{ttindent}

As you can see, the semicolons are {\em required\/} in both cases.

\ixx{Magnification}{magnification!for metafont@for \MF} can 
be expressed with the \verb|magstep()| function
if the font is being built at a standard magstep size.  For example,
the following command builds a 14.44pt version of \filename{cmr12}:

\begin{ttindent}
  \$ \textbf{mf {\bs}mode=hplaser; mag=magstep(1); {\bs}input cmr12}
\end{ttindent}

This corresponds to \cs{magstep1}.  This example also illustrates
how a different mode can be selected. In this case, the font is being
built for an HP LaserJet printer.  The command is shown without quotations,
but they are still necessary on \Unix\ systems.

\subsection{Making a PK Font}

\index{PK fonts@\ext{PK} fonts!made with metafont@made with \MF}\index{metafont@\MF!PK fonts}\index{fonts!PK}Running 
\MF, as shown above, produces a \ext{TFM} file and a \ext{GF}
file.  \TeX\ only needs the \ext{TFM} file, which can simply be copied
to the appropriate directory.  For the Computer Modern fonts, you
probably already have copies of the \ext{TFM} files, so the duplicates
produced by \MF\ can
be deleted. The \ext{TFM} file does not change at
different magnifications, so you do not need to save different
\ext{TFM} files for different sizes.

The \ext{GF} file contains a font in the ``generic bitmap'' format.
Most \dvidriver{}s use \ext{PK} files.  The \program{GFtoPK} program
converts \ext{GF} fonts into \ext{PK} fonts.
If \MF\ creates the file \filename{cmr12.gf}, you can convert it into
a \ext{PK} file by issuing the command:

\begin{ttindent}
  \$ \textbf{gftopk cmr12.gf cmr12.pk}
\end{ttindent}

After conversion, you should move the \ext{PK} file to the appropriate
directory for your \dvidriver\ and delete the \ext{GF} file.  \ext{PK} files
are generally stored in one of two directory structures.  On operating systems
that support long filenames, it is most common to store all the \ext{PK}
files for a given font in the same directory, using the extension to identify
the resolution.  In this case, a 360dpi version of \filename{cmr10} would be
stored as \filename{cmr10.360pk}.  Under operating systems like MS-DOS, which
do not support long filenames, different resolutions are stored in their
own directories.  In this case the 360dpi font just described would be stored
as \filename{cmr10.pk} in a directory called \filename{360dpi} or
\filename{dpi360}.

\section{What About Errors?}

Because I'm operating under the assumption that you are not writing your
own \MF\ programs, errors\index{error messages!metafont@\MF} are 
far less likely.  Nevertheless,
there are a few things that can go wrong.

Figure~\ref{fig:eight} shows a simple figure eight created with \MF.
The code which creates this symbol is shown in Example~\ref{ex:eightcode}.

\begin{figure}
  \dimen0=\textwidth\advance\dimen0 by -5pt%
  \begin{center}
    \leavevmode\vbox{\hrule\hbox to \dimen0{\vrule\hfil\vbox{%
        \vskip2em
        \font\eight=eight
        \hbox{\eight A}
        \vskip1em
        }\hfil\vrule}\hrule}
    \caption{A figure eight created with \protect\MF}
    \label{fig:eight}
  \end{center}
\end{figure}

\begin{example}{ex:eightcode}{The Code for the Figure Eight}
mode_setup;
u# := 2mm#;
define_pixels(u);

beginchar("A", 8u#, 9u#, 5u#);
  z1 = ( 0u,  0u);
  z2 = ( 8u,  0u);
  z3 = ( 1u,  8u);
  z4 = ( 7u,  8u);
  pickup pencircle scaled 1u#;
  draw z4 ..  z1 .. z2 ..  z3 .. cycle;
  pickup pencircle scaled 3u#;
  drawdot z1;
  drawdot z2;
  drawdot z3;
  drawdot z4;
endchar;

\end
\end{example}

Let's look at some of the things that can go wrong.  In the following
examples, a \Unix\ implementation of \MF\ is being used.  Quotation marks are
used to prevent misinterpretation of the backslashes.

\begin{iplist}{.25in}
  \ipitem[\texttt{! I can't find file `mode=laserwriter.mf'.}]

  If you forget to put a backslash in front of the \verb|mode| parameter,
  \MF\ thinks that mode specification is the name of the \ext{MF} file.
  For example, \texttt{mode=laserwriter} is misinterpreted as the
  filename:

\begin{shortexample}
$ mf 'mode=laserwriter; \input eight'
This is METAFONT, Version 2.71 (C version 6.1)
! I can't find file `mode=laserwriter.mf'.
<*> mode=laserwriter
                    ; \input eight
Please type another input file name: 
\end{shortexample}

\ipitem[\texttt{! Value is too large (xxxx).}]

This error occurs if some element of the image goes outside of the bounds
allowed by \MF.  Frequently, this occurs in proofing mode when the image is so
large that it can't be rendered.\footnote{Obtaining this error from
\filename{eight.mf} required changing unit size from 2mm to 5mm.}  In this
example, \MF\ is in proofing mode because I forgot the \cs{mode=laserwriter}
parameter on the command line:

\begin{shortexample}
$ mf '\input eight'
This is METAFONT, Version 2.71 (C version 6.1)
(eight.mf
! Value is too large (4097).
<recently read> ;
                 
beginchar->...rdp:=(EXPR3);w:=hround(charwd*hppp);
                                       h:=vround(charht*hppp);
l.5 beginchar("A", 8u#, 9u#, 5u#)
                                 ;
? 
\end{shortexample}

This doesn't mean that \MF\ can't render very large images; it just means that
my design isn't robust enough to handle an extremely high resolution.

\ipitem[\texttt{! Curve out of range.}]

This error is also caused by an image that is too large to be
rendered.\footnote{Obtaining this error from \filename{eight.mf} required
changing unit size from 2mm to 4mm.}  Again, \MF\ is trying to create the
image at an extremely high resolution; it is in proofing mode because
I forgot the \cs{mode=laserwriter} parameter on the command line:

\begin{shortexample}
$ mf '\input eight'
This is METAFONT, Version 2.71 (C version 6.1)
(eight.mf
! Curve out of range.
<to be read again> 
                   ;
l.11   draw z4 ..  z1 .. z2 ..  z3 .. cycle;
                                            
? 
\end{shortexample}

\ipitem[\texttt{unknown string}]

This is what happens if you forget the semicolon after the \cs{mode}
parameter.  In this example, \MF\ thinks that ``\verb|laserwriter \input eight|''
is the mode, which doesn't make sense:

\begin{shortexample}
$ mf '\mode=laserwriter \input eight'
This is METAFONT, Version 2.71 (C version 6.1)
(eight.mf
>> unknown string mode_name0
! Not a string.
<to be read again> 
                   ;
mode_setup->...nput "&mode)else:mode_name[mode]fi;
                                                  if.unknown.mag:...
l.1 mode_setup
              ;
? 
\end{shortexample}

\ipitem[\texttt{! Extra tokens will be flushed.}]

Oops, another missing semicolon:

\begin{shortexample}
$ mf '\mode=laserwriter; mag=magstep(1) \input eight'
This is METAFONT, Version 2.71 (C version 6.1)
(eight.mf
! Extra tokens will be flushed.
<to be read again> 
                   warningcheck
mode_setup->warningcheck
                        :=0;if.unknown.mode:mode=proof;fi...
l.1 mode_setup
? 
\end{shortexample}

\ipitem[Nothing happens\protect\ldots]

\MF\ is sitting at the asterisk prompt waiting for you to do something.
You forgot to \cs{input} the font:

\begin{shortexample}
$ mf '\mode=laserwriter; eight'
This is METAFONT, Version 2.71 (C version 6.1)

*
\end{shortexample}

\ipitem[\texttt{Output written on eight.nnngf\protect\ldots}]

Success!  A 300dpi version of \filename{eight.mf} has been created and stored
in the file \filename{eight.300gf}:

\begin{shortexample}
$ mf '\mode=laserwriter; \input eight'
This is METAFONT, Version 2.71 (C version 6.1)
(eight.mf [65] )
Font metrics written on eight.tfm.
Output written on eight.300gf (1 character, 1964 bytes).
Transcript written on eight.log.
\end{shortexample}

On systems that place restrictions on the length of a filename extension, it
is abbreviated to \filename{eight.300}.  The next step is to run
\textit{gftopk eight.300gf}. This will produce \filename{eight.300pk}, which
our \dvidriver\ can use.

\end{iplist}

\subsection{Other Errors}

The list of errors in the preceding sections is hardly exhaustive.  For
example, the gothic fonts (see Appendix~\ref{app:fonts},
\textit{\nameref{app:fonts}}) produce two more errors when rendered at low
resolutions: \textit{! inconsistent equation}, caused by apparently
contradictory statements (probably due to rounding errors) and \textit{!
strange path}, caused by a path that does not appear to run
counter-clockwise (again probably caused by rounding errors, these are very
complex fonts).\footnote{There are very specific rules about which way paths
must be drawn so that \MF's algorithms for filling in the character's outline
can succeed.}  

Most of the other errors that are encountered can be ignored (the preceding
examples from the gothic fonts can certainly be ignored---at least at 300dpi).
Entering \key{H} at \MF's question mark prompt will describe the cause of the
error in more detail.

\section{Output at Very High Resolutions}

Producing \MF\ output for very high resolution
\index{metafont@\MF!high-resolution output} devices like phototypesetters
is sometimes difficult.  This will frequently require a big \MF\ for
the same reasons that processing a large document requires a big \TeX.

It is possible to construct examples that cannot be rendered at very high
resolutions because they are simply too large and complex.

\section{Output at Very Low Resolutions}

\index{metafont@\MF!low-resolution output}The intricate 
detail of many characters is difficult to render at very 
low resolutions.  Yet, it is sometimes desirable to produce output at low
resolutions for on-screen previewers and dot matrix printers.

Usually, errors like ``\texttt{strange turning path}'' that result from attempting
to produce output at low resolutions can be ignored, but sometimes the
resulting characters will be distorted.
