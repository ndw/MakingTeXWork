\chapter{Fonts} 
\RCSID$Id: ch05.tex,v 1.1 2002/08/23 14:58:46 nwalsh Exp $ 
\label{chap:fonts}

\ifincludechapter\else\endinput\fi

All of the common \TeX\ macro packages use 
the \idx{Computer Modern fonts}\index{fonts}\index{tex@\TeX!fonts}\index{fonts!Computer Modern}
by default.  In fact, the Computer Modern fonts are so frequently used
in \TeX\ documents that some people believe they are \emph{the} \TeX\ fonts
and that no other options are available.

This is not the case.  In fact, using different fonts in \TeX\ is quite
easy.  However, many interrelated font issues can be quite complicated,
and it is possible to do things that
make your documents print incorrectly (or make them unprintable).

This chapter explores all of the issues related to fonts and how these
issues are resolved by a combination of font files, \TeX\ macros,
\dvidriver{}s, and careful planning.  At a high level, it works like
this:

\begin{enumerate}
  \item \TeX\ macros 
%are interpreted which \emph{eventually}
        select a 
        font (by assigning \verb|\font\fontid=fonttfm|).  The macros
%which are interpreted 
        may be very simple or quite complex (as is
        the case in \LaTeX's New Font Selection Scheme).
  \item \TeX\ loads the metric information from 
        \filename{fonttfm.tfm}\index{fonttfmtfm files@fonttfm.tfm files}.
        Many implementations of \TeX\ look for this file in the 
        directories on the \envvar{TEXFONTS} path.  \TeX\ cannot process
        your document if it cannot find 
        the \ext{TFM} file\index{TFM files}.\footnote{Some
        fonts may be preloaded in the format file.  \TeX\ does not need
        \ext{TFM} files for those fonts since the metric information is
        already available.}
  \item \TeX\ writes a \ext{DVI} file\index{DVI files}.  The \ext{DVI} 
        file contains the name
        of each font (the name of the \ext{TFM} file) and the magnification
        used (magnification is discussed in the section 
        ``\nameref{sec:issueofsize}''
        later in this chapter).
  \item The \dvidriver\ attempts to locate font files for each font used
        at each magnification.  Depending on the \dvidriver, fonts can come
        from many places.  Many drivers consult a list of built-in fonts to
        see if the desired font resides in the printer.  Some also consult
        a list of font substitutions.  (This can be used to substitute existing
        fonts for fonts that you do not have, like Computer Modern Roman
        in place of Times Roman if you don't have Times Roman.)

        Assuming that the font is not 
        built-in\index{fonts!built-in} or replaced by substitution
        of a built-in font, the \dvidriver\ looks for a font file.  Most
        modern \dvidriver{}s look for \ext{PK} fonts, although some also
        look for \ext{GF}\index{fonts!GF} 
        and \ext{PXL} fonts\index{fonts!PXL} as well.

        The exact location of these files varies.  Some implementations look
        in the directories on the \envvar{TEXFONTS} path, others look in
        the \envvar{TEXPKS} or \envvar{PKFONTS} paths.

        A typical font directory on a \Unix\ system is 
        \filename{/usr/local/lib/tex/fonts/pk}.
        The files in this directory typically have names of the form
        \filename{tfmname.999pk}; where \textit{tfmname} is
        the name of the font and \textit{999} is the resolution.

        On file systems which have short filenames (for example, MS-DOS)
        a typical font directory is 
        \filename{\bs tex\bs fonts\bs 999dpi} (or
        \filename{\bs tex\bs fonts\bs dpi999}).\footnote{Older \dvidriver{}s
        may still use \ext{PXL} files.  They are frequently stored 
        in directories
        with names like \filename{pxl999}.}  The files are then
        simply \filename{tfmname.pk}.

        Some \dvidriver{}s employ automatic font generation to attempt
        to create the font if it cannot be found.
  \item The \dvidriver\ produces output suitable for a particular device.
        This may include one or more forms of downloaded fonts as well as
        requests for built-in fonts that are assumed to exist.
\end{enumerate}

\begin{note}{NOTE}
  \TeX\ output isn't printable directly.
  \TeX\ typesets a document in a device-independent fashion; it's the
  \dvidriver\ that actually {\em prints\/} the document.  This is important
  because many of the font issues are really \dvidriver\ issues more than
  \TeX\ issues, and because the distinction between \TeX{}ing a document
  and printing a document is another layer of complexity that can be a
  source of difficulty.
\end{note}

\section{What \protect\TeX\ Needs To Know}

\TeX\ needs remarkably little information about a font.  Recall from
Chapter~\ref{chap:tex}, \textit{\nameref{chap:tex}}, that \TeX\ typesets
each page using ``\idx{boxes} and \idx{glue}.''  In order to perform this
function, \TeX\ needs to know only the size of each character (its
width, height, and 
depth)\index{boxes!height}\index{boxes!depth}\index{boxes!width}\index{characters!as  boxes}\index{height of box}\index{depth of box}\index{width!of box}.
In practice, \TeX\ fonts contain a little
bit more information than simply the size of each character.
Generally, they contain ligature and kerning information as well.

{\em Ligatures\/}\index{ligatures} are a 
typographic convention for replacing some letter
combinations with single symbols.  In English, this is done solely to improve
the appearance of the letter combinations.  Figure~\ref{fig:filig}
shows a common ligature in English, ``fi.''  Other common ligatures
in English are ``ff'', ``fl'', and the combinations ``ffi'' and ``ffl.''
Other languages have different ligatures.

\epsbox[fi-ligs.eps]{fig:filig}{``fi'' as two 
  characters and as a ligature}

{\em Kerning\/}\index{kerning} is the process of 
adding or removing small amounts of space
between characters
to improve the appearance of particular letter combinations.   Although every
character has a natural width, some combinations of characters give the
illusion of too much or too little space.  Figure~\ref{fig:kerning}
shows a common example in the word ``We.''  

\epsbox[we-we.eps]{fig:kerning}{``We'' unkerned 
  and kerned}

This is all the information that \TeX\ needs, and it is all that is contained
in the \ext{TFM} files that \TeX\ uses.\footnote{Actually, \ext{TFM} files
contain a little bit more information.  The width of a space, the amount of
stretch and shrink allowed between words, the amount of extra space allowed
after a punctuation mark, and some specialized information used only in fonts
for mathematics also appear in the \ixx{\ext{TFM} files}{TFM files}.  In 
principle, \ext{TFM}
files can contain even more information, although they rarely do.  For a
complete, detailed description of the information stored in a \ext{TFM} file,
consult Appendix F of {\it The \TeX{}book}~\cite{kn:texbook} and \textit{The
TF To PL Processor}~\cite{texware:tftopl}.}

\subsection{Selecting a Font in \protect\TeX}

\TeX's macro language includes a \cs{font} primitive for
loading a 
font\index{fonts!metrics}\index{fonts!selecting  with tex@selecting  with \TeX}\index{metrics font}.  This primitive operation 
associates a control sequence\index{control sequences!font}
with the metrics in a particular \ext{TFM} file.  For example, the
following line associates the control sequence \cs{tinyfont} with
the metrics in \filename{cmr5.tfm} (the Computer Modern Roman 5pt font):

\begin{shortexample}
\font\tinyfont=cmr5
\end{shortexample}

\begin{note}{NOTE}
  Remember, only letters can appear in a \TeX\ control sequence.  You
  cannot say \verb|\font\cmr5=cmr5| because \cs{cmr5} is not
  a valid control sequence.
\end{note}

The control sequence defined with \cs{font} can subsequently be
used to change the current font.   After the above command,
\cs{tinyfont} in your document selects the Computer
Modern Roman 5pt font as the current font.

In practice, using the primitive operations to select fonts has a
number of disadvantages.  Later in this chapter,
the worst of these disadvantages is described
and a better alternative is offered.  For simplicity, the primitive
operations are used in most of the examples in this chapter.

\subsection{Which Character Is Which?}

How does \TeX\ know which
character\index{characters!array as font} to 
use when it reads a symbol from an input file?  The
answer is simple: \TeX\ considers each font to be an array of
characters.  It uses the numeric 
ASCII\index{ASCII character set!code in font array} 
code of each character to
determine what element in the array to use.\footnote{This is true even
if your computer's natural character set is not ASCII.  It is the
responsibility of the implementor to map the computer's natural
character set into ASCII.}  
This is a reasonable and efficient scheme as long as the ASCII values
of the characters in your input file are the same as the metric information
that \TeX\ is using.

The section called ``\nameref{sec:fonts:encodingvec}'' in this chapter
describes how the ordering of characters in a font is determined. It also 
discusses
some of the problems that can arise when \TeX\ and your printer have
conflicting information about the arrangement of characters in the font.  The
``\nameref{sec:fonts:virtualfonts}'' section describes the \TeX\ mechanism for
constructing fonts with different arrangements of characters.

%Some implementations of \TeX\ allow you to define a custom mapping
%between the computer's displayed font and the ASCII character (or
%characters) that it represents.  For example, \emTeX\ has a feature
%that allows you to define ``\TeX\ Code Pages'' which perform this
%mapping.  This feature is unimportant, albeit very convenient for the
%typist, and can be ignored.  For a number reasons, explored in the
%``\nameref{sec:fonts:encodingvec}'' and
%``\nameref{sec:fonts:virtualfonts}'' sections of this chapter, another
%level of translation when \TeX\ reads the input file cannot solve many
%of the problems that occur when using fonts.\editorial{this is confusing!}

\section{The Issue of Size}
\label{sec:issueofsize}

In the purest sense, selecting a font determines only 
what shape\index{shape of fonts}\index{fonts!shape} each 
character will have; it does not determine the 
size\index{size of fonts}\index{fonts!size} of the font.
Unfortunately in practice, the issue of shape and size cannot be
separated.  The same shapes {\em appear\/} to be different at different
sizes.  
\ixx{\textit{Optical scaling}}{optical scaling}\index{scaling!optical} 
uses different designs to
make letters at different sizes appear to have the same shape.
Conversly, \idx{linear scaling}\index{scaling!linear} produces 
characters with
exactly the same shape at different sizes, although they appear
slightly different.

%In Figure~\ref{fig:sizeandshape}, for example, only the first
%row of R's {\em appear\/} to be from exactly the same font.  This is because
%the characters in the first row were designed to appear the
%same at the sizes shown.  This is called {\em optical scaling}.  The
%remaining three rows show each of the fonts from the first row {\em
%linearly scaled\/} to the requested size.
%
%\input{ft-scale.fig}

Because the size of an object affects the way that the human eye and
the brain perceive its shape, a simple linear scaling of each
character does not produce the most aesthetically pleasing results.
To overcome this problem, \TeX\ uses two different quantities to
express the notion of size: design size and magnification.

{\em Design size\/}\index{fonts!design size}
addresses the issue of perception: a font looks most like
the way it was designed to appear when printed at or close to its
design size.  Furthermore, two fonts with different design sizes (say
8pt and 12pt) that are the same typeface should {\em appear\/}
identical when printed at 8pt and 12pt, respectively, even though the
actual shapes may be slightly different.  The design size of a font is
intrinsic to the font itself and cannot be changed or influenced by
\TeX.  It is important because it has a direct impact on the aesthetics
of the typeset page.

{\em Magnification\/}\index{fonts!magnification} addresses the 
notion of linear scaling.  Changing 
magnifications
changes the size of each character without altering its actual shape.
Although, as noted above, changing its size may change its {\em apparent\/}
shape.  In general, you should print fonts as close to their design
size as possible---in other words, with a magnifcation as close to 1.0 as
possible.  Figure~\ref{fig:Rs} demonstrates how different characters
may look when printed at very large magnifications.  
%
%The two R's have
%the same apparent shape when printed at 10pt and 17pt, respectively
%(``{\fontfamily{cmr}\fontshape{n}\fontsize{10}{12pt}\selectfont R}'' and
%``{\fontfamily{cmr}\fontshape{n}\fontsize{17}{18pt}\selectfont R}'', see?).

\epsbox[two-rs.eps]{fig:Rs}{The Computer Modern Roman letter ``R'' at 150pt:
  (a) from a 5pt design; (b) from a 17pt design}

\subsection{Expressing Design Size in \protect\TeX}

The design size\index{fonts!design size} of 
a font is an integral part of the \ext{TFM} file (because
it is intrinsic to the font that 
the \ext{TFM} file\index{TFM files!design size} describes).
In order to select a different design size, you must select a
different \ext{TFM} file.  For example, the Computer Modern Roman font
is usually distributed at eight design sizes: 5, 6, 7, 8, 9, 10, 12,
and 17 points.  Metrics for these sizes are stored in the \ext{TFM}
files \filename{cmr5.tfm}, \filename{cmr6.tfm}, \filename{cmr7.tfm},
$\ldots$, and \filename{cmr17.tfm}.

Every font that you select has a specific design size, even though you may
elect to use the font at another size.

%When you select a font in \TeX, you are really selecting a font and a
%design size.

\subsection{Expressing Magnification in \protect\TeX}

Magnification\index{fonts!magnification} can be 
expressed either implicitly or explicitly in
\TeX.  Implicitly, magnification can be expressed by selecting a
particular font \verb|at| a particular size.  For example, the
following line defines the control sequence \cs{big} to be the
Computer Modern Roman 10pt font \verb|at| a size of 12pt (an implicit
magnification of 120\%):

\begin{shortexample}
\font\big=cmr10 at 12pt
\end{shortexample}

Explicit magnification is selected by requesting a font
\verb|scaled| to a particular extent.  For example, the following
line defines the control sequence \cs{bigger} to be the Computer
Modern Roman 10pt font at a magnification of 144\% (in other words,
at 14.4pt):

\begin{shortexample}
\font\bigger=cmr10 scaled 1440
\end{shortexample}

As you can see, \TeX\ expects the \verb|scaled| magnification to be ten times
the percentage of magnification.  The magnification that you request must be
an integer (you can't say \verb|scaled 1440.4|).  Multiplying the
magnification by 10 allows \TeX\ to accept fractional percentages like
104.5\%.

\subsection{Standard Magnifications}

\TeX\ provides seven standard 
magnifications.  There are several good
reasons to use these magnifications whenever possible.  The most
important is that most \TeX\ systems can easily print fonts at
these sizes.  As described later in the section ``\nameref{sec:printing},''
\TeX's ability to select any font at any magnification does
not guarantee that it can be printed at that size.  By using standard
sizes, you increase the likelihood that your document will be
printable on your system and on other \TeX\ systems (if portability is
an issue).  Using standard sizes will also give consistency to your
documents.  If you write many documents separately that may eventually
be collected together (as a collection of short stories or a series of
technical reports, for example), the internal consistency of sizes
will make them appear more uniform.  Finally, the standard sizes have
aesthetic characteristics as well.  Each size is 1.2 times the
preceding size.  The geometric relationship between the sizes is
designed to make them appear pleasing when mixed together.

The standard sizes can be selected with the \cs{magstep} control
sequence.  The standard sizes (or steps) are called \textit{magsteps} in
\TeX\ jargon.  The natural size of a font is its \cs{magstep0}
size.  The \cs{magstep1} size is 20\% larger.  And
\cs{magstep2} is 20\% larger than that (44\% larger than the
original design), etc.  To select the Computer Modern Roman 10pt font
at its next largest standard size, use:

\begin{shortexample}
\font\larger=cmr10 scaled\magstep1
\end{shortexample}

For those occasions when you want a font that is only a little bit
larger, \TeX\ includes the control sequence \cs{magstephalf} which
is halfway between \cs{magstep0} and \cs{magstep1}.

Most \TeX\ formats that are based upon Plain \TeX\ define seven magsteps:
\cs{magstep0}, \cs{magstephalf}, and \cs{magstep1} through 
\cs{magstep5}.

By using different design sizes and different standard magnifications,
you gain access to a very wide range of sizes.  For example, given
the seven standard design sizes and seven standard magsteps, it is
possible to print Computer Modern Roman at any of the following sizes
(all sizes are in points):

{\def\b{\phantom{0}}\def\d{\phantom{.}}%
\begin{tabular}{rrrrrrrr}
5\d\b\b & 7.67    &  9.6\b   & 10.95    & 13.15  & 16.59 & 19.91 & 29.38\\
5.48    & 8\d\b\b &  9.86    & 11.52    & 13.82  & 17\d\b\b & 20.4\b  & 29.86\\
6\d\b\b & 8.4\b   & 10\d\b\b & 12\d\b\b & 14.4\b & 17.28 & 20.74 & 35.25\\
6.57    & 8.64    & 10.08    & 12.1\b   & 14.52  & 17.42 & 22.39 & 42.3\b \\
7\d\b\b & 8.76    & 10.37    & 12.44    & 14.93  & 18.62 & 24.48 & \\   
7.2\b   & 9\d\b\b & 10.8\b   & 12.96    & 15.55  & 18.66 & 24.88 & \\   
\end{tabular}%
}

\subsection{Where Do TFM Files Come From?}
\label{sec:gettingtfms}

In order to use {\em any\/} font in \TeX, you must have a \ext{TFM} file
for it.  How you can 
acquire \ext{TFM} files\index{TFM files!getting} for the fonts you use depends
on the kinds of fonts and where they were developed.  The following
list offers suggestions for the most commonly used fonts:

\begin{iplist}{.25in}
\ipitem[PostScript fonts]
  
  Every vendor that supplies PostScript 
  fonts\index{PostScript!fonts}\index{fonts!PostScript} (either as font files
  or built-in to a printer or cartridge) should also supply Adobe Font
  Metric (\ext{AFM}) files.  \ext{AFM} files provide complete metric
  information about the fonts.\footnote{Actually, for math fonts, \TeX\
  requires several metrics that are not 
  usually provided.  \ixx{\program{afm2tfm}}{afm2tfm}
  has options which allow you to specify these extra metrics when you
  convert the \ext{AFM} file.}
  The \ext{AFM} files can be converted
  into \ext{TFM} files with the \program{afm2tfm} utility distributed
  with \program{dvips}.

\ipitem[LaserJet built-in fonts]

  The complete metric information for LaserJet built-in 
  fonts\index{fonts!LaserJet built-in} is supplied
  by Hewlett-Packard in ``Tagged Font Metric'' files.  These are available
  directly from Hewlett-Packard (unfortunately, they are not distributed
  with the printers).  The Tagged Font Metrics can be converted into
  \TeX\ \ext{TFM} files with \program{hptfm2pl}, a free utility written
  by, well, me, actually.

  A collection of \ext{TFM} files for the standard built-in fonts on the
  LaserJet III and IV printers is available in the CTAN archives in
  \ctandir{/tex-archive/fonts/ljmetrics}.

\ipitem[LaserJet softfonts]

  Bitmapped LaserJet softfonts\index{fonts!LaserJet softfonts}
  can be converted into \TeX\ fonts with
  the \ixx{\program{SFPtoPK}}{SFPtoPK} utility.  The resulting font includes both
  \TeX\ \ext{PK} and \ext{TFM} files.

  Scalable LaserJet softfonts should be distributed with Hewlett-Packard 
  Tagged Font Metric files.  These can be converted into \TeX\ \ext{TFM}
  files with the free utility \ixx{\program{hptfm2pl}}{hptfm2pl}.

\newpage
\ipitem[\TeX\ fonts (\textnormal{\MF})]

  The \MF\index{fonts!metafont@\MF} program 
  renders \TeX\ \ixx{\ext{MF} files}{MF files@\ext{MF} files} 
  and produces a \ext{TFM}
  file.  Usually it produces a \ext{GF} file\index{GF files} as 
  well, but the special
  mode \textit{tfmonly} can be used to create just the 
  \ext{TFM} file.\footnote{If your \MF\ distribution
  does not include a \textit{tfmonly} mode, you can find one in the 
  \filename{modes.mf} file on CTAN.  Consult Chapter~\ref{chap:mf}, 
  {\it\nameref{chap:mf}}, for more information.}
  \MF\ modes and other aspects of font creation with \MF\ are 
  described in Chapter~\ref{chap:mf}, {\it \nameref{chap:mf}}.

\ipitem[TrueType fonts] 

  Incomplete metrics are frequently\index{fonts!TrueType}
  distributed in the form
  of Windows' \ixx{\ext{PFM} files}{PFM files@\ext{PFM} files}.  Some commercial previewers
  for Windows can extract metric information and build a \ext{TFM} file.
  However, at present, I do not know of any free utilities which can
  build \TeX\ metrics from TrueType fonts. 
\end{iplist}

\section{The New Font Selection Scheme}
\label{sec:nfss2}

The \idx{New Font Selection Scheme} is a method 
for selecting fonts in Plain
\TeX\ and \LaTeX.  It was introduced briefly in the section
``\nameref{macpack:sec:latexvslatexe}'' of Chapter~\ref{chap:macpack},
{\it\nameref{chap:macpack}}.  This section describes release two of the New
Font Selection Scheme (known as the 
NFSS2)\index{New Font Selection Scheme!version 2 (NFSS2)}
as it exists in the \LaTeXe\
format.  Because version one\index{New Font Selection Scheme!version 1 (NFSS)}
is now obsolete, it is not described. 

The NFSS defines a method of font selection used in place of \TeX's primitive
\cs{font} command.  The problem with font selection
using \cs{font} is that it ties a control sequence to a particular font at
a particular size, which has unpleasant consequences when more than one font
is used in a document.  Consider the definition \verb|\font\it=cmti10|.  This
associates the control sequence \cs{it} with the italic Computer Modern
font (at 10pt).  After this definition, a sentence like:

\begin{shortexample}
This requires {\it emphasis}.
\end{shortexample}

has the desired result, if Computer Modern Roman at 10pt is the font in
use when \cs{it} is encountered:

\begin{exindent}
\fontfamily{cmr}\selectfont
This requires {\it emphasis}.
\end{exindent}

If you are using some other font, perhaps in a chapter heading, you get:

\begin{exindent}
\fontfamily{cmr}\fontseries{bx}\fontsize{14}{16pt}\selectfont
This requires {\fontfamily{cmr}\fontseries{m}\fontsize{10}{12pt}\selectfont
\it emphasis}.
\end{exindent}

This is almost certainly not what you wanted.  The NFSS
overcomes this difficulty by describing each font with five
independent parameters: encoding, family, series, shape, and size.

\newpage
\begin{iplist}{.25in}
\ipitem[Font encoding]
 
The encoding\index{font encoding (in NFSS2)}
parameter identifies the encoding vector of the font.  Encoding
vectors play an important role in the selection of characters in a font.
Encoding vectors are described more thoroughly 
later in this chapter.
\TeX\ Text, \TeX\ Math Italic, and Adobe Standard are all 
\ixx{\textit{encoding vectors}}{encoding vectors}.

\ipitem[Font family]

The family parameter describes the typeface of the font\index{font family (in NFSS2)}.
Computer Modern,
Times Roman, Helvetica, Galliard, and Gill Sans are all {\em families}.

\ipitem[Font series]

Font series\index{font series (in NFSS2)} describes 
the joint notions of weight and width.  
Weight\index{weight of font} is a
measure of how darkly each character is printed, 
and width\index{width!of font} is a measure of how
wide or narrow the font is.  Standard abbreviations for weight and width are
shown in Table~\ref{tab:fontshapeww}.  Normal, bold-compressed,
extrabold-ultraexpanded, and light-medium are all examples of font {\em
series}.

{\LTleft=.25in%
\begin{xtable}{l|l||l|l}
  \caption{Standard Weight and Width Designations
    \label{tab:fontshapeww}}\\
  \rm\bf Abbr & \bf Weight        & \rm\bf Abbr & \bf Width\\
  \hline
  \tstrut
  ul & Ultra-light     & uc & Ultra-condensed  \\
  el & Extra-light     & ec & Extra-condensed  \\
  l  & Light           & c  & Condensed        \\
  sl & Semi-light      & sc & Semi-condensed   \\
  m  & Medium (normal) & m  & Medium           \\
  sb & Semi-bold       & sx & Semi-expanded    \\
  b  & Bold            & x  & Expanded         \\
  eb & Extra-bold      & ex & Extra-expanded   \\
  ub & Ultra-bold      & ux & Ultra-expanded   \\[2pt]
  \hline
\end{xtable}
}

The general rule for combining weight and width to form a series abbreviation
is to use the abbreviation for weight followed by the abbreviation for width,
unless one is ``medium,'' in which case it is left out.
Table~\ref{tab:fontshapeex} shows how several weight/width combinations are
used to form the series.  The series designation for a light-medium
font demonstrates that a single medium attribute is omitted.  If both the
width and weight are medium, use a single ``m'' for the series.

{\LTleft=.25in%
\begin{xtable}{l|l}
  \caption{Weight and Width Are Combined to Form Series
    \label{tab:fontshapeex}}\\
  \bf Weight and Width & \bf Series \\
  \hline
  \tstrut
  Bold extended               & bx \\
  Light medium                & l \\
  Medium extra-expanded       & ex \\
  Light extra-expanded         & lex \\
  Normal (medium, medium) & m \\[2pt]
  \hline
\end{xtable}
}

\ipitem[Font shape]

The shape\index{font shape (in NFSS2)}, in 
conjunction with series, defines the appearance
of the font.  Shape generally refers to the style of the face.  Bold,
italic, slanted, and outline are all examples of font shape.

The standard designations of font shape are shown in 
Table~\ref{tab:fontshapeshape}.

{\LTleft=.25in%
\begin{xtable}{l|l}
  \caption{Standard Abbreviations of Font Shape
    \label{tab:fontshapeshape}}\\
  \rm\bf Abbr. & \bf Shape \\[2pt]
  \hline
  \tstrut
  n  & Normal \\
  it & Italic \\
  sl & Slanted \\
  sc & Small caps \\
  u  & Upright italics \\[2pt]
  \hline
\end{xtable}
}

\ipitem[Font size]

Font size\index{font size (in NFSS2)} defines 
both the size of the characters and the spacing
between lines of text in that size.  The distinction between
design size and magnification, discussed at length in the first
part of this chapter, is hidden within the NFSS;
you need only select the size you want.\footnote{This 
doesn't mean that every size you want is available.  See the section
``\nameref{sec:fonts:defnfss2}'' for information about defining
font sizes in the NFSS.}

The spacing between lines of text is described as the (vertical)
distance between the baselines of two consecutive lines of type.
It is usually about 20\% larger than the size of the font.  For
example, a 10pt font is usually printed with 12pts between the
baselines of consecutive lines.  The inter-line distance that looks
best depends on the font and other design elements
of the document.  There really isn't a good rule for the value
that looks best, which is why you have to specify it.
\end{iplist}

\subsection{Selecting Fonts with the New Font Selection Scheme}

One of the most visible differences in the NFSS2 is that you are 
encouraged to change the way you 
select\index{fonts!selecting with NFSS2} fonts.  The NFSS2 defines
nine control 
sequences\index{control sequences!for selecting fonts with NFSS2} for 
user-level font selection.  They are
shown in Table~\ref{tab:nfss2:userlevel}.

\begin{xtable}{l|l}
  \caption{User-level Font Selection Control Sequences in NFSS2
    \label{tab:nfss2:userlevel}}\\
  \bf Control Sequence & \bf Resulting Change \\[2pt]
  \hline
\endfirsthead
  \caption[]{User-level Font Selection Control Sequences in NFSS2 (continued)}\\
  \bf Control Sequence & \bf Resulting Change \\[2pt]
  \hline
\endhead 
  \tstrut
  \cs{textrm}\verb|{}| & Switch to roman family \\
  \cs{textsf}\verb|{}| & Switch to sans-serif family \\
  \cs{texttt}\verb|{}| & Switch to typewriter family \\
  \cs{textbf}\verb|{}| & Switch to bold face weight/width \\
  \cs{textmedium}\verb|{}| & Switch to medium weight/width \\
  \cs{textit}\verb|{}| & Switch to italic shape \\
  \cs{textsl}\verb|{}| & Switch to slanted shape \\
  \cs{textsc}\verb|{}| & Switch to small-caps shape \\
  \cs{emph}\verb|{}|   & Switch to emphasized text \\[2pt]
  \hline
\end{xtable}

Instead of using \cs{it}, for example, to change to an italic font within a
group, you should use the \cs{textit} command with the text as an argument.
For example, instead of using:

\begin{shortexample}
This is some {\it italic\/} text.
\end{shortexample}

you should use:

\begin{shortexample}
This is some \textit{italic} text.
\end{shortexample}

The advantage of the new scheme is that these macros are much more
intelligent than the old ones.  Notice that I did not specify italic
correction (\cs{/}) in the second case.  This is because the \cs{textit}
macro is able to determine if the correction is necessary, and if so, inserts it
automatically.\ff{If you wish to suppress italic correction, use
\cs{nocorr} at the end (or beginning) of the text.}  
The macros can also be nested.

The NFSS2 allows you to use the old font selection macros, so existing
documents will not be affected.  If you want to set several paragraphs
in a different font, you should continue to use the old selection
macros because you cannot pass more than one paragraph of text to a
macro.\ff[2]{It is possible to write macros that accept
multiple paragraphs of text, but the NFSS2 font selection macros do
not do so.}

\subsubsection{Low-level interface to NFSS}

The user-level font selection commands are implemented in terms of six
low-level commands.  At this level, you must specify the encoding, family,
series, shape, and size of the font in order to select it.  There are six
control sequences for selecting a font: one each for specifying the font
parameters and one for actually switching to the new font.  This strategy
allows the parameters to be independent; any parameters that you do not 
explicitly change remain the same as the current font.

These six control sequences are:

\begin{iplist}[\textnormal]{.25in}
\ipitem[\texttt{\cs{fontencoding}\ttopenbrace\textit{enc}\ttclosebrace}]

      Selects the encoding scheme 
      \textit{\tt enc}\index{encoding vectors!NFSS2}. The encoding schemes
      officially supported by the NFSS2 are shown in
      Table~\ref{tab:nfss2:encoding}.

{\LTleft=.25in%
      \begin{xtable}{l|l}
        \caption{Encoding Schemes Supported by NFSS2
          \label{tab:nfss2:encoding}}\\
        \bf Encoding & \\
        \bf Scheme   & \bf Encoding Name \\[2pt]
        \hline
        \tstrut
        T1  & \TeX\ text Cork encoding \\
        OT1 & Old \TeX\ text encoding (the CMR encoding) \\
        OT2 & University of Washington Cyrillic encoding \\
        OT3 & University of Washington IPA encoding \\
        OML & \TeX\ math (italic) letters \\
        OMS & \TeX\ math symbols \\
        OMX & \TeX\ math extended \\
        U   & Unknown encoding \\[2pt]
        \hline
      \end{xtable}
}

\ipitem[\texttt{\cs{fontfamily}\ttopenbrace\textit{fam}\ttclosebrace}]

      Selects the family \textit{\tt fam}.

\ipitem[\texttt{\cs{fontseries}\ttopenbrace\textit{ser}\ttclosebrace}]

      Selects the series \textit{\tt ser}.

\ipitem[\texttt{\cs{fontshape}\ttopenbrace\textit{shp}\ttclosebrace}]

      Selects the shape \textit{\tt shp}.

\ipitem[\texttt{\cs{fontsize}\ttopenbrace\textit{ptsize}\ttclosebrace
  \ttopenbrace\textit{bskip}\ttclosebrace}]

      Selects the font size \textit{\tt ptsize} with a distance of \textit{\tt
      bskip} between lines.  Note that \textit{\tt ptsize} is a simple number,
      whereas \textit{\tt bskip} is a \TeX\ distance and you must specify
      units after the number.  Under the NFSS2, \textit{\tt ptsize} specifies
      the \emph{actual} size of the font,\ff{Under the NFSS1, the
      \textit{\texttt{ptsize}} was simply a label; the actual font loaded was
      the nearest magstep.} but the NFSS2 will attempt to find a
      closest match if there is no exact match for the requested size.  A
      warning message is issued if the size does not exist and a closest match
      has to be selected.  You can control the sensitivity of the warning with
      the \cs{fontsubfuzz} parameter.  It is initially set to 0.4pt,
      meaning that any font within 0.4pt of the requested size will be used
      without issuing a warning.

\ipitem[\texttt{\cs{selectfont}}]

      Switches to the font described by the current values of
      encoding, family, series, shape, and size.
\end{iplist}

From this description, can you figure out how the control sequence
\cs{it} could be defined under the NFSS\label{footnote:question}?  
The answer is on the bottom of page~\pageref{footnote:answer}.

\subsection{Defining Fonts with NFSS2}
\label{sec:fonts:defnfss2}

The standard Computer Modern fonts and most PostScript fonts can be
selected by using the appropriate style files or inputting the
appropriate macros.  However, if you have nonstandard fonts or fonts
for some other device, you can easily add them to the
NFSS\index{fonts!defining with NFSS2}.

The internal interface to the NFSS has been entirely redesigned.  
The new interface is much cleaner than the interface to the NFSS1, but
similar in design.

If you are not very familiar with \TeX, what follows may be a bit confusing;
treat this example as a sort of ``cookbook recipe'' and substitute the
font you wish to define for the \verb|logo| font (the
\verb|logo| font is the typeface used for the \MF\ logo).

\subsubsection{Declaring a family}

In order to add a new typeface, you must declare a new 
font family\index{font family (in NFSS2)!declaring} with
the control sequence 
\cs{DeclareFontFamily}.  If you are only adding new sizes or shapes to an
existing family, do not redeclare the family.

The parameters to \cs{DeclareFontFamily} are the encoding, name, and
loading options for the family.  Loading options are any commands that
should be executed every time this family is selected.  For most fonts,
there are no loading options.\footnote{One possible use of loading
options is to inhibit hyphenation in fixed-width fonts intended for use
in verbatim material.  The loading options \texttt{\ttbackslash 
hyphenchar\ttbackslash font=-1} have this effect.}

The following declaration creates the \textit{logo} family with the old \TeX\
encoding and no loading options:

\begin{shortexample}
\DeclareFontFamily{OT1}{logo}{}
\end{shortexample}

\subsubsection{Declaring shapes}

After a family has been created, you must specify what font 
shapes\index{font shape (in NFSS2)!declaring} are
available with \cs{DeclareFontShape}.  To make the font usable, you must
declare at least one font shape for each family.

The general form of a call to \cs{DeclareFontShape} is: 

\begin{ttindent}
\verb|\DeclareFontShape{|\textit{enc}\verb|}|%
   \verb|{|\textit{fam}\verb|}|%
   \verb|{|\textit{series}\verb|}|%
   \verb|{|\textit{shape}\verb|}|%
   \verb|{|\textit{sizes}\verb|}|%
   \verb|{|\textit{loading options}\verb|}|
\end{ttindent}

The family \texttt{\textit{fam}} (with the appropriate encoding \texttt{\textit{enc}})
must already have been created with \cs{DeclareFontFamily}.  The
\texttt{\textit{series}} and \texttt{\textit{shape}} parameters identify the
name of the series and shape.  Table~\ref{tab:fontshapeex} and
Table~\ref{tab:fontshapeshape} list some common series and shapes.  The
\textit{\tt sizes} parameter is a list of \textit{font-shape declarations},
described below, and the \textit{\tt loading options}, if specified, override
the loading options for the font family.

Each font-shape declaration indicates how the request for a font should be
handled.  The complete syntax for font-shape declarations is described in
\textit{Interface Description of NFSS2}~\cite{nfss2interface}.  Here we look
at three simple cases: substituting another font for the one requested,
generating the name of the \ext{TFM} file for the requested font, and
identifying a particular \ext{TFM} file for a size or range of sizes.  Each of
these techniques is used in the declaration of the medium, normal logo font in
Example~\ref{ex:logoshape}.  There should be no extra spaces in the font size
parameter.  If you spread it over multiple lines in your input file, make sure
that a comment character (\%) appears at the end of each line.

\begin{example}{ex:logoshape}{Font-shape Declaration with NFSS2}
\DeclareFontShape{OT1}{logo}{m}{n}{%
  <-8>sub * cmr/m/n%
  <8><9><10>gen * logo%
  <10.95>logo10 at 10.95pt%
  <12->logo10}{}
\end{example}

Each font-shape declaration begins with one or more 
size\index{font size (in NFSS2)!declaring} specifications in
angle brackets.  This indicates either a specific size (\verb|<10>| for a 10pt
font) or a range of sizes (\verb|<8-9>| for any size larger than or equal to
8pt and less than 9pt, \verb|<-8>| for all sizes less than 8pt, or
\verb|<10->| for all sizes larger than or equal to 10pt).  

In Example~\ref{ex:logoshape}, the first font-shape declaration indicates that
font substitution should be performed for any request at a size
smaller than 8pt.  The string \texttt{cmr/m/n} indicates that medium, normal,
Computer Modern Roman should be substituted in its place.  In general, the
substitution specifies the
\texttt{\textit{family}/\textit{shape}/\textit{series}} to be substituted.

The second declaration indicates that the name of the external font for 8pt,
9pt and 10pt fonts should be generated from the string \texttt{logo}
and the size (at 8pt, \texttt{logo8} will be used; at 9pt,
\texttt{logo9}; and at 10pt, \texttt{logo10}).

The third declaration demonstrates that the font specified (\texttt{logo10 at
10.95pt}) can be any valid \TeX\ font selection command.  After special
declarations have been processed (substitution, generation, etc.), the
remaining declaration text is passed to the \TeX\ \cs{font} primitive.

The last declaration specifies that \textit{any} size larger than 12pt is
valid.  Any size larger than 12pt will use the font \texttt{logo10}, scaled
appropriately.  Automatic font generation, if it is being used, can take care 
of actually generating the font.  Example~\ref{ex:logoshape} was constructed
to demonstrate several features of the font-shape declaration syntax.  A
simpler, more likely declaration for the medium normal logo font is
shown in Example~\ref{ex:logoshape2}.  

\begin{example}{ex:logoshape2}{Font-shape declaration with NFSS2 (simplified)}
\DeclareFontShape{OT1}{logo}{m}{n}{%
  <-8>sub * cmr/m/n%
  <8-9>logo8%
  <9-10>logo9%
  <10->logo10}{}
\end{example}

\newpage
The 8pt design could have been scaled down for sizes less
than 8pt, but you should try to avoid
large deviations from the design size.  Because the design for my book
does not require the logo font at sizes less than 8pt, substitution
was the best choice.\footnote{The only reason that sizes less than 8pt need to
be declared at all is that marginal notes (used for editorial comments
while the book was in revision) were set in 5pt, and the logo font 
occasionally turned up in a marginal note.}

\subsection{Storing Font Definitions}

The NFSS will load your font definitions automatically if you store
them\index{fonts!storing} in \ext{FD} files\index{FD files@\ext{FD} files} in 
a directory where \TeX\ looks for input files.
Whenever an unknown encoding/family is requested, NFSS attempts to load
the file \filename{encfamily.fd}. 
For example, if the font declarations described in the preceding sections
are stored in a file called \filename{OT1logo.fd}, nothing special
has do be done to use the logo family.  The first time the logo family
is selected, the definitions will be read from \filename{OT1logo.fd}.

If you are using \LaTeXe\ (or \LaTeX\ with the NFSS), there must already be a
large number of \ext{FD} files on your system.  For more information about
building the \LaTeXe\ format (which includes the NFSS2), consult the section
``\nameref{sec:buildlatexe}'' in Chapter~\ref{chap:macpack},
\textit{\nameref{chap:macpack}}.

\subsection{Changing the Defaults}

Sometimes you want the fonts that you define to replace the standard
Computer Modern fonts.  This is easily accomplished.  The NFSS
defines control sequences\index{control sequences!for identifying default fonts} which 
identify the default
fonts.  These control sequences are listed in Table~\ref{tab:nfssdefaults}.

\begin{xtable}{l|l}
  \caption{Default Fonts
    \label{tab:nfssdefaults}}\\
  \bf Control Sequence & \bf Font \\[2pt]
  \hline
  \tstrut
  \cs{rmdefault} & The default normal (roman) font \\
  \cs{bfdefault} & The default boldface font \\
  \cs{sfdefault} & The default sans-serif font \\
  \cs{itdefault} & The default italic font \\
  \cs{scdefault} & The default caps and small-caps font. \\
  \cs{defaultshape} & The default shape (\verb|n|) \\
  \cs{defaultseries} & The default series (\verb|m|) \\[2pt]
  \hline
\end{xtable}

If you have defined a new font, perhaps \verb|garamond| as I did
for this book, you can make it the default normal font by changing
\cs{rmdefault}.  In \LaTeX, the following command makes
\verb|garamond| the default normal font:

\begin{shortexample}
\renewcommand{\rmdefault}{garamond}
\end{shortexample}

\newpage
In Plain \TeX, it's written like this:

\begin{shortexample}
\def\rmdefault{garamond}
\end{shortexample}

\subsection{NFSS Pitfalls}

Defining your own fonts with the NFSS is 
straightforward, but it is not without its subtleties.  There are 
several things going on behind the scenes that you must be aware
of if you are going to define your own fonts.

The first consideration is font substitution.  When NFSS cannot find a
font, it tries to substitute a different one.  First, it tries to find the
font you requested in the default shape, then in the default series, and
finally, in the default family.  You can change the defaults used by NFSS2
with the \cs{DeclareFontSubstitution} command:

\begin{exindent}
  \verb|\DeclareFontSubstitution{|\texttt{\textit{encoding}}%
  \verb|}{|\texttt{\textit{family}}%
  \verb|}{|\texttt{\textit{series}}%
  \verb|}{|\texttt{\textit{shape}}%
  \verb|}|
\end{exindent}

You must specify the defaults for each encoding you use because
the encoding is never substituted.

The next consideration is the use of 
mathematics\index{mathematics!typesetting}\index{typesetting!mathematics},
which is typeset
using an entirely different set of fonts. Even if you don't use
any math in your document, the NFSS is prepared to function in math-mode
at a moment's notice.  In order to be prepared, it has to know what fonts
to use.  Ordinarily, mathematics is typeset using fonts at the same size
as the current text font.  This means that every time you change font sizes
for text, NFSS changes font sizes for mathematics as well.

By default, the NFSS  defines \idx{math fonts}\index{fonts!math} only 
at the following sizes:
5, 6, 7, 8, 9, 10, 11, 12, 14, 17, 20, and 25pt.  If you define a new
font at a different size, 24pt for example, the first time you
try to use that font, you will get several warning messages indicating
that \texttt{cmr/m/n/24} (Computer Modern Roman),
\texttt{cmm/m/it/24} (Computer Modern Math Italic), 
\texttt{cmsy/m/n/24} (Computer Modern
Math Symbols), and \texttt{lasy/m/n/24} (\LaTeX\ Symbols, if
using \LaTeX), are not available.
This is very confusing because
it won't appear that you have selected 24pt Computer Modern {\em anywhere}
in your document.  

There are two ways to solve this problem:\footnote{Under earlier versions
of the NFSS, this problem actually resulted in an error rather than a warning.
If you wish, you can simply ignore the warning; you no longer have to fix
the problem.}

\begin{itemize}
  \item Redefine the required fonts at the sizes requested.
        This involves copying the definition of each of the fonts
        from NFSS \ixx{\ext{FD} files}{FD files@\ext{FD} files} and adding new sizes.  

  \item Tell NFSS to use existing math sizes for the new text
        sizes that you define.  This
        option is reasonable only if you won't be using any math
        at the new sizes (or the sizes are so close that the mathematics
        doesn't appear disproportional).  NFSS includes a macro called
        \cs{DeclareMathSizes} for this purpose.  Insert it
        directly after the \cs{DeclareFontShape} command that
        declares the new font sizes.  You must call \cs{DeclareMathSizes}
        once for each new size.  For example, to use the 25pt math sizes
        with 24pt text,
        insert the following control sequence
        after you define the 24pt font:

\begin{shortexample}
\DeclareMathSizes{25}{24}{20}{17}
\end{shortexample}

        The general form of a call to \cs{DeclareMathSizes} is:

\begin{ttindent}
\verb|\DeclareMathSizes{|\textit{text-size}\verb|}|%
   \verb|{|\textit{math-size}\verb|}|%
   \verb|{|\textit{script-size}\verb|}|%
   \verb|{|\textit{script-script-size}\verb|}|
\end{ttindent}

        Where \textit{\texttt{math-size}},
        \textit{\texttt{script-size}}, and
        \textit{\texttt{script-script-size}} are the normal (123), 
        script ($x^{123}$), and script-script ($x^{y^{123}}$) math sizes
        for the specified \textit{\texttt{text-size}}.
\end{itemize}

\LaTeX\ sometimes resets the current
font to the \cs{rmdefault} font.  For example, it does this when a
\cs{ref} is going to be printed.  This means that the font you define as
the \cs{rmdefault} font should be available in every size that you use.
This may require redefining the Computer Modern Roman font, as described
above, if you add a new size but leave Computer Modern as the default font.

\section{PostScript Fonts Under NFSS}

PostScript fonts\index{PostScript!fonts!under NFSS2}\index{fonts!PostScript}
and other scalable font technologies like TrueType
differ from the way the ``standard'' \TeX\ fonts work.  They do not separate 
the notions of design size and magnification.  Instead, PostScript fonts 
can be rendered at \emph{any} size from a single design.  
In daily use, the PostScript fonts under NFSS are indistinguishable from
non-PostScript fonts. 
The NFSS distribution includes style files for accessing the 35 standard
PostScript fonts.

\subsection{Adjustments to Scale}

\index{fonts!scaling}Some 
combinations of PostScript fonts, particularly PostScript fonts
with Computer Modern mathematics, look bad because there is a large
discrepancy between the apparent sizes of the fonts.  For example, as a consequence
of design, 10pt Helvetica looks bigger than 10pt
Computer Modern Math Italic.  In order to correct this problem, 
you can specify a scaling factor when declaring PostScript 
fonts.\footnote{Actually, you can declare a scaling factor in any font
declaration, although it seems to make less sense for non-PostScript
fonts.}
The scaling factor is specified in square brackets at the beginning of
the font-shape declaration in the \cs{DeclareFontShape} command.

\newpage
The easiest way to find an approximation of the correct scaling factor is to
look at the \emph{x-height} of each font.\footnote{Usually, scaling fonts to
the same x-height makes them look acceptable together, but depending
on the particular fonts involved, a little more or a little less scaling may be
required to achieve a pleasant balance.}  The x-height is the
height of a lowercase ``x'' in the font.  
The following macro will print the x-height of a font:

\begin{shortexample}
\def\showxheight#1{%
  \font\fontfoo=#1 at 10pt%
  \message{The x-height of #1 at 10pt is \the\fontdimen5\fontfoo}}
\end{shortexample}

%The document in Example~\ref{ex:scaledhelv} 
The following \TeX\ input will print the x-heights of
Helvetica and Computer Modern Roman (assuming that your Helvetica font
is called \filename{phvr}):

%\begin{example}{ex:scaledhelv}{A \protect\TeX\ document to display the x-height of Helvetica and Computer Modern Roman.}
\begin{shortexample}
\input showxheight
\showxheight{phvr}
\showxheight{cmr10}
\bye
\end{shortexample}
%\end{example}

On my system, the x-height of Helvetica is 5.23pt and the x-height of
Computer Modern Roman is 4.30554pt.  The following font declaration
makes Helvetica have the same apparent size as Computer Modern Roman:

\begin{shortexample}
\DeclareFontShape{OT1}{phv}{m}{n}{%
  <-> [0.8232] phvr}{}
\end{shortexample}

Compare scaled Helvetica to unscaled Helvetica:

{%%%%%%%%%%%%%%%%%%%%%%%%%%%%%%%%%%%%%%%%%%%%%%%%%%%%%%%%%%%%
\DeclareFontFamily{OT1}{phv}{}
\DeclareFontFamily{OT1}{sphv}{}
\DeclareFontShape{OT1}{phv}{m}{n}{%
  <-> phvr}{}
\DeclareFontShape{OT1}{sphv}{m}{n}{%
  <-> [0.8232] phvr}{}

\begin{exindent}
  \rm\fontfamily{phv}\selectfont
  Unscaled Helvetica, 
  \fontfamily{cmr}\selectfont
  Computer Modern Roman,
  \fontfamily{sphv}\selectfont
  Scaled Helvetica.
\end{exindent}%
%%%%%%%%%%%%%%%%%%%%%%%%%%%%%%%%%%%%%%%%%%%%%%%%%%%%%%%%%%%%
}

Unscaled Helvetica looks much larger than Computer Modern, but the scaled
Helvetica appears a little too small.  Experimentation is the only way to find
the scale that looks best.

\section{When Things Go Wrong}

A number of font-related problems can arise\index{error messages!fonts (tex)@fonts (\TeX)} which either prevent you
from formatting and printing your document or cause the output to differ
from what you anticipated.  The following sections describe many common
problems and their solutions.

\subsection{When \protect\TeX\ Complains}

The first time an error can occur is when \TeX\ is processing your
document.  Some of these errors prevent \TeX\ from continuing while
others are simply warnings. 

\begin{iplist}{.25in}
\ipitem[\texttt{! Font \bs myfont=xxxxx not loadable: %
               Metric (TFM) file not found.}]

This error indicates that \TeX\ tried to process a \cs{font} control sequence which
assigned the font \texttt{xxxxx} to the control sequence \cs{myfont}, but \TeX\
could not find a \ext{TFM} file for the font \texttt{xxxxx}.  All the characters
from the missing font will be blank in the resulting \ext{DVI} file.

You cannot process your document until this error is corrected, which is 
a matter of fixing the offending \cs{font} command if 
you are using the old font selection scheme.

Under the NFSS2, this error occurs if you specify
an invalid font in the command \cs{DeclareFontShape}.  
Examine the font that you were attempting to select
and make sure that it exists.

\ipitem[\texttt{Warning Font/shape `x/y/z' undefined on input line \textit{n}}]

This is a warning message from the NFSS.  It indicates
that you requested the font family \texttt{x}, series \texttt{y}, and shape
\texttt{z},
but the requested font does not exist.  This message is followed by 
another indicating
which font NFSS chose to substitute in place of the one you requested.
NFSS substitutes the closest possible font to the one you requested.
It is usually an acceptable replacement.

For example, if you are currently using Computer Modern Bold Extended and
you select Computer Modern Typewriter, the NFSS will report that there
is no bold-extended-typewriter font and that normal typewriter is being
substituted in its place.

\ipitem[\texttt{! Font x/y/z/999 not found.}]

This is a fatal error from the NFSS.  It indicates
that you selected font family \texttt{x}, series \texttt{y}, shape \texttt{z}
at a size
of 999pts and that no such font exists.  This error occurs when
the size 999 is not defined in the \cs{DeclareFontShape} command
for x/y/z.  

\ipitem[\texttt{Missing character: There is no X in font foo!}]

This is a warning message from \TeX.  Usually it occurs only in the 
log file.  This error occurs when you attempt to access a character
that does not exist in the current font.
This can happen if you select the wrong font or if the selected font has a
different encoding vector than anticipated.  See the section
``\nameref{sec:fonts:encodingvec}'' in this chapter for more information.
\end{iplist}

\subsection{When the DVI Driver Complains}

Getting \TeX\ to successfully produce a \ext{DVI} file\index{error messages!fonts (DVI)} is only half
the battle.  The next hurdle is getting a \dvidriver\ to print it.
Here are some of the things that can go wrong:

\subsubsection{Can't find PK file}

When a \dvidriver\ complains that it cannot find the appropriate \ext{PK}
file, there are several things that could be wrong.

\begin{itemize}
\item The font is built into the printer.

If the \dvidriver\ complains that it cannot find the \ext{PK} file
for a built-in font, you need to adjust the \dvidriver's configuration
to indicate that the font is built-in.
Exactly how this 
is done depends on the \dvidriver\ that you are using.  For example,
if you are using \program{dvips}, add an entry to the 
\filename{psfonts.map}
file\index{psfontsmap files@psfonts.map files}.  If you are using \emTeX, add an entry to 
the font substitution
file specified in the configuration file.  Consult the
references for your particular \dvidriver\ for more information about
using built-in fonts.

\item The font does not exist on the printer you are using.

This is the same problem as the error above except that it cannot be
fixed.  For example, I have \ext{TFM} files for many PostScript fonts
at home because it helps me format bits of documentation that I bring
home from work without error.  However, if I try to print one of these
documents without first changing the fonts, the \dvidriver\ complains
that it cannot find several fonts.  There is no way that I can
correct this because the missing fonts are built-in fonts for a printer
that I do not have at home.  In this case, font substitution by the
\dvidriver\ may allow you to preview (and even print) the document,
but it won't look very good unless the font metrics of the substituted
font are very close to the metrics for the original.

\item Uncommon size or font (no \ext{PK} file).

The Computer Modern family contains a number of fonts that are very
rarely used.  If you don't keep \ext{PK} files for these fonts around
all the time and attempt to use one of them (or attempt to use
a common font at a very unusual size), the \dvidriver\ will not be
able to find the necessary \ext{PK} file.  You have to build the 
\ext{PK} file first.  Consult Chapter~\ref{chap:mf}
for more information about building \ext{PK}
files with \MF.  Also consult the section called ``\nameref{sec:autofont}''
later in this chapter.
\end{itemize}

\subsubsection{Accents don't work or the wrong characters are printed}

This problem is usually caused by a bad encoding vector.  See the
section ``\nameref{sec:fonts:encodingvec}'' in this chapter for 
more information.

\subsubsection{Printer prints the right characters but the wrong font}

This is usually an indication that you are trying to use a font that
does not exist on the printer.  PostScript printers, for example,
substitute Courier for any font that does not exist.

Another possibility is that you have configured your \dvidriver\ incorrectly.
I once told \program{dvips} to download Galliard when I used Garamond.
It took me quite a while to find that error.
Make sure that the printer contains the fonts that you think it does
and make sure that you are mapping the \TeX\ font names to the correct printer
fonts.

\newpage
\section{Encoding Vectors}
\label{sec:fonts:encodingvec}

An \ixx{encoding vector}{encoding vectors} describes 
the order and position of characters within a
font.  The purpose of this section is to help you understand the role that
encoding vectors play in the translation of text from your input file to
printed output.

Encoding vectors cause a lot of confusion.  Whenever the characters that
you type in your input file are printed incorrectly (``\texttt{flight}'' in
your input file prints
out as ``ight''; ``\verb|\oe{}vres|'' prints out as ``\'uvres''; or
``\verb|---|'' prints out as ``-{}-{}-'' instead of ``---''), you've probably
encountered some sort of encoding problem.

The root of this problem is that the characters in your input
file are really just numbers between 0 and 255.\footnote{In some
special versions of \TeX, notably those for handling languages like
Japanese, this may be extended to a much larger number, but the
problem is the same.}  The number 65 is usually a capital A, but
there is nothing intrinsic to the value 65 to signify this.  In order
to display these byte values, they have to be translated into symbols.

These problems arise because there are at least four different
translations occurring between what you type and what appears on the
printed (or previewed) page:

\begin{enumerate}
\item You type some symbols on the keyboard, and they are displayed by
your editor.  The configuration of your system determines how these characters
appear.  Sometimes they are from the ISO Latin-1 symbol set, sometimes US
ASCII, sometimes something else, and sometimes it is user configurable.  If
you confine yourself to pure ASCII, you're pretty safe, but that's not
convenient for languages other than English.  When you are typing documents
in Spanish, it's very convenient to be able to type ``\~n'' directly in your
document.

\item \TeX\ reads your input file and translates it into an internal
encoding (basically ASCII).  (The reverse translation is performed
before \TeX\ prints any output to the terminal or the log file.)
The translation tables used in this step are generally determined when
the \TeX\ program is compiled, but several modern \TeX{}s allow you
to modify these tables at runtime.

\TeX\ assumes that fonts use \TeX's internal encoding.  For example, on
an IBM mainframe, which uses the \idx{EBCDIC character set}, \TeX\ translates
a capital A (EBCDIC 193) into ASCII 65 and assumes that
position 65 of the metric
information for the current font contains the information
(including ligature and kerning information) for a capital A.

%This used to severly limit your ability to include characters outside
%of US ASCII in your input file.  Version 3 of \TeX\ addressed this shortcoming
%with virtual fonts, discussed in the next section.

\item If you use control sequences to represent special characters
(\cs{oe} for ``\oe'', for example), the macro package you 
use is responsible for defining those control sequences so they
produce the correct characters.  If the macro package 
assumes that the ``\oe'' character appears at position 247 of the current
font, the output will not be correct in a font with a
different encoding.

\item The \dvidriver\ reads \TeX's \ext{DVI} file and assumes that the
encoding vector of each font used in the \ext{DVI} file is the same as
the encoding vector of the actual fonts in the output device.  For
\ext{PK} fonts, this is probably true (since the \dvidriver\ sends the
font to the printer), but for fonts built into the printer, it is less
likely to be true.
\end{enumerate}

You may think of a font as a collection of 1 to 256 different
symbols in a particular order.  This is really a font plus a particular
encoding vector.  It is more accurate to think of a font as simply a
collection of symbols (with no particular number or ordering).  An encoding
vector selects which symbols are used and in what order they appear.
Typically, an encoding vector can contain only 256 different symbols.
It is important to note that changing an encoding vector of a font does not
simply permute the order of the characters in the font, it can change which
symbols are actually present as well.

Encoding vectors are either implicit or explicit.  Most fonts have an implicit
encoding, but some (for example, Adobe Type~1 fonts) contain an explicit,
configurable encoding.
\TeX\ fonts have an implicit encoding.  \TeX\ actually uses several different
encoding vectors (it uses more than one because it has both text or body
fonts and several kinds of math and symbol fonts).  The character set tables
in Appendix~\ref{app:fonts}, {\it\nameref{app:fonts}}, show the encoding of
several different fonts.

% Table~\ref{tab:cmrencode}
%shows the ordering of characters in the \TeX\ text encoding vector used
%for text fonts.  For comparison, Table~\ref{tab:font:corkencoding} shows the
%Cork encoding for international
%\TeX\ fonts, Table~\ref{tab:adbstdencode} shows the Adobe Standard encoding,
%and Table~\ref{tab:winansi} shows the Windows ANSI encoding.
%
%\input{ft-cmr.tab}
%
%\input{ft-cork.tab}
%
%\input{ft-adb.tab}
%
%\input{ft-ansi.tab}
%
%In order to talk about encoding vectors, a distinction is made between
%characters and ``glyphs.''  A {\em glyph} is a ``splotch of ink'', a symbol
%without any meaning.  A character, on the other hand, is a particular
%symbol: the upper-case letter A, an open parenthesis, and
%lambda are all {\em characters}.
%

\font\mathit=cmmi10
Most macro packages assume that the TEX TEXT encoding is being used.
For example, Plain \TeX\ defines the control sequence \cs{AE} as
an abbreviation for character number 29 in the current font.  Even
when control sequences aren't involved, problems can arise if the 
output font does not have the anticipated encoding.  The byte-value
34 in your input file is almost always the literal
double quote character  (\verb|"|) and usually prints as a double
quote (which is probably what your editor displays and probably what
you expect).  If the font encoding is something different, the result
will not be what you expect.  For example, if the current font has
the TEX MATH ITALIC encoding, the result is ``{\mathit\char34}''.

%No matter what encoding vector you use, the following four conditions
%must hold:
%
%\begin{enumerate}
%\item \ext{TFM} files have to be created with that encoding vector.
%\item Your \dvidriver\ must be aware of the encoding vector.
%\item Macros that refer to symbols in the font have to be redefined
%      to use the new encoding.
%\item System-specific metric information (like Windows \ext{PFM} files)
%      must be created with the correct encoding vector.
%\end{enumerate}
%
%Virtual fonts can help alleviate these problems.  See the
%``\nameref{sec:fonts:virtualfonts}'' section in this chapter.

\section{Virtual Fonts}
\label{sec:fonts:virtualfonts}

\ixx{Virtual fonts}{virtual fonts}\index{fonts!virtual} (stored 
in \ixx{\ext{VF} files}{VF files@\ext{VF} files}\footnote{For 
a complete description, consult \textit{The VP To VF
Processor}~\cite{texware:vptovf} and \textit{The VF to VP
Processor}~\cite{texware:vftovp}.}) are a relatively new addition to the \TeX\
family.  When \TeX\ was originally defined in 1970, 
Knuth\index{Knuth, Donald} chose a
character encoding that suited his purposes, and very little effort was made
to parameterize the encoding.  (In fact, any \TeX\ macro writer can use {\em
any\/} encoding she wants, but no general mechanism for identifying the
encoding exists.\footnote{Version 2 of NFSS includes ``encoding'' as a
parameter in font selection.  In the long run, this will allow macro writers
to hide many of these difficulties, dramatically reducing the
burden currently placed on the user.})  Virtual fonts combat this problem by allowing the creator
to define a virtual font in terms of (multiple) characters from one or more
fonts.

\newpage
Virtual fonts are used most commonly to change the encoding
vector of a font.  This provides a convenient way of mapping different
fonts into the required encoding so that they are easy to use in \TeX.
A virtual font consists of a \ext{VF} file and a \ext{TFM} file.  \TeX\
uses the \ext{TFM} file as it would any other.

Figure~\ref{fig:vfdiagram} shows how a virtual font is used by the \TeX\
system.  In this case, a virtual font \verb|ctmr| has been created
combining characters from the fonts \verb|trr0n|, \verb|trr6m|, 
\verb|trr6j|, and \verb|trr10j|.

\epsimage{fig:vfdiagram}{How \protect\TeX\ uses a virtual font}

The document uses the font \verb|ctmr|.  \TeX\ uses the
font metric information in \filename{ctmr.tfm} to compose the \ext{DVI} file.
The \dvidriver, however, discovers a \ext{VF} file named \verb|ctmr|, so it
uses the instructions in the virtual font to select characters from the four
built-in fonts.  This is a practical example.  It
shows how the HP LaserJet built-in fonts may be accessed with the standard
\TeX\ text encoding.

In practice, virtual fonts suffer from one limitation:  A virtual font can
only permute the encoding vector; it cannot access characters that do
not appear in the encoding vector of the ``real'' font.  For example, an
Adobe Type~1 font might contain all of the characters that appear in the
\TeX\ text encoding, but many of them do not appear in the standard 
Adobe encoding vector.  A virtual font alone cannot remap the characters 
into the \TeX\ text encoding.  The encoding vector of the font must also
be changed.  Several PostScript \dvidriver{}s have this ability.

\newpage
Virtual fonts are binary files and are very difficult to edit.
To make it possible to construct virtual fonts by hand, \TeX\
includes standard utilities (\ixx{\program{VFtoVP}}{VFtoVP} and 
\ixx{\program{VPtoVF}}{VPtoVF}) for
converting the binary \ext{VF} format into a human-readable \ext{VPL}
format.  The \ext{VPL} files are text files that can be edited with
a text editor.

A complete description of how to create virtual fonts is beyond the scope of
this book.  If you are interested in experimenting with virtual fonts, I
strongly recommend that you examine the \program{fontinst}
package.\footnote{You can retrieve the \program{fontinst} package from the
CTAN archives in the directory \ctandir{fonts/utilities/fontinst}.}
\ixx{\program{fontinst}}{fontinst} allows you 
to construct 
\ext{VPL} files with \TeX\ documents.  
This isn't as strange as it first sounds.  Virtual font
files have to include the metric information from \ext{TFM} files for each
font that they include.  Because \TeX\ can easily read this information, and
the output is a plain text document anyway, \TeX\ is a very capable, if
somewhat slow, tool.

\section{Automatic Font Generation by DVI Drivers}
\label{sec:autofont}

Recent releases of many popular \dvidriver{}s (\ixx{\dvips}{dvips}, 
\ixx{\xdvi}{xdvi}, \ixx{\SeeTeX}{SeeTeX@\SeeTeX},
and \ixx{\emTeX}{emTeX@\emTeX}'s \dvidriver{}s, to name a few) include the ability
to generate missing fonts 
automatically\index{fonts!automatic generation}\index{DVI drivers!automatic  font generation}.  
Automatic font generation overcomes two problems simultaneously: it
reduces disk space requirements, and it makes font generation easier.
Most implementations of \TeX\ are distributed with a complete set of
Computer Modern fonts at seven or more magnifications.  This can easily
amount to several megabytes of disk space (even more if you are using
other fonts, like the \ixx{\AmS\ fonts}{AmS fonts@\AmS\ fonts} from the 
\idx{American Mathematical Society}).  In practice, you probably use only a small subset of these
fonts.

One way to combat the disk-space problem is to delete all of the \TeX\
fonts on your system and build just the ones you actually use.
In the days before automatic font generation, this would have
been quite unpleasant.  The first time you discovered that you needed
a font that you did not have, you would have to stop your \dvidriver,
figure out what size was missing, figure out how to get \MF\ to
build the font that was missing at the appropriate resolution,
run \MF, store the font in the right place, and return to your
\dvidriver.  Moments later, you might discover that you had to do
the whole process again for some other font.

Using a driver that provides automatic font generation makes it nearly
painless to delete fonts and let them be built automatically.\footnote{It's
still a bit tedious, at least on slow machines, but now you can walk away and
get a cup of coffee ;-).  All the fonts will be built automatically.}
The \dvidriver\ determines the resolution required, runs \MF\
with the appropriate parameters, stores the font in the correct
place, and continues processing the document uninterrupted.

The down side of automatic font generation is that you must keep
the \MF\ program around and make available the source files
for the Computer Modern fonts (and any other \MF\ fonts that you
use).  If you do not already have these files on your
hard disk, the potential disk space savings are  somewhat reduced.
(\MF\ is discussed fully in Chapter~\ref{chap:mf}.)
Moreover, automatic font generation
can only build \ixx{\ext{PK} files}{PK files@\ext{PK} files}; if \ext{TFM} files are also missing,
you will have to build those by hand before \TeX\ can process your
document.\footnote{Recent versions of \TeX\ derived from the 
\Web2C package can build \ext{TFM} files automatically as well.}

Another benefit of automatic font generation is that it can be used to
provide previewers, even non-PostScript previewers, with the ability
to preview documents that use PostScript fonts, provided that you have
the \ixx{Printer Font ASCII (\ext{PFA})}{Printer Font  ASCII (PFA)} or 
\ixx{Printer Font Binary (\ext{PFB})}{Printer Font Binary (PFB)} font sources 
and \ixx{Adobe Font Metric (\ext{AFM})}{Adobe Font Metric (AFM)} for 
your PostScript fonts.

The following lines, added to the \ixx{\filename{MakeTeXPK}}{MakeTeXPK} script
distributed with \program{dvips}, provide automatic font generation
for PostScript fonts with \ixx{\program{ps2pk}}{ps2pk}.  This is useful even on
systems where PostScript printers are used for output, because previewers
like \ixx{\program{xdvi}}{xdvi} also use \program{MakeTeXPK} to build missing
fonts.

\begin{shortexample}
# Look for a PostScript outline font...
if [ -r /usr/local/lib/tex/ps/outlines/$NAME.pfa ] 
then
  echo Building TeX font from PostScript outline
  # Hack.  If $6 is null, $DESTDIR => $6 ...
  PStoTeXfont $1 $2 $3 $4 $5 $6 $DESTDIR
  exit 0
else
  echo Building TeX font from MetaFont outline
fi
\end{shortexample}

In this example, PostScript fonts are stored in the 
\filename{/usr/local/lib/tex/ps/outlines} directory.
You should change this directory to something appropriate for your system.
On a \Unix\ system, the \program{PStoTeXfont} script shown in 
Example~\ref{ex:pstotexfont} is appropriate.\footnote{OK, so not
\emph{every} example in this book is in \program{Perl}.}  A \program{Perl}
version of \program{MakeTeXPK} that handles both \MF\ and PostScript
fonts is shown in Example~\ref{ex:maketexpk} in Appendix~\ref{app:examples}, 
\textit{\nameref{app:examples}}.

\begin{example}{ex:pstotexfont}{The PStoTeXfont script}
#!/usr/local/bin/bash
#
#   This script file makes a new TeX font from a PS outline.  
#
#   Parameters are:
#
#   name dpi bdpi [mag mode destdir]
#
#   `name' is the name of the font, such as `ptmr'.  `dpi' 
#   is the resolution the font is needed at.  `bdpi' is 
#   the base resolution.
#
#   This script ignores the remaining parameters.  They are 
#   left here to document the fact that the caller may provide 
#   them.  They may be provided because the caller thinks 
#   MetaFont is going to do the work...
#
#   Of course, this needs to be set up for your site.
#
# TEMPDIR needs to be unique for each process because of the 
# possibility of simultaneous processes running this script.

TEMPDIR=/tmp/temp-tex-PS.$$
NAME=$1
DPI=$2
BDPI=$3

LOCALDIR=/usr/local/lib/mf/fonts
DESTDIR=$LOCALDIR/pk

BASENAME=$NAME.$DPI
PFADIR=/usr/local/lib/tex/ps/outlines

# Clean up on normal or abnormal exit
trap "cd /; rm -rf $TEMPDIR" 0 1 2 15

mkdir $TEMPDIR
cd $TEMPDIR

# We proceed by making a 10pt font at the resolution 
# requested...
echo Making ${DPI}dpi version of $NAME.
ps2pk -X$DPI -P10 -a$PFADIR/$NAME.afm \
    $PFADIR/$NAME.pfa ${BASENAME}pk

mv ${BASENAME}pk $DESTDIR

exit 0
\end{example}

A similar script for using automatic font generation under \emTeX\ with \program{4DOS}
is shown in Examples~\ref{ex:dvidxx} and~\ref{ex:makepk} in
Appendix~\ref{app:examples}.
Several other options are available for performing automatic font generation
on a number of platforms, including a \program{REXX} version for OS/2 and a compiled
program called \program{MKTeXPK}.

\section{Math Fonts in \protect\TeX}

Changing \idx{math fonts}\index{fonts!math} is 
more difficult than changing text fonts.  In addition
to the large number of special symbols that must be available, \TeX\ needs a
lot more information to use the fonts because the characters are combined more
frequently and in more complex ways.  For example, the open brace character
(\{) in math mode is ``extensible;'' this means that it can be as large as
required.  In order for \TeX\ to construct a brace of arbitrary size, one of
the math fonts (the math extensions font) contains four different characters
that \TeX\ combines to form the brace:

\begin{exindent}
\font\mathex=cmex10
{\mathex\char'070}, {\mathex\char'102}, 
{\mathex\char'074}, and {\mathex\char'072}
\end{exindent}

Extensible recipes for characters like ``\{'', ``\}'', ``('', and ``)''
are examples of additional metric information that must be available in
math fonts for \TeX.

In addition to the Computer Modern math fonts, there are really only
three other choices at present: the \ixx{\AmS\ fonts}{AmS fonts@\AmS\ fonts} 
(\MF\ fonts freely
distributed by the \idx{American Mathematical Society} which extend but do
not replace the Computer Modern math fonts), the \idx{Lucida Bright+New Math fonts}, and the \idx{MathTime fonts}.
Lucida Bright and MathTime are both sets of commercial PostScript Type~1
fonts.

\section{Concrete Examples}

The following sections describe by example how you can use several
different kinds of fonts in \TeX.  The tools described are generally
free and generally available for multiple platforms.  Where specific
commercial tools are used, free alternatives are discussed.  

The specific tools I mention here are not
the only tools available nor are they necessarily the best,
although I hope I've found the best ones.  If you have
found a different and better solution, don't abandon it in favor of what
I use here.  But please do tell me about the method that
you have found.  With every passing day, more free software becomes
available.  

\subsection{\protect\MF\ Fonts}

Chapter~\ref{chap:mf} describes how to use
\MF\ to create fonts for \TeX.\index{metafont@\MF!fonts}\index{fonts!metafont@\MF} If 
you have \MF\ installed, you can
easily create fonts that are usable in \TeX.  

\MF\ reads \ext{MF} files, which are plain text files that describe
a font analogous to the way \TeX\ reads \ext{TEX} files that
describe a document.

Appendix~\ref{app:fonts}, \textit{\nameref{app:fonts}}, contains
examples of many \MF\ fonts.  Consult the ``Definitive List of All
Fonts Available for {\MF}''~\cite{lreq:metafonts} for an up-to-date
list with availability information.

\subsection{PostScript Type 1 Fonts}
\label{sec:t1fonts}

PostScript printers have many PostScript 
Type~1\index{PostScript!fonts}\index{fonts!PostScript} fonts 
built in.  If you want
to use built-in fonts, you only need the metric information for them
and  a PostScript \dvidriver. The \program{dvips} driver is the most
popular free \dvidriver.  Several commercial drivers are also available.

The metric information should be available in the form of Adobe Font
Metric (\ext{AFM}) files\index{AFM files@\ext{AFM} files} from the printer vendor or directly from
\idx{Adobe Systems}, Inc.  You need a program that will convert \ext{AFM}
files into \ext{TFM} files\index{TFM files}.  There are several free programs that
will do this conversion (one is included with \program{dvips}), and
your \dvidriver\ may have included one.  In general, if one came with
your \dvidriver, that is the one you should use.

If you want to use PostScript Type~1 fonts on non-PostScript devices,
like most screen previewers, you need the ``sources'' for the fonts
that you want to use.  Many font vendors sell fonts in Adobe Type~1
format.  In addition, there are many Type~1 fonts available on the
Internet and on bulletin board systems.\footnote{Before you use these 
``free'' fonts, be aware that many are of questionable
legality.}

Several companies have made complete, high-quality fonts available.
They are: 

{\LTleft=.25in%
\begin{xtable}{ll}
  IBM Courier       & URW Antiqua \\
  Bitstream Courier & URW Grotesk Bold \\
  Bitstream Charter & Nimbus Roman No9 \\
  Adobe Utopia      & Nimbus Sans 
\end{xtable}%
}

Check the license that accompanies these fonts to make sure that you can use
them legally.

PostScript Type~1 font sources are available in 
\ixx{Printer Font ASCII (\ext{PFA})}{Printer Font ASCII (PFA)} and 
\ixx{Printer Font Binary (\ext{PFB})}{Printer Font Binary (PFB)}
formats.\footnote{PostScript fonts for the Macintosh are stored in
\ext{PFB} format (essentially) in the resource fork of the printer
font file.  The metric information is stored in the screen font, which
also contains bitmaps for on-screen display.}  The \ext{PFB} format is
more compact, but less portable.  \Unix\ systems usually use \ext{PFA}
files, while MS-DOS and OS/2 systems use \ext{PFB} files.  Several existing
programs can convert \ext{PFB} files into \ext{PFA} files
and vice-versa.  For the remainder of this section, I will
consistently refer to PostScript Type~1 source files as \ext{PFA}
files, although you can use \ext{PFB} files instead if they are
supported on your platform.

In addition to the \ext{PFA} files, you will also need metric information
for the fonts.  The metric information {\em should\/} be available in
\ext{AFM} format.  If you purchased fonts from a vendor and did not receive
\ext{AFM} files, you should complain.  Fonts from free sources, like
Internet archive sites and bulletin board systems sometimes include only
\ixx{Printer Font Metric (\ext{PFM})}{Printer Font Metric (PFM)} files.  
These are Microsoft Windows
printer metric files, and they do not contain enough information to 
make a \ext{TFM} file.  

It is possible to create a \ext{TFM} file from the \ext{PFM}
file, but the metrics are not particularly good.  To do this:

\begin{enumerate}
  \item Convert the \ext{PFM} file into an incomplete \ext{AFM} file
        with the \ixx{\program{PFM2AFM}}{PFM2AFM} utility.
  \item Use the \ixx{\program{PS2PK}}{ps2pk} program to make a \ext{PK} file.  
  \item Use \ixx{\program{PKBBOX}}{PKBBOX} to create a more complete \ext{AFM} file
        from the incomplete \ext{AFM} file and
        the \ext{PK} file.  
\end{enumerate}

The \ext{AFM} file manufactured in this way can be
used to create a \ext{TFM} file.
The \ext{PK} file can be used by any \dvidriver\ that understands
\TeX\ \ext{PK} fonts (almost all drivers have this ability).

Creating \ext{PK} files does require more disk space, but it has the
advantage that you can print \TeX\ documents which use PostScript Type~1
fonts on non-PostScript devices.  This includes fast, \ext{PK}-based
screen previewers like \program{xdvi} and \emTeX's \program{dviscr}
that do not understand PostScript.

Programs like \ixx{\program{dvipsone}}{dvipsone} and \ixx{\program{dviwindo}}{DVIWindo} which 
run under
the Microsoft Windows environment can use PostScript fonts directly if
\idx{Adobe Type Manager (ATM)} is installed on the system.  However, it is
still necessary to construct the \ext{TFM} files for \TeX.

\subsubsection{Using a new PostScript font in \protect\TeX\ (for
PostScript printers)}

This section presents a step-by-step description of how
to use a new PostScript font in \TeX\ with a PostScript printer or previewer.
In this situation, you have a PostScript
printer, and you use \program{dvips} to print your documents.  This
method also allows you to preview your documents with
\program{Ghostscript}.

For either method, you must first obtain the PostScript sources for
the font that you want to use.  You {\em must\/} have a \ext{PFA} or
\ext{PFB} file and an \ext{AFM} file.  As a concrete example, I'll use
the ``Nimbus Roman 9L Regular'' font.  For this font, I obtain
\filename{unmr.pfa} and \filename{unmr.afm}.  These fonts are
available from the CTAN archives in the directory
\ctandir{fonts/urw}.\footnote{The Nimbus fonts actually come with
\ext{TFM} and \ext{VF} files which makes some of the following steps
redundant.  But because most fonts \emph{don't} come with \TeX\ metrics,
the example is important.}
In order to use the font in \TeX, you must create \TeX\ font metrics
for it using \ixx{\program{afm2tfm}}{afm2tfm} (in particular, the version of
\program{afm2tfm} that comes with \program{dvips}).  

First, however, we must decide what encoding vector to use.
Frequently, the fonts you obtain use Adobe Standard Encoding.
The problem with this encoding is that it isn't very complete; it
leaves out a lot of standard \TeX\ characters (like the ligatures
``fi'' and ``fl'' as well as many accented and international letters).
Instead of using \idx{Adobe Standard Encoding}, I recommend using 
the \idx{Cork Encoding}.  The Cork Encoding has 
several advantages; it is a superset
of the original \TeX\ text encoding; it is becoming a new standard for
\TeX; and it is supported by the NFSS2.  Of course, the Cork Encoding
is not suitable for all fonts; there's no reason to try to re-encode a
symbol font into the Cork Encoding---that doesn't even make any sense.

Luckily, using the Cork Encoding is no more difficult than using
whatever encoding the distributed font contains.  \program{afm2tfm}
will do all the work.  You only need to obtain the appropriate encoding
file.  In this case, the file is \filename{ec.enc}, and it is distributed
with both \program{dvips} and NFSS2.
Use \program{afm2tfm} to generate the metrics:

\begin{ttindent}
\$\ {\bf afm2tfm unmr.afm -v unmr.vpl -T ec.enc unmr0.tfm}
\end{ttindent}

This command reads \filename{unmr.afm}, the original \ext{AFM} file with
an arbitrary encoding, and the encoding vector \filename{ec.enc}.  It
creates the virtual font \filename{unmr.vpl} and the \ext{TFM} file
\filename{unmr0.tfm}.

The relationship between these files is subtle.  The \ext{AFM} file
contains metric information for all the possible \idx{glyphs} in the font.
The encoding file establishes the encoding vector---which particular
characters occur in exactly what order.  These two files are combined
to produce character metric information for the specific encoding
vector.  This information is saved in the \ext{TFM} file.  This is a
``raw'' \ext{TFM} file\index{TFM files!raw}.\footnote{Raw \ext{TFM} files 
used to be identified
with an ``r'' prefix, but recently the trend has turned towards a ``0''
suffix.  For a more complete discussion of font names, consult
\textit{Filenames for Fonts}~\cite{tug:filenames-fonts}.}
It does not have any
ligature or kerning information and \emph{may} have a different
encoding from the virtual font (although, in this case it has 
the same encoding).

The next step is to produce a virtual font from the \ext{VPL} 
file\index{VPL files}:

\begin{ttindent}
\$\ {\bf vptovf unmr.vpl unmr.vf unmr.tfm}
\end{ttindent}

This produces a virtual font file, \filename{unmr.vf}, and an appropriate
\ext{TFM} file.  This \ext{TFM} file has ligature and kerning
information for the characters in the font as well as the metrics for
the individual glyphs.

The names of the virtual font files and the related \ext{TFM} files
are entirely arbitrary.  You can give them any names you wish.  In the
long run, you will benefit if you choose a naming scheme that allows
you to determine which files are which simply by examining the names.
If you have a lot of fonts, take a look at \emph{Filenames for
Fonts}~\cite{tug:filenames-fonts}.  It is also available electronically
from CTAN in the directory \ctandir{info/filename}.

Now you should install the \ext{VF} and \ext{TFM} files in the appropriate
directories and proceed to use the \texttt{unmr} font.  \TeX\ will
do the right thing because the \ext{TFM} file contains the appropriate
metric information, and the \dvidriver\ will do the right thing because
it has a virtual font which specifies how the characters should be mapped
into the printer.  

Now that \TeX\ is happy, we have the additional problem of making the
PostScript printer happy.  The easiest way to do this is to tell
\program{dvips} to do it for us.  The \filename{psfonts.map} file used
by \program{dvips} identifies which fonts are built into the printer and
which fonts need to be downloaded. 

Each line in the \filename{psfonts.map} file describes how a particular
\TeX\ font should be interpreted.  The simplest lines identify the PostScript
name of fonts built into the printer.   For
\linebreak
\newpage
example, the following line
indicates that \ixx{\program{dvips}}{dvips} should use the PostScript font Times-Roman
(which is assumed to be resident in the printer) everywhere that the 
\ext{DVI} file uses the font \texttt{rptmr}:

\begin{shortexample}
rptmr Times-Roman
\end{shortexample}

To automatically download a 
PostScript font\index{fonts!PostScript!downloading}, 
add \verb|<|\texttt{\textit{fontfile}}
to the corresponding line in the \filename{psfonts.map} file.  The following
entry indicates that the PostScript font CharterBT-Roman should be used
where \texttt{rbchr} is used in the document.  In addition, \program{dvips}
should download the font from the file
\filename{/usr/local/lib/fonts/psfonts/bchr.pfa} if it is used in the
document.

\begin{shortexample}
rbchr CharterBT-Roman </usr/local/lib/fonts/psfonts/bchr.pfa
\end{shortexample}

Additional PostScript commands can be added to the entry to perform special
effects.  Some of these are described in the documentation for
\program{dvips}. They require a knowledge of PostScript that is beyond the 
scope of this book.

Adding the following line to \filename{psfonts.map} will download and use the
Nimbus URW font we installed above. (It is shown on two lines only because of
the constraints of the page; you should enter it on one line):

\begin{shortexample}
unmr0 NimbusRomanNo9L-Regular <unmr.pfa
      "ECEncoding ReEncodeFont" <ec.enc
\end{shortexample}

This line identifies the font \texttt{unmr0} as the NimbusRomanNo9L-Regular
PostScript font,  re-encoded with the ECEncoding
described in the file \filename{ec.enc}.  Because this font is not resident
in the printer, you must also tell \program{dvips} to download the font
from \filename{unmr.pfa}.  If you keep encoding files or PostScript fonts
(\ext{PFB} or \ext{PFA} files) in nonstandard locations, you will have
to specify the full path of \filename{ec.enc} and/or \filename{unmr.pfa}.

Notice that you specify the \emph{raw} font in the \filename{psfonts.map}
file.  \TeX\ and \program{dvips} will use the virtual font to determine
which character(s) from which raw font(s) should \emph{actually} be used
to print your document.  \program{afm2tfm} prints the line that should
be added to the font map file when it is finished converting the font,
so you don't always have to remember which is which.

The \filename{psfonts.map} file (and the encoding and font files) are
typically stored in a system-default location.  On \Unix\ systems,
this is frequently \filename{/usr/local/lib/tex/ps}.  If you
can't (or don't want) to change files in this directory, you can use
your own font map file. 

When you run \program{dvips}, it loads a initialization file 
(typically \filename{\tttilde/.dvipsrc}).  
If you put the following line in that file:

\begin{shortexample}
p +/home/jdoe/myfonts.map
\end{shortexample}

it extends the system-wide font map using the file
\filename{/home/jdoe/myfonts.map}, which has the same format as the
\filename{psfonts.map} file.

\newpage
If you use the \verb|<font.pfa| syntax in your font map file to download
PostScript fonts, you may discover that \program{dvips} is producing very
large output files (or takes a long time to print, if output is going
directly to a printer).  The reason is that \program{dvips} is downloading
the font at the beginning of every document that uses it.  If you have
a few fonts that you use all the time, it may be faster and more convenient
to download the fonts manually at the beginning of the day and then remove
the \verb|<font.pfa| portion of the font mapping line (leave the 
encoding file, however).  This will produce smaller output files because
\program{dvips} will assume that the fonts are already downloaded.
Of course, the trade-off is that your documents will not print correctly
(because the appropriate fonts are not attached to the PostScript file)
if you mail them to a colleague who doesn't download the same fonts you
do, or if you forget to download the fonts again after someone power-cycles
the printer.\footnote{Some spooling software will transparently handle
these problems (by downloading fonts that you forget) if the fonts
are available.}

\subsubsection{Using a new PostScript font in \protect\TeX\ (for
non-PostScript devices)}

Using PostScript fonts for screen previewing or printing on non-PostScript
printers is a very different process from printing on PostScript devices.
You still need the \ext{AFM} and \ext{PFA} files.

This time you're going to create \ext{PK} files with \ixx{\program{ps2pk}}{ps2pk}, so
virtual fonts are less useful.  In fact, you have to give the \ext{AFM} file
the correct encoding in order to get the right \ext{PK} file, so we'll
avoid virtual fonts altogether.  Of course, if you need several different encodings
(perhaps \TeX\ Text, Cork, and Adobe Standard) for different documents, you'll
be better off with one or two raw font files and several virtual fonts, but
for the moment let's imagine that that is not the case.

Example~\ref{ex:encafmpl} in Appendix~\ref{app:examples},
\textit{\nameref{app:examples}}, is a Perl script that changes the encoding
vector in an \ext{AFM} file to reflect the encoding specified in an encoding
file (like the ones used in the preceding section).  This is an important step
because \program{ps2pk} uses the encoding in the \ext{AFM} file to determine
which glyphs to render.\footnote{This step isn't necessary when using a
PostScript printers because the PostScript font always contains all the
glyphs.  The font created by \program{ps2pk} will only contain the characters
in a single encoding.}

To install this script, save it in a file called \filename{enc-afm.pl} and
change the top of the script (the \verb|#!| line) to reflect where Perl is
installed on your system.  Users of Perl on non-\Unix\ systems may have to
work a little harder, or simply use the syntax \textit{perl
enc-afm.pl} to run the script.  \Unix\ users can create a symbolic link to
\filename{enc-afm.pl} called \filename{enc-afm}, and mark it as executable.

Using this script, transform \filename{unmr.afm} into \filename{unmrX.afm}:

\begin{ttindent}
\$\ \textbf{enc-afm unmr.afm ec.enc > unmrX.afm}
\end{ttindent}

\newpage
Now the \filename{unmrX.afm} file has the Cork encoding.  We can use this
file to create a \ext{TFM} file for \TeX:

\begin{ttindent}
\$\ \textbf{afm2tfm unmrX.afm unmr.tfm}
\end{ttindent}

This \ext{TFM} file should be moved to the directory where you store \ext{TFM}
files for \TeX.  

Unlike PostScript fonts on PostScript devices, different \ext{PK} files must
be created for each size required.  As a concrete example, let's assume we
want a 14pt font.
The easiest way to create a 14pt font with \program{ps2pk} is to use the
\textit{-P} parameter to specify the size.  Unfortunately, this will
cause some \dvidriver{}s to complain because \textit{-P} sets the design
size of the \ext{PK} file, and it won't match the design size in the \ext{TFM}
file.  A better solution is to determine the resolution you need with the
following formula: 

$$\hbox{resolution} = {{\hbox{base resolution} * \hbox{point size}}\over{10}}$$

The factor 10 is used because it is the nominal design size of PostScript
fonts (at least, that's what  \program{afm2tfm} uses).  For example, if
we are creating a 14pt font for a 300dpi laser printer, the desired resolution
is $300*14/10 = 420$.  Now we can create the font:

\begin{ttindent}
\$\ \textbf{ps2pk -X420 -aunmrX.afm unmr.pfa unmr.420pk}
\end{ttindent}

This command creates a 420dpi \ext{PK} file using the \ext{AFM} file
\filename{unmrX.afm}.  The font source comes from \filename{unmr.pfa}, and
the resulting \ext{PK} file is called \filename{unmr.pk}.  On MS-DOS
systems, name the output \ext{PK} file \filename{unmr.pk} because
\filename{unmr.420pk} is not a valid filename.  Move the resulting \ext{PK}
file to the appropriate directory.  On \Unix\ systems, it is probably
\filename{/usr/local/lib/tex/fonts/pk}; while on MS-DOS and other systems,
it is probably something like \filename{/texfonts/420dpi}.

\subsection{HP LaserJet Softfonts}

Bitmapped HP LaserJet Softfonts\index{fonts!LaserJet softfonts} can 
easily be converted into \TeX\
\ext{PK} files with the \program{SFPtoPK} utility.  Then they can
be used with almost any \dvidriver.

Scalable LaserJet Softfonts can be used by \TeX\ if two conditions are
met: metric information is available, and the \dvidriver\ that you are
using can access built-in printer fonts.  Metric information for Scalable
LaserJet Softfonts distributed by Hewlett-Packard come in
the form of Tagged Font Metric files.  Tagged Font Metric files 
have the extension \ext{TFM}, but they should not be confused
with \TeX\ \ext{TFM} files.  The utility \program{hptfm2pl} converts
Tagged Font Metric files into \TeX\ \ext{PL} files (\ext{PL} is
a human-readable text format of \ext{TFM} files).  \ext{PL} files
are transformed into \ext{TFM} files by the standard \TeX\ utility
\program{PLtoTF}.

To use Scalable LaserJet Softfonts, you must convert the metric
information into \TeX\ \ext{TFM} files so that \TeX\ can use the
font, and then you must inform your \dvidriver\ that the fonts
are built into the printer.  Before printing your document, download
the Scalable LaserJet Softfont to your printer.  The \program{sfload}
utility that is part of Sfware can download Scalable LaserJet Fonts.
Many other free and shareware font downloading programs exist, too.

\subsection{TrueType Fonts}

At present, TrueType fonts\index{fonts!TrueType} can 
be used only on systems that have built-in
TrueType support (the Macintosh and MS-DOS computers running Microsoft
Windows 3.1 fall into this category).  You must use \dvidriver{}s that
communicate with the printer through the system's printer interface.

\footnotetext{This\label{footnote:answer} is the answer to the
question posed on page~\pageref{footnote:question}: ``$\backslash${\tt
fontshape\ttopenbrace it\ttclosebrace}$\backslash${\tt selectfont}''}

% Local Variables: 
% mode: latex
% TeX-master: "driver"
% End: 
