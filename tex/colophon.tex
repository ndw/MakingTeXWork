\RCSID$Id: colophon.tex,v 1.1 2002/08/23 14:58:46 nwalsh Exp $
\cleardoublepage
\pagestyle{empty}

\begin{center}
\Large \textit{About the Author}
\end{center}
\vskip4pt
\hrule
\vskip6pt

Norm Walsh is a Production Tools Specialist at O'Reilly and Associates'
Cambridge office. Before Norm joined ORA, he was a research assistant
at University of Massachusetts, Amherst where he earned his master's degree in computer science.

Besides maintaining a number of \TeX\ and font-related resources on
the Net, Norm enjoys bicycling, herpetology, prestidigitation, and
browsing record and book stores. Norm lives in Boxborough, MA with
his wife Deborah, two cats, a box turtle, two toads, and two frogs.

\begin{center}
\Large \textit{Colophon}
\end{center}
\vskip4pt
\hrule
\vskip6pt

Our look is the result of reader comments, our own experimentation,
and feedback from distribution channels.

Distinctive covers complement our distinctive approach to
technical topics, breathing personality and life into potentially
dry subjects. \Unix\ and its attendant programs can be unruly beasts.
Nutshell Handbooks\regtm\ help you tame them.

The animal featured on the cover of \textit{Making \TeX\ Work} is
the European garden spider. Garden spiders are orb weavers,
known for their intricate orb-shaped webs. Web building is a
complex process: support lines are constructed first; then the
radial lines; and finally the spiraling strands are spun from
the center outward. Without training from adults, even the tiniest
just-hatched spiderlings are able to spin silk and weave webs.
Another orb weaver, \textit{nephila}, builds an extremely thick and
strong web, up to eight feet in diameter. People in Southeast Asia
have found an interesting use for this spider's web--they bend
a pliable stick into a loop and pass it through the large web,
resulting in a surprisingly strong and effective fishing net!

Spiders produce silk from glands called spinnerets. Orb weavers can have
three or four pairs of these glands, each producing different textures
of silk: non-stick silk for the radial web lines, and sticky silk
for the spiraling strands. Some spiders even produce an ultraviolet silk
to attract insects. Spider silk, a super protein that hardens as it
is stretched from the spinnerets, may look delicate but it is unbelievably
tough. The relative tension necessary to break it is far greater
than for steel.

When finished with the construction of its web, the garden spider will often
go to the center, hang upside down, and wait for a flying or jumping insect
to become ensnared. Having poor eyesight, orb weavers rely on a
highly-developed sense of touch. When an insect becomes caught in the web
and struggles, the spider is alerted by the vibrations. It rushes out
to secure its prey, usually wrapping it in silk. A poison is injected
into the 
\linebreak
\newpage
victim, paralyzing it and converting the contents of its body
to liquid. The spider returns, later, to insert its tube-like fangs
and suck up its meal.

The garden spider's profound sense of touch has another purpose: it provides
male spiders with a channel to communicate with females. Before climbing onto
the female's web, the male taps out a special message. Then he cautiously
crawls toward his mate--a perilous task, for he is always in danger of being
mistaken for prey. It is commonly thought that the female spider kills
and eats the male after mating, but this is an exaggeration. The male,
who stops eating during his mate-hunting ordeal, generally dies of
malnourishment and exhaustion.

Spiders are similar to, but not the same as insects. They belong to
the class \textit{Arachnida}, named after Arachne, a maiden in Greek mythology.
She defeated the goddess Athena in a weaving contest. In a fury of anger,
Athena destroyed Arachne's weaving and beat the girl about the head.
In utter disgrace, Arachne hanged herself. A regretful Athena changed
Arachne into a spider so that she could weave forever.

While they are certainly not going to win any popularity contests, spiders'
insect-eating habits are extremely helpful to humans. Every year, billions of
spiders do away with a large number of disease-carrying and crop-destroying
insects. If every spider ate just one a day for a year, those insects,
piled in one spot, would weigh as much as 50 million people. Spiders are,
by far, the most important predator of insects in our world.

Edie Freedman designed this cover and the entire \Unix\ bestiary that
appears on other Nutshell Handbooks. The beasts themselves are adapted
from nineteenth century engravings from the Dover Pictorial Archive.
The cover layout was produced with QuarkXPress 3.1 using the
ITC Garamond font.

The inside layout was designed by Edie Freedman and implemented
by Norm Walsh in \TeX\ using the ITC Garamond font family.
The figures (except those in chapter 6) were created in Aldus Freehand 3.1
by Chris Reilley. The colophon was written by Elaine and Michael Kalantarian.
