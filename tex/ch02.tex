\chapter{Editing} 
\RCSID$Id: ch02.tex,v 1.1 2002/08/23 14:58:45 nwalsh Exp $
\label{chap:writing}
\label{chap:editing}

\ifincludechapter\else\endinput\fi

This chapter describes several writing environments available for
creating and modifying \TeX\ documents.  In practice, you can use almost
any \ixx{editing program}{editors}\index{tex@\TeX!editing}\index{editing tex@editing \TeX} you 
wish, but I'll focus on \ixx{GNU emacs}{GNU Emacs} in this chapter.
I've chosen GNU emacs for two reasons: it is a popular and very powerful
editor available on many platforms (\Unix, NeXT, MS-DOS, 
OS/2, Macintosh, VMS, Amiga, $\ldots$), and it has the
most comprehensive \TeX\ editing environment that I've ever seen.  At the
end of the chapter, I'll discuss several other editors for MS-DOS and OS/2
that also have \TeX\ editing environments.

The sole requirement for a \TeX\ editor is that it must save
files in a flat, \ixx{ASCII}{ASCII character set} format 
without any additional formatting characters
or special encodings.  Word processors, such as 
\product{\idx{Microsoft Word}}, usually 
add
special formatting information to your document when they save it to disk.
\TeX\ will not understand this information, so you cannot use a word
processor to edit \TeX\ documents.\footnote{Most word processors can be coerced into saving your
document in plain text.  If your word processor can do this, you may be able
to edit \TeX\ documents with it.}

A program that allows you to edit flat ASCII files is usually called
an editor (as opposed to a \ixx{word processor}{word processing}, for 
example).  Choosing
an editor is a remarkably personal decision.  Computer programmers
and other people who use editors every day frequently become very
attached to a particular editor.  This chapter explores a number of features that an
editor can provide to make editing \TeX\ documents easier.  If
possible, choose an editor that offers these features.  In any event,
make sure you choose an editor you will be comfortable
using.  You'll spend a lot of time using it.  

\pagebreak
Many editors have some sort of built-in programming language.  The features
described in this chapter rely on modifying the behavior of the editor with
its programming language.  This does not mean that {\em you\/} will have to
do any programming.  All of the editors discussed in this chapter have
\TeX\ editing environments already available.  However, if you
use another editor and it doesn't have a programming language, it's
unlikely that these features will be available to you.

The next section explores some helpful
editor features.  All of the editors in this
chapter provide some or all of the features discussed, and require
little or no programming on your part.

\section{What Can an Editor Do?}
\label{sec:whatcando}

You will do most of your work with \TeX\ in the 
editor\index{editors!programmable}.  There are two
ways that an editor can help you edit \TeX\ documents.  One is to
provide typing shortcuts that are either intuitive replacements for
cumbersome operations or quick ways of typing common \TeX\ commands.
The other way that an editor can help is by running \TeX\ for you and
automatically displaying the location of formatting errors in your
document.

\subsection{Quoting Automatically}
\index{`` and '' (quotation marks)!with editor}

Most typewriters and many word processors use the same symbol for
opening and closing quotations.  You are probably used to typing the
double-quote key to insert quotation marks into your text.  However,
if you look carefully at \TeX\ output (or any professionally typeset
document) you will notice that the opening and closing quotation marks
do not look the same.  The proper way to type an opening quote (``) in
\TeX\ is with two single back-quotes in a row (\verb|``|).  A closing 
quote ('') is entered with two single quotes (apostrophes) in a 
row (\verb|''|).  This process is
tedious and error-prone since you are used to typing something else.

If you accidentally use the double quote symbol in your input, you most
frequently get text that looks like ''this.''  The exact result is actually
dependent upon the font you are using.  This is explained in
Chapter~\ref{chap:fonts}, {\it \nameref{chap:fonts}}.
Most programmable editors can change the meaning of the double-quote
key to insert the correct quotation marks.

If you are interested in programming your editor to do this, you can use
the following
algorithm to select the correct quotation marks most of the time:  if the
character to the immediate left of the cursor is a space, opening
brace, parenthesis, or bracket, insert opening quotes; otherwise, insert
closing quotes.  As a further enhancement, double quotes should be inserted
immediately following a backslash.

\subsection{Matching Braces}
\index{braces!with editor}

Braces, as mentioned in Chapter~\ref{chap:tex}, {\it \nameref{chap:tex}}, are
used by \TeX\ to delimit sections of text; they appear often in \TeX\
documents.  \ixx{GNU emacs}{GNU Emacs}, \product{\idx{Multi-Edit}}, and 
\ixx{\program{Brief}}{Brief@Brief (program)} can all be programmed to
highlight the matching open brace whenever you type a closing brace.

For example, if you have entered the text

\begin{shortexample}
\footnote{This is {\it not\/} the only case.}
\end{shortexample}

and the next character that you type is \key{\ttclosebrace}\,, the open
brace immediately following the word \cs{footnote} is highlighted,
or all of the text between that brace and the current cursor position
is highlighted.
This feature makes it easier to find places where you have forgotten
to insert a closing brace.

If you usually work in ``insert'' mode, you may also find it convenient to
have the editor insert braces in pairs and then insert text between them.
This can be accomplished in almost any editor that can be programmed,
even one that isn't equipped to handle brace matching.

\subsection{Inserting Common Control Sequences}

Most editors can be programmed to insert arbitrary text when a special
key is pressed.  This feature can be used to insert common control
sequences\index{control sequences!inserting with editor}.  For 
example, you might have \key{Alt-c} insert the
\cs{chapter} control sequence or \key{Ctrl-e} insert
\verb|\begin{enumerate}|.

\subsection{Running \protect\TeX\ Automatically}

One of the nicest features that an editor can provide is the ability
to run \TeX\ automatically\index{running tex@running \TeX!automatically} without 
leaving the editor.  
This is a feature that programmers demand because it allows them to
run \idx{compilers} directly from the editor.  Editors that
provide this feature usually include some mechanism for programming
the editor to locate the position of errors reported by the compiler.
The editor searches for \idx{error messages} in the output generated by the
compiler, and it positions the cursor at the location of each error.  This
speeds up the traditional edit/compile/debug cycle of programming.
You can take advantage of these features to shorten the edit/typeset/rewrite
cycle of creating a \TeX\ document.  

All of the editors discussed in this chapter can easily be adapted to
run \TeX\ in this way.  If you already have a favorite editor and
want to add this functionality for \TeX, see the section
``\nameref{sec:texcomp}'' later in this chapter; it describes the process
at a very general level.

\section{GNU Emacs}

This section describes the \TeX\ modes distributed 
as part of \ixx{GNU emacs}{GNU Emacs}.
Emacs is one of the most popular and most powerful editors around.
Distributed by the \idx{Free Software Foundation (FSF)}, GNU 
emacs is the de facto
standard editor in many \Unix\ environments.  Recently, GNU emacs has been
ported to many other platforms, including MS-DOS (\ixx{\program{demacs}}{demacs}),
OS/2, Macintosh, VMS, Amiga, and NeXT.

\begin{sidebar}
This section and the following section on \auctex\ assume that you are
familiar with general emacs concepts.  In particular, you should be
familiar with the concepts of buffers, files, regions, command keys,
editing modes, and prefix arguments.  If you are unfamiliar with these
concepts, you can learn about them in the online help, called Info
pages, for GNU emacs.  Info pages should be available by pressing
\key{Ctrl-H} \key{i} in emacs.  You can also consult a reference to
GNU emacs, such as {\it Learning GNU Emacs}~\cite{or:emacs} for more
information.
\end{sidebar}

Customizing some of the features of GNU emacs requires familiarity
with \ixx{GNU emacs lisp}{GNU Emacs!Lisp}, which is also described in the 
Info pages.

\subsection{Starting \TeX\ Mode}

\ixx{GNU emacs}{GNU Emacs!tex mode@\TeX\ mode} provides 
two similar \TeX\ editing modes: one designed for
editing Plain \TeX\ documents (\textit{plain-tex-mode}) and the other for
editing \LaTeX\ documents (\textit{latex-mode}).  The
\textit{latex-mode} is a superset of \textit{plain-tex-mode} that
provides additional shortcut keys for some \LaTeX\ control sequences.

There are three ways to start \TeX\ mode in GNU emacs:

{\LTleft=.25in%
\begin{xtable}{ll}
  \cmd{M-x tex-mode}       & Attempts to select the correct mode. \\
  \cmd{M-x plain-tex-mode} & Always selects Plain \TeX\ mode. \\
  \cmd{M-x latex-mode}     & Always selects \LaTeX\ mode. \\
\end{xtable}
}

If you use \cmd{M-x tex-mode}, emacs examines the top of the buffer in
order to select the appropriate mode.  If the control sequences 
\cs{documentstyle} or
\verb|\begin{document}| occur near the top of the buffer, 
\emph{latex-mode} is selected; otherwise, the default mode is selected.  The
default mode is stored in the emacs lisp variable
\verb|TeX-default-mode|.

\pagebreak
You can also tell emacs to invoke \TeX\ 
mode automatically whenever
you edit a file that has a name ending 
in \filename{.tex}\index{tex files}.  To do so,
add the following lines to your 
emacs startup file\index{emacs startup files@.emacs startup files}, 
usually called
\filename{.emacs} in your home directory:\footnote{On file systems that don't allow filenames
to begin with a period, the name frequently begins with an underscore
instead.} 

\begin{shortexample}
(setq auto-mode-alist (append '(("\\.tex$" . tex-mode))
                               auto-mode-alist))
\end{shortexample}

\subsection{Typing in Emacs}

The emacs \TeX\ modes change the meaning of several keys to provide
features useful for 
editing \TeX\ documents.  These 
special\index{GNU Emacs!changing key meanings}\index{GNU Emacs!matching braces} key 
bindings apply only to buffers that you edit while emacs
is in \textit{plain-tex-mode} or \textit{latex-mode}.

Automatic quotation, brace balancing in paragraphs, inserting brace
pairs, skipping over unmatched braces, and closing open environments
are supported.

\subsection{Running \TeX\ Automatically}

When \TeX\ typesets your document, it produces processing and error 
messages.  By\index{running tex@running \TeX!automatically}\index{GNU Emacs!running tex automatically@running \TeX\ automatically} running 
\TeX\ for you, emacs can
capture these messages and display them in a window.  You can use this
feature to help locate and correct errors.

Functions that run \TeX\ from inside the editor rely on emacs'
ability to run a \idx{subshell}.  Some
implementations of emacs, particularly implementations for MS-DOS,
which is unable to run concurrent processes, cannot use this
feature.  \auctex, a different editing environment for GNU emacs,
does allow you to use these features with MS-DOS 
ports\index{GNU Emacs!MS-DOS} of GNU
emacs.  (\auctex\ is described in the next section.)

GNU emacs supports processing of both buffers and regions.  These
modes don't locate errors for you automatically, but they do place
the output from \TeX\ in an emacs buffer so that you can find them
yourself.  Chapter~\ref{chap:running}, {\it \nameref{chap:running}}, describes how to 
interpret \TeX\ output and find the location of errors.

\section{\protect\auctex}
\label{sec:auctex}

This section provides an overview of \ixx{\auctex}{auctex@\auctex},\index{GNU Emacs!auctex package@\auctex\ package} a powerful emacs macro package
for editing \LaTeX\ documents.
\auctex\ is available on the
CTAN archives in \ctandir{support/auctex}.
More detailed information about installing and customizing \auctex\
can be found in the documentation distributed with the package.

\auctex\ provides extensive support for editing \TeX\ and \LaTeX\ documents
in emacs.  It provides many more features than the ordinary GNU emacs
\TeX\ modes.  Although useful for both Plain \TeX\ and \LaTeX\ documents,
\auctex\ is designed with the \LaTeX\ user in mind.

\auctex\ is {\em a lot\/} more complex than GNU emacs \TeX\
mode.  In fact, it is so complex that it may not be useful if you have
a relatively slow computer.\footnote{That's one of the reasons that the
regular GNU emacs \TeX\ modes are described first.}  On my
machine,\footnote{A 16MHz 386SX machine with only 8Mb of memory
struggling to run GNU emacs under OS/2. ;-)} \auctex's performance 
leaves a lot to be desired.  On the other hand, \auctex\ is extensively
configurable, and it is possible to streamline it quite a bit.

The descriptions that follow are for \auctex\ version 8.0.
The versions
change frequently as new features are added. Consult the documentation
which comes with \auctex\ for a list of the new features that have been
added since this book was published.

%This section assumes that you are familiar with general emacs concepts.
%In particular, you should be familiar with the concept of buffers, files,
%regions, command keys, editing modes, and prefix arguments.  If you are
%unfamiliar with these concepts, they are explained in the online help,
%called Info pages, for GNU emacs.  Info pages should be available by pressing
%\key{Ctrl-H} \key{i} in emacs.  You can also consult a reference to GNU
%emacs, such as {\it Learning GNU Emacs}~\cite{or:emacs} for more information.

\subsection{Starting \auctex}

The instructions provided with each release of \auctex\ describe how to build
and install the software so that \auctex\ will 
automatically be invoked\index{auctex@\auctex!starting} when 
you edit a file with a name ending in \filename{.tex}.

If you think \auctex\ is already installed on your system (because the
emacs Info page for it is present, for example), ask your system administrator
where it is installed.  One common location is 
\filename{/usr/local/lib/emacs/site-lisp/auctex}.

Adding the following line 
to your \ixx{emacs startup file}{emacs startup files@.emacs startup files} (typically 
\filename{.emacs} in your home directory) will load \auctex\ each time
you start emacs:

\begin{exindent}
\begin{alltt}
  (load-file "\textit{/path/for/auctex}/tex-site.elc")
\end{alltt}
\end{exindent}

\subsection{Typing in \auctex}

\auctex\ provides a large number of typing shortcuts\index{auctex@\auctex!keystroke shortcuts}.  Many of the
keystroke shortcuts are designed specifically to aid in typing
\LaTeX\ documents.
In addition to command keys, \auctex\ provides another typing
shortcut---command completion\index{auctex@\auctex!command completion} for 
\LaTeX\ control sequences with
\key{C-c} \key{TAB}.

If you type a backslash followed by the beginning of a control
sequence name into a buffer and then type \key{C-c} \key{TAB}, \auctex\
completes as much of the control sequence as possible.  Every time
\auctex\ adds another letter to the control sequence name, it compares
the resulting name to a configurable list of \LaTeX\ control sequences.
If it reaches a point where the control sequence could be continued in
two different ways, it stops and displays all the possible
completions.
For example, there are two control sequences in standard \LaTeX\ that
begin with \cs{re}: \cs{renewcommand} and \cs{renewenvironment}.

\pagebreak
If you type:

\begin{exindent}
  \cs{re} \key{C-c} \key{TAB}
\end{exindent}

\auctex\ will insert \verb|new| because that much of the control sequence
name can be deduced from the known possibilities.  This changes the text
in your document to:

\begin{shortexample}
\renew
\end{shortexample}

Because \auctex\ cannot figure out which of the possibilities you want,
it displays a list of the \LaTeX\ commands that begin with
\cs{renew}.  You can complete the command by typing
\textbf{c} or \textbf{e} and pressing \key{C-c} \key{TAB} again.
If you want the \cs{renewcommand} function, type:

\begin{exindent}
  \cs{renewc}\key{C-c} \key{TAB}
\end{exindent}

Now, \auctex\ will insert \verb|ommand| and return to normal 
typing mode with the cursor positioned just after the control sequence
name:

\begin{shortexample}
\renewcommand _
\end{shortexample}

\auctex\ has special support for \LaTeX\ sectioning commands and
environments, changing fonts, commenting out sections of a document,
reformatting the input text, and entering mathematics.

\TeX\ contains a lot of support 
for \ixx{typesetting mathematics}{typesetting!mathematics!with auctex@with \auctex}.  In 
that spirit,
\auctex\ provides a\index{mathematics!minor mode} minor mode\footnote{A minor mode is a kind of editing
environment provided by GNU emacs.  If you are unfamiliar with minor modes,
consult your emacs reference.} for entering mathematical formulae.

In mathematics minor mode, pressing \key{`} changes the meaning of the
next character you type.  The next character is interpreted as an
abbreviation for a mathematical symbol or function.
For example, in \auctex, typing

\begin{exindent}
  \key{\ttdollar} \key{C-c} \key{\tttilde} \key{`} \key{a} \key{`}
  \key{<} \key{`} \key{b} \key{\ttdollar}
\end{exindent}

inserts

\begin{shortexample}
$\alpha \leq \beta$
\end{shortexample}

into your document (which is typeset like this: $\alpha\leq\beta$).

Table~\ref{tbl:mathops} lists all of the abbreviations.
The first column of the table displays the shortcut keys.
The second and third columns show the command inserted in your
document and the typeset symbol, respectively%
\index{mathematics!auctex operators@\auctex\ operators}\index{auctex@\auctex!mathematics operators}.

\pagebreak
\input{aucmath.tab}

Additional miscellaneous commands in \auctex\ provide automatic 
quotation, completion of ``items'' in appropriate environments (\verb|itemize|
and \verb|enumeration| environments, for example),
and insertion of brace pairs\index{auctex@\auctex!matching braces} and 
skeletal control sequences.

\subsection{\protect\auctex\ Outline Mode}

\ixx{Outline mode}{auctex@\auctex!outline mode} is a 
convenient way to edit large documents.  In outline
mode, portions of the document that you are not editing are hidden
from view.  They aren't removed or deleted.  Emacs indicates hidden text with ellipses.

For example, in a large document with many sections, you can use
outline mode to hide all text except the section headings, and then
selectively expand just the sections that you wish to edit.

\subsection{Intelligent Paragraph Reformatting}

\auctex\ understands the \TeX\ constructions for many kinds of environments
(the list environments, for example) and 
performs \ixx{paragraph reformatting}{auctex@\auctex!reformatting paragraphs} within 
the restrictions of these environments.  \auctex\ won't concatenate a 
whole series of list items together into one huge paragraph, for example.

\subsection{Multi-file Documents}
\label{sec:multifile}

It is often convenient to edit a large document in small 
pieces\index{auctex@\auctex!handling multi-file documents} rather
than in one huge file.  For example, when writing a book, it is convenient
to work on chapters independently and store them in different files.  If
you are working on a collaborative project, it may be absolutely necessary
to separate the document into pieces.

The \TeX\ \cs{input} command allows you to construct a driver file
that automatically combines the individual files that make up your
document when you run \TeX.  The driver file contains the document
style options and other setup information for the whole document.
Each chapter contains just the necessary text.  When \TeX\ encounters
an \cs{input} command, it typesets all of the
text in the specified file before continuing with the current document.
Example~\ref{ex:driver} shows an example of a driver file.

\begin{example}{ex:driver}{A Simple Driver File}
\documentstyle[ora]{book}
\begin{document}
   \input{intro}
   \input{chap1}
   \input{chap2}
\end{document}
\end{example}

\auctex\ provides seamless support for multi-file documents.
In order to provide this support, \auctex\ relies on the ``file variables''
feature of emacs, which allows you to associate editor variables
with particular buffers.
An \ixx{editor variable}{editors!variables for}\index{variables!editor} is 
a named variable that
is local to the current buffer and accessible by macro packages like
\auctex\ running under emacs.
Look up file variables in your emacs reference for a more complete
description.  Example~\ref{ex:localvars} shows some common local
variables in \auctex.

\pagebreak
When you run \TeX\ on a buffer, \auctex\ looks for the editor variable
\verb|TeX-master|.  If \verb|TeX-master| is set to a filename, 
\auctex\ runs \TeX\ on that file instead of running it directly on the file 
you are editing.
This is a tremendously useful feature because it means that you do not
have to put macro definitions, document style options, and other setup
information at the top of each chapter.  Simply set the \verb|TeX-master|
variable in each chapter to name the driver file.

If you do not set the \verb|TeX-master| variable, \auctex\ will prompt
you for it the first time you run a command. If the
\cs{documentstyle} command occurs near the top of your document,
\auctex\ assumes that the current buffer \textit{is} the master
(because it contains setup information) and will not
prompt you for a different master file.

Editor variables are defined by a ``Local Variables'' declaration at
the bottom of your file.  \auctex\ inserts a Local Variables
declaration automatically if it prompts you for a master file.
Example~\ref{ex:localvars} shows how local variables
\ixx{\texttt{TeX-master}}{auctex@\auctex!TeX-master@\protect\texttt{TeX-master}}
and 
\ixx{\texttt{TeX-command-default}}{auctex@\auctex!TeX-command-default@\protect\texttt{TeX-command-default}}
can be set to
\filename{driver.tex} and \verb|LaTeX|, respectively.  Because
these lines begin with a percent sign, they are considered comments by
\TeX\ and do not appear in the output.  These should be 
the last lines in the file.

\begin{example}{ex:localvars}{Local Variables in an Emacs Buffer}
% Local Variables:
% TeX-master: "driver.tex"
% TeX-command-default: "LaTeX"
% End:
\end{example}

\subsection{Running \TeX}

Like GNU emacs \TeX\ mode, \auctex\ allows you to 
run \TeX\index{auctex@\auctex!running tex automatically@running \TeX\ automatically} directly
from within emacs as a subshell.  Running a program inside emacs
creates a process\index{process (in auctex)@process (in \auctex)}.  You 
can have only one active process for
each document, plus one process for \TeX{}ing a \idx{region}.  If you 
try to
run two processes on the same document, \auctex\ will ask for
permission to kill the first before running the second.  \auctex\ supports
the processing of both the documents and the regions of a document.

\subsection{Finding Errors}

No matter how much experience you have with \TeX, some of the
documents that you write will contain 
errors\index{auctex@\auctex!finding errors}.  \auctex\ eases the
burden of correcting these errors by locating them automatically
in your document.  
\newpage

\section{Multi-Edit}

\program{\idx{Multi-Edit}}\index{MS-DOS!editors} is an 
editor for the MS-DOS environment.  This section describes
the text-based version of \program{Multi-Edit}.  A Windows version is in the works, and
it may exist by the time you read this.

Built to be a programmer's editor, \program{Multi-Edit} has a number of features
designed to add language-specific intelligence to the editing environment.
These features (template editing, a customizable spellchecker, and
configurable brace matching) can be exploited for \TeX\ as readily as for
any programming language.  For several years, at least since version 5.0,
\program{Multi-Edit} has included support for \TeX.

\subsection{Setting Up \protect\TeX\ Support}

Support for \TeX\index{Multi-Edit!setting up tex support@setting up \TeX\ support} is 
included as a language type in \program{Multi-Edit}.
Language types are selected by filename extension; this means that you
configure \program{Multi-Edit} to provide support for the ``\TeX\ language'' whenever
you edit files with the 
extension \filename{.tex}\index{tex files} (and 
other extensions if you choose).

Selecting the ``Other/Install/Filename extensions...'' menu displays
a list of configured extensions.  If TEX is not listed,
press \key{Insert} to add it.  You will see a screen like the one shown
in Figure~\ref{fig:meextsetup}.

\epsimage{fig:meextsetup}{Extension setup in Multi-Edit}

\pagebreak
Customize the right margin, tab spacing, indent style, and 
colors to values that you find comfortable.  The edit mode should 
be ``text,'' and the tab settings should be set to ``use tab and margin 
settings, ignore format line.''

Select TEX as the language type, and add \TeX\ as a
compiler.  One possible setting for \TeX\ as a compiler (for \LaTeX\
documents, in this case) is shown in Figure~\ref{fig:melatexcmp}.

\epsimage{fig:melatexcmp}{\protect\LaTeX\ as a compiler in Multi-Edit}

\subsection{Typing in Multi-Edit}

\ixx{Brace matching}{Multi-Edit!matching braces}  is 
provided automatically with Multi-Edit's
template expansion support.  Consult your Multi-Edit reference for
more information about templates.

Multi-Edit language support does not include any \TeX\ key bindings by
default.  However, the macros are provided, and you can install them with
the ``Other/Install/Key mapping...'' menu.
The following macros are available:

\begin{itemize}
  \item \verb|tex^texquote| inserts the appropriate quotation marks.
        This macro can be bound to \key{"} to provide smart 
        quoting\index{Multi-Edit!quotation marks}.
  \item \verb|tex^texnquote| inserts the literal double quote.
        It can be bound to \key{Alt-"}\,, for example.
  \item \verb|tex^texreformat| is a replacement for the reformat macro.
        If the filename extension is \filename{.tex}, this macro reformats
        the paragraph with sensitivity to \TeX\ macros.  Otherwise it calls
        the default reformatting macro.
\end{itemize}

\subsection{Running \TeX}

Running \TeX\ within Multi-Edit is accomplished by specifying a compiler for
\TeX\ or \LaTeX\ documents\index{Multi-Edit!running tex automatically@running
\TeX\ automatically}.  In Figure~\ref{fig:melatexcmp}, a batch file called
\filename{TEXIT} is being used as the compiler for documents with the
extension \filename{.tex}.

Automatic compilation and location of errors is provided by Multi-Edit
language support when you have selected \verb|TEX| as the language-type
for \filename{.tex} files.

\section{Brief}

\ixx{\program{Brief}}{Brief@Brief (program)}\index{MS-DOS!editors}\index{OS/2!editors} is 
a powerful programmer's editor recently acquired by
\idx{Borland International}.  It is available for both DOS and OS/2 systems.
Like the other editors discussed in this section, \program{Brief} offers
a strong set of programming features, including the ability to run 
compilers automatically, and a flexible, C-like macro programming language
that allows you to customize the editor.

There is a \LaTeX\ editing environment for \program{Brief} in the CTAN archives.
It includes multiple-language support (currently
supporting Norwegian) and control-key shortcuts for many common \LaTeX\
commands.

The installation and setup program includes instructions
for defining program compilers.  \TeX\ can be defined to process 
files with particular extensions.  Once again, it is a good idea to have
the editor run a batch file so that the batch file can determine what
format file to use when processing the document.

\section{MicroEMACS}

\ixx{\program{MicroEMACS}}{MicroEMACS}\index{MS-DOS!editors}\index{Microsoft Windows!editors} is a powerful emacs-like editor for MS-DOS and 
Microsoft Windows.\footnote{There are two versions of the program; the one
discussed here is the version for Windows.}
Figure~\ref{fig:editing:mew} shows an example of \program{MicroEMACS} editing
a \LaTeX\ document.  In this case, the
\LaTeX\ extensions to \program{MicroEMACS} have been loaded, and ``Help''
is selected.

\epsimage{fig:editing:mew}{Editing a file with \protect\program{MicroEMACS}}
 
The online help for \LaTeX\ is shown in Figure~\ref{fig:winhelp}.
This is a Microsoft Windows help file, available independently of
\program{MicroEMACS} (although it is nicely integrated here).

\epsimage{fig:winhelp}{Microsoft Windows online help for \protect\LaTeX}

The \program{MicroEMACS} environment also has the ability to run other
programs (including \TeX\ to process documents) and process the error 
output to aid in locating errors.

\section{epm: OS/2's Enhanced Editor}

\program{epm}\index{epm}\index{OS/2!editors}, the 
enhanced editor for OS/2, can be used to edit
\TeX\ documents.  The 
\ixx{\program{epmtex}}{epmtex} package adds a \TeX\ item to
the menu bar as shown in Figure~\ref{fig:editing:epm}.

\epsimage{fig:editing:epm}{Editing a \protect\TeX\ document with epm 
  under OS/2}

\section{Other Tools}

There are many other editors that can be effective tools for editing
\TeX\ documents.  Some of the editors that you might want to consider
are idx{\program{Jove}}, an emacs-like editor; \ixx{\program{Xnot}}{Xnot}, 
a Windows port
of emacs; \ixx{\program{LSedit}}{LSedit}, the VMS language-sensitive editor; 
and \ixx{\program{Alpha}}{Alpha} and \ixx{\program{BBedit}}{BBEdit}, two Macintosh editors described
in Chapter~\ref{chap:mac}, \textit{\nameref{chap:mac}}.

There are also a number of tools designed specifically for editing 
\TeX\ documents.
\ixx{\program{Scientific Word}}{Scientific Word@Scientific Word (program)}, a commercial environment, is described in
Chapter~\ref{chap:commercialtex}, \textit{\nameref{chap:commercialtex}}.
Several free tools (\ixx{\program{MathPad}}{MathPad},
\program{Doc}, and \ixx{\program{X\TeX{}Shell}}{Xtex{Shell}@X\TeX{}Shell}, for example) also exist,
although none
were available\footnote{Or known to me, anyway.} in time for review
in this edition of \textit{\booktitle}.

Another helpful tool is \ixx{\program{LaCheck}}{LaCheck}, a \LaTeX\ syntax checker.
\program{LaCheck} attempts to find and identify problems with your document
that will cause it to format incorrectly.  It runs much more quickly than
\TeX\ over a long document and may identify some things (like 
\newpage
missing
italic correction) that are potentially incorrect, even though they are
not errors that will prevent the document from formatting.

\section{\protect\TeX\ as a Compiler}
\label{sec:texcomp}

This section discusses how to use \TeX\ like 
a compiler\index{editors!using tex as a compiler@using \TeX\ as a compiler} from within
your editor, and it is rather technical.  You should read this section
before you attempt to program your own editor to run \TeX\ (if it
doesn't already include support for \TeX).  You may
not be interested in this material if you aren't planning to do that
programming yourself.
More details on the types of \TeX\ output used in this section can be
found in Chapter~\ref{chap:running}, {\it \nameref{chap:running}}.

Many editors can run a compiler, capture the error messages that the
compiler produces, and walk through the source file highlighting each
error.  Most editors with this functionality can run \TeX\ as a
``compiler'' for documents.  

The discussion that follows is at a very abstract level.  The details vary
so much from one editor to the next that presenting more detail only
confuses the issue further.  To implement this feature, you'll need to
read your editor reference carefully, and probably experiment on your
own a little bit.

\subsection{Processing a Document}

Establish the \TeX\ program as a compiler using whatever features your
editor provides to set up a compiler.  Experiment with your editor until
you can process an error-free document.

After you have everything working, it may be desirable to modify the
compiler to run a batch file or shell script instead of invoking the
\TeX\ program directly.  This will allow you to program the shell script
so that it can determine which format file to use.

Running \TeX\ is only the first step.  Next, you have to interpret the
\ixx{error messages}{error messages!tex@\TeX}  that \TeX\ produces.  
As a concrete example,
Example~\ref{ex:madeuperr} shows some contrived output which demonstrates
an ``undefined control sequence'' error on line 6 of the file
\filename{wilma.tex}.

\begin{example}{ex:madeuperr}{A Contrived Error}
This is TeX version 3.141...
** (flintstone.tex
some
messages [1]
(fred.tex [2]
some more
messages
[3] ) (wilma.tex [4]
! Undefined control sequence.
the dog \dino
              tipped over the car.
l.6 the dog \dino\ tipped over the car.
\end{example}

Here are some suggestions for attacking the problem of programming
your favorite editor to process \TeX\ error 
messages\index{error messages!programming an editor to process}:

\begin{iplist}{.25in}
\ipitem[Using the log file]

Don't worry about capturing the error messages that \TeX\ produces.
It is much easier to get the 
information from the \ixx{log file}{log files}.
Log files are described in the section called ``\nameref{sec:logfiles}''
in Chapter~\ref{chap:running}, {\it \nameref{chap:running}}.

\ipitem[Finding error messages]

Any line in the log file that begins with an exclamation point is an
error message.  In Example~\ref{ex:madeuperr}, the line that begins

\begin{shortexample}
! Undefined control sequence.
\end{shortexample}

is an example of an error message.

\ipitem[Finding the source line]

Following the error message, \TeX\ shows the context in which the error
occurred.  After that, the line that begins with \texttt{l.\textit{nnn}} (where
\texttt{\textit{nnn}} is some decimal number) identifies the line of the input
file that \TeX\ was processing when the error occurred (in
Example~\ref{ex:madeuperr}, \TeX\ was on line 6 when the error occurred).

\ipitem[Finding the source file]

When processing a document that uses the \cs{input} command to
include other files, there is no
guaranteed method of finding out the name of the file \TeX\ was
processing when the error occurred.
Whenever \TeX\ starts processing a file, it prints an open parenthesis
followed by the name of the file.  When it finishes processing the
document, it prints a close parenthesis.\footnote{Unfortunately,
parentheses can occur in the log file for other reasons.}  So
the following algorithm {\em usually\/} identifies what file the error
occurred in:

Beginning at the line in the log file that announces the error message
(the line beginning with ``!''), search backwards for the first
unmatched open parenthesis.  The word following that open parenthesis
is probably the name of the file \TeX\ was processing when the
error occurred.

\pagebreak
\ipitem[Ignoring errors]

When you run \TeX\ ``by hand,'' you want \TeX\ to stop and report
errors to you as they occur in your document.  But if your editor is
going to handle any errors that occur, it is inconvenient to have
\TeX\ stop and ask questions.  In fact, it may not be possible to run
\TeX\ from your editor in a way that makes it even {\em feasible\/}
for \TeX\ to stop and ask questions.  You can use several built-in
control sequences to control the way \TeX\ responds to errors.  They
are summarized in Table~\ref{tab:modes}.

{\LTleft=.25in%
\begin{xtable}{l|l}
  \caption{\protect\TeX\ Modes of Interaction
    \label{tab:modes}}\\
  \bf Mode  & \bf \TeX's Behavior\\[2pt]
  \hline
  \tstrut 
  \cs{errorstopmode} & Stop on errors (the default behavior)\\
  \cs{scrollmode}    & Scroll errors, stop on missing files \\
  \cs{nonstopmode}   & Scroll errors and missing files \\
  \cs{batchmode}     & Scroll both and don't summarize on screen \\[2pt]
  \hline
\end{xtable}%
}

One common way of invoking \TeX\ from an editor to process a document
is to use the command:

\begin{shortexample}
tex \nonstopmode \input flintstone
\end{shortexample}

This uses the features discussed in the section
``\nameref{sec:clineopts}'' in Chapter~\ref{chap:running}
to pass a command to \TeX\ on
the command line.  In this case, the command tells \TeX\ not to stop
on any kind of error.

\ipitem[Handling other errors]

When searching for errors in the log file, you may find it helpful to
search for lines that begin with the 
words ``Overfull box''\index{boxes!overfull} or
``Underfull box''\index{underfull} as well 
as lines that begin with an exclamation
point.  Because \TeX\ does not print the \texttt{l.nnn} form of line-number
message in this case, you will have to look for the line
numbers in the warning message.

Overfull box messages have the form:

\begin{exindent}
\begin{alltt}
Overfull \textit{box} (\textit{99.9}\,pt too wide) in paragraph at lines \textit{n--m}
\end{alltt}
\end{exindent}

The \texttt{\textit{box}}\, will be either \verb|hbox|, indicating that something
is too wide, or \verb|vbox| indicating that something is too tall
or too deep.  The distance, \texttt{\textit{99.9}\,pt}, indicates how 
badly the box is overfull, and \texttt{\textit{n}} is 
the first line of the paragraph in which the error
occurs.  Underfull box messages are the same, except that they begin
with the word ``Underfull.''

There are several control sequences that you can use to control how sensitive
\TeX\ is to ``bad'' boxes.  Any good \TeX\ reference will discuss these
parameters in detail.

\end{iplist}

\section{Spellchecking}

Checking for spelling mistakes\index{spellchecking with editor} is an 
important part of any document
creation process.  Checking \TeX\ documents is difficult because these
documents contain control sequences that aren't words in the
traditional sense.

Some editors offer ways of customizing the spellchecker.  Multi-Edit, for 
example, allows you to indicate that any word beginning with a backslash 
should be ignored for the purpose of spellchecking.  
Figure~\ref{fig:mespell} shows the 
``Other/Install/Edit Settings...'' 
dialog where word delimiters are controlled.  This simple 
customization goes a long way towards making spellchecking tolerable.

External spellcheckers, such as the ones described here, can also 
frequently be customized to ignore \TeX\ control sequences.  

\epsimage{fig:mespell}{Edit settings control word delimiters}

\subsection{ispell}

\program{ispell}\index{ispell} is 
a common \Unix\ spellchecker.  In addition to
being available from the shell prompt, GNU emacs 
includes an \textit{ispell-mode} that handles \TeX\ documents intelligently.

\subsection{amSpell}

\program{\idx{amSpell}} is an 
MS-DOS spellchecker that includes special
support for \TeX\ documents.  When spellchecking a \TeX\ document,
\program{amSpell} ignores all \TeX\ control sequences, as well as
mathematics and the arguments to reference and citation commands.  The
standard accent primitives are also recognized, and \program{amSpell}
can identify and correct misspellings in words that use them.

An example of \program{amSpell} checking a document is shown in
Figure~\ref{fig:amspell}.

\epsimage{fig:amspell}{Spellchecking a document with amSpell}

\section{Revision Control}

\ixx{Revision control}{revision control} allows you 
to track modifications to a file.  
It is frequently associated with programming where the ability to 
find and correct bugs relies on being able to recreate a
problem exactly.  It can be just as useful for writers wishing to keep
track of changes to an evolving document.  For example, I use
revision control to keep track of which versions of each chapter
my editor has seen.

One of the most common revision control systems is \idx{RCS}, which was
derived from \idx{SCCS}, a commercial package.  RCS is freely available under
the GNU license.  It is possible to use RCS under \Unix, MS-DOS, and
OS/2.

The \TeX\ macros shown in Example~\ref{ex:rcsmac} can be used to
include RCS information as marginal notes in a document.  The
marginal notes in this example are printed only when a draft
is being produced.

These macros are my own, you will find others in the CTAN archives
in the directory \filename{macros/latex/contrib/misc}.

\begin{example}{ex:rcsmac}{Revision Control Macros for \protect\TeX\ 
    Documents Using RCS}
%%%%%%%%%%%%%%%%%%%%%%%%%%%%%%%%%%%%%%%%
% RCS definitions...
%
\newif\ifdraft
\def\RCSID$#1${%
  \ifdraft{\tolerance=100000%
           \hbadness=100000%
           \raggedright%
           \marginpar{\tiny Draft #1}}%
           \typeout{Draft #1}%
  \else\typeout{Production run #1}\fi%
}

\def\RCSmargid$#1: #2 #3 #4 #5 #6 #7${%
% #1 = ``Id''
% #2 = filename
% #3 = vers
% #4 = date
% #5 = time
% #6 = author
% #7 = state [locker]
  \ifdraft
    \setbox0=\hbox to 0pt{%
        \tolerance=100000%
        \hbadness=100000%
        \parbox{4in}{%
          \rm\tiny #2\\ #3\\ #4}%
        \hss}%
    \marginpar{\box0}%
    \typeout{Draft Id: #2 #3 #4 #5 #6 #7}%
  \fi
}
\end{example}

\section{\protect\TeX\ Shells}

The iterative nature of \TeX\ (edit, \TeX, \BibTeX, make indexes, preview,
repeat) can be tedious to perform by hand.  As a result, several \TeX\
``shells''\index{shell!tex@\TeX} have been 
developed which provide a more automatic interface
to many aspects of the \TeX\ process.  Several of these packages are
described below.  In everyday use, they make \TeX\ much more user
friendly.

\subsection{\protect\TeXShell}

\ixx{\TeXShell}{texShell@\TeXShell} was designed as an add-on for \ixx{\emTeX}{emTeX@\emTeX} (it 
installs directly
into the \emTeX\ directory hierarchy).  This character-based interface
is available for both MS-DOS and OS/2. \TeXShell\ provides an editor, 
a complete help system, and fully
customizable push-button access to \TeX. An
example of the \TeXShell\ interface is shown in
Figure~\ref{fig:texshell}.  The default \TeX\ menu and help windows
are shown.

\epsimage{fig:texshell}{\protect\TeXShell}

\subsection{\protect\TeXPert}

\ixx{\TeXPert}{texPert@\TeXPert} is a character-based 
shell for \TeX.  MS-DOS and OS/2 versions of
\TeXPert\ are available in either German or English.  The default 
configuration files for \TeXPert\ are designed to work with \emTeX, but 
\TeXPert\ does not install directly into the \emTeX\ directory heirarchy.
The \TeXPert\ interface is shown in Figure~\ref{fig:texpert}.

\epsimage{fig:texpert}{\protect\TeXPert}

\TeXPert\ provides an editor, an archive tool for speedy access to commonly
used files, and an interface to \program{grep} for quickly scanning files.
The exact look and feel of \TeXPert\ can be customized.

\subsection{4\protect\TeX}

The \ixx{\fourTeX}{4TeX@\fourTeX} system uses the
\ixx{\program{4DOS}}{4DOS} extended batch language and a number of utilities
to integrate \ixx{\emTeX}{emTeX@\emTeX} with a wide variety of free,
shareware, and commercial tools.  The system is well documented in a manual
that describes installation, setup, use, and customization.
The \fourTeX\ interface is shown in Figure~\ref{fig:fourtex}.

\epsimage{fig:fourtex}{\protect\fourTeX}

\subsection{\protect\PMTeX}

\ixx{\PMTeX}{PMTeX@\PMTeX} is an OS/2 presentation manager program.  Although it does not 
provide a built-in editor, it can control four independent DOS and OS/2
sessions, one each for editing, \TeX\ processing, previewing, and 
printing.  The options passed to each program can be controlled from
the \PMTeX\ Options menu.

\PMTeX\ was designed to work with \emTeX, but can be configured to use
any \TeX\ system.  \PMTeX\ was also designed to support two preprocessors
for phonetic transcription: 
the \ixx{Teu\TeX-P}{Teutex-P@Teu\TeX-P} and \ixx{ALD\TeX-P}{ALDtex-P@ALD\TeX-P} scanners.  These
scanners are useful in dialectology and are available separately from
\PMTeX's author.

A example of the \PMTeX\ interface is shown in Figure~\ref{fig:pmtex}.
Note that the menu bar is very wide and does not fit in a standard 640x480
VGA window.

\epsimage{fig:pmtex}{\protect\PMTeX}

\subsection{\protect\TeX{}it}

The \ixx{\TeX{}it}{tex{}it@\TeX{}it} shell (which I wrote) is much less ambitious in many ways.
Written entirely in Perl, \TeX{}it does not offer a full screen interface
or an editor at all.  Instead, it provides a simple menu of choices as
shown in Figure~\ref{fig:texit} (all of the menu choices are completely 
customizable).

\begin{figure}
\begin{FramedVerbSmall}
Processing: ./driver.tex
Again? [?]status, [T]eX & View, [b]ibtex, [c]leanup, 
       [e]dit a file, [p]rint, [q]uery printer, [t]ex, 
       [v]iew, e[x]it: 
\end{FramedVerbSmall}
\caption{\TeX{}it}
\label{fig:texit}
\end{figure}

One of \TeX{}it's strengths is its ability to parse the log file created
by \TeX\ and to determine when additional actions are required.  For example,
\TeX{}it will recognize when a document contains unresolved references
and citations and can automatically run \BibTeX\ and repeatedly run
\TeX\ to resolve the references.  With the addition of a few ``user specified''
rules in the document, \TeX{}it can easily handle index construction and
other more sophisticated relationships.

