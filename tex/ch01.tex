\chapter[The Big Picture]{The Big Picture}
\RCSID$Id: ch01.tex,v 1.1 2002/08/23 14:58:45 nwalsh Exp $
\label{chap:overview}
\label{chap:tex}

\ifincludechapter\else\endinput\fi

\section{What Is \protect\TeX?}

\ixx{\TeX}{tex@\TeX!introduction}\index{typesetting!tex@\TeX} is 
a \emph{typesetting system}.  It is a collection of
programs, files, and procedures for producing professional quality
documents with minimum effort.

\TeX's job is to translate the text you type into a 
\ixx{beautiful}{beauty of text}
typeset page.  The key word here is ``beautiful,'' and it is a very
lofty goal.\ff{Before I proceed, the notion of beautiful in this context
needs some explanation.  Several people have pointed out that the logo
type used by many \TeX-related programs (including \TeX\ itself) is
intrinsically ugly.  These same folks argue that a sentence like
``\TeX\ is designed to typeset beautiful pages'' is self-contradictory
because it begins with such an ugly construction.  Obviously, \TeX\
can't {\em prevent\/} you from typesetting ugly things. But \TeX\ can
typeset beautiful things too.  We at O'Reilly \& Associates think that
this book, typeset completely in \TeX, is an excellent example.}
What I mean by beautiful is that \TeX, when presented with several
paragraphs of plain text and left to its own devices, produces a remarkably
aesthetic page.  Despite the fact that \TeX\ may have to contend with
\ixx{multiple fonts}{fonts!multiple} and \idx{mathematics}, it still
manages to typeset pages in which each of the following aesthetic principles
hold {\em simultaneously:}

\begin{itemize}
  \item The right margin is \ixx{justified}{justification}.
  \item Proper justification is achieved without \idx{letterspacing}.
  \item Interword spacing is neither too tight nor too 
        loose.\index{spacing between words}
  \item The page is evenly gray.
  \item The \idx{baselines}\index{fonts!baselines} of multiple fonts 
        are properly aligned.
  \item \ixx{Hyphenation}{hyphenation} is automatic, if required, and usually correct.
  \item \ixx{Ladders}{ladders} are avoided.
\end{itemize}
\goodbreak

\TeX\ processes documents a paragraph at a time, rather than a line
at a time like most other programs.  Internally, \TeX\ computes a value 
called {\it\idx{badness}} for each
line of the \ixx{paragraph}{paragraphs (building)}.  Anything that
detracts from the appearance of a line (tight or loose spacing, a
hyphen, etc.) increases the badness associated with that line.  Every
paragraph that \TeX\ produces is optimal in terms of the total amount
of badness present.  Because \TeX\ searches for an optimal
solution, changing the last word of a paragraph can affect
the spacing of the first line of the paragraph.
After you've gained a little bit of experience with \TeX, you'll be
able to override any one, or all, of the rules it uses to compute
badness, but in most situations you won't want to.  I will describe more
of \TeX's approach to text formatting and how it differs from that
of word processors, desktop publishers, and other markup languages in
the following sections.

\TeX\ is not a simple program, but a set of
programs, tools, fonts, and other types of files. 
Two programs form the core of the \TeX\ typesetting
system.  One of them is \TeX\ itself, the program that reads
your input files and transforms them into typeset form.  The other
program is \MF, a tool for creating fonts. 
Producing \TeX\ documents involves a series of steps, including
editing the document, running \TeX\ itself, and processing \TeX's
output in various ways.

%\TeX's twin goals of beauty and efficiency (minimum effort) are
%frequently in conflict.  The result of this conflict is an environment
%where minimum effort does not necessarily mean ``easiest for the
%novice.''  Don't worry, though, it's not really difficult so
%much as it is different. 
%
%The rest of this chapter surveys the ``lay of the land''; if you've
%never seen \TeX\ before, this chapter will give you the ``big picture''
%introduction to document production with \TeX.  If you're an old hand
%at \TeX\ hackery, this chapter will help you quickly find the sections
%of the book that you're most interested in.

Over the years, \TeX\ has been made available on almost every computer
\ixx{platform}{tex@\TeX!platforms}, so it is probably available for the
computer system that you use.  Compiling \TeX\ on different systems
has been possible, in large part, because \TeX\ is a text formatter and
not a word processor. 
Unlike a word processor, \TeX\ never deals directly with displaying
text on the screen or interacting with input from the keyboard (except in a
very basic way).  These features of an application are
typically the most difficult to port from one system to another.

Beyond the technical details that make translation from one system to
another possible, \ixx{Donald Knuth}{Knuth, Donald} added an 
important stipulation to
the free distribution of \TeX: in order for any program to be called
``\TeX,'' it must pass a rigorous \idx{test suite}.  This means that
the \TeX\ you use behaves exactly like the \TeX\ I use.\ff{This
is not a whole-truth.  Implementors of \TeX\ may make some system-dependent 
alterations as long as the resulting program still passes
the test suite; so our \TeX{}s may not behave {\em exactly\/} the
same way.  They will, however, produce identical documents given
identical input (unless the input relies on system-dependent features
not available in both \TeX{}s, naturally.  ;-)}  This feature has
contributed greatly to \TeX's success.  It means that a large
community of \TeX\ users can transparently
\ixx{share}{documents!sharing} documents.

\section{\protect\TeX\ for Beginners}
\label{sec:texforbeginners}

If you are already familiar with \TeX, you may find some of the
material in this section repetitive.  If so, just skim it quickly.
This section will help you understand how \TeX\ interprets the things
you type into your input file.  When you understand the concepts
discussed here, you'll be ready to write really, really simple
documents in \TeX.

\subsection{Boxes and Glue}
\index{boxes|(}

Despite the apparent complexity of \TeX's job, it uses a very simple
metaphor: all typographic elements are \idx{boxes}.  The simplest boxes,
individual \ixx{characters}{characters!as boxes}\index{boxes!characters} have 
a set shape
defined by the font they come from.  There are three parameters
that define a box: 
\ixx{width}{boxes!width}\index{width!of box}, 
\ixx{height}{boxes!height}\index{height of box},
and \ixx{depth}{boxes!depth}\index{depth of box}.   
The distinction between height and
depth is a bit subtle.  When a row of characters is typeset,
every character rests on an imaginary line called the \textit{baseline}.  
Some characters, like the lowercase ``g,'' 
descend below the \ixx{baseline}{baselines}.
The distance from the baseline to
the top of a box is its height; the distance from the baseline to the
bottom is its depth.  

Figure~\ref{fig:charbox} shows the character boxes formed by the \ixx{Computer Modern}{Computer Modern fonts} Roman letters ``g'' and ``h.''  The {\it x-y\/}
distance of each box is its height and the {\it y-z\/} distance is its depth.
The \ixx{\textit{reference point}}{reference point (of boxes)} of the box,
marked with an \textit{r}, is on the leftmost edge of the box where the height
and depth meet.  Characters that have no \textit{\idx{descenders}} 
(no elements that go below the baseline), have a depth of zero.  
\TeX\ uses the character box metrics, but font designers are free to allow 
glyphs to extend outside the box (for example, at the top of the ``g'').

\epsbox[g-h.eps]{fig:charbox}{The Letters ``g'' and ``h'' inside their boxes.}

The following paragraph demonstrates how \TeX\ uses the metrics from
the physical dimensions of each character to build word, line, and
paragraph boxes. 

\bgroup
\tolerance=10000
\hbadness=10000
\def\surit#1{\hbox{\hskip.5pt\vbox{\vskip.5pt\hbox{#1}\vskip.5pt}\hskip.5pt}}
\def\boxit#1{%
  \setbox0=\hbox{#1}%
  \dimen0=\ht0
  \advance\dimen0 by .9pt
  \hbox{\vbox to \ht0{%
      \vss\hbox{\vbox to \dimen0{%
          \hbox{\vrule\vbox{\hrule\surit{#1}\hrule}\vrule}\vss}}}}}
\dimen1=\textwidth
\advance\dimen1 by -8pt
\beginsbox{\sidebarbox}
  \tolerance=100000
  \begin{minipage}{\dimen1}
    \TeX\ \ixx{``glues'}{glue}' 
    character boxes together to form words.  When boxes are
    joined, they are always joined so their reference points are horizontally
    aligned as shown in Figure~\hbox{1-1}.\footnotemark\ Character-boxes (like
    \boxit{t}\kern.5pt\boxit{h}\kern.5pt\boxit{i}\kern.5pt\boxit{s}) are
    joined to form words, \ixx{word-boxes}{boxes!word} 
    (like \boxit{this}) are joined
    \advance\dimen1 by -1.8pt \boxit{\hbox to \dimen1{to form lines, and
    \ixx{line-boxes}{boxes!line} (like this) form paragraphs.  
    \TeX\ accomplishes the task}} of
    forming a justified paragraph by allowing the glue between words to
    \ixx{stretch}{stretching} and 
    \ixx{shrink}{shrinking} a little
    bit and by occasionally breaking the glue between characters to insert a
    hyphen.  Although the rules are slightly different, \TeX\ builds a page
    out of vertical boxes (paragraphs, figures, etc.) in an analogous manner.
  \end{minipage}
\endsbox
\leavevmode
\hbox{\vrule\vbox{\hrule\surit{\usebox{\sidebarbox}}\hrule}\vrule}
\footnotetext{This is subtly different from saying that they are joined at the
baseline.  There are \TeX\ commands which can change the position of the
reference point in a box, whereas the baseline is an imaginary line that
depends solely on the shape of the character.}
\egroup

This is a very generalized overview.  In reality, a lot of subtlety
is required to capture all of the nuances of typographical appearance.

\index{boxes|)}
\subsection{Control Sequences}
\label{sec:tex:controlsequences}

A \textit{\ixx{control sequence}{control sequences}} 
is a special ``word'' that you put in your
document.  These extra words are instructions for \TeX, and they do
not usually appear in your typeset document.
Example~\ref{ex:csexample} shows a contrived example of a \TeX\
document that uses several control sequences.

\begin{example}{ex:csexample}{An Example of a \protect\TeX\ Document}
\def\ora{O'Reilly \& Associates}
\font\orafont=grlg10
\parskip=\baselineskip
\parindent=0pt
\pageno=5
This book is published by \ora\ in
the \textit{Nutshell} series.
\bye
\end{example}

In most macro packages, a control sequence is
a backslash followed by a sequence of letters.\ff[2]{Technically,
it's any character defined to be in the ``escape'' category followed
by any sequence of characters defined to be in the ``letter'' category or
a single character in the ``other'' category.}  \TeX\ is case-sensitive, 
so the control sequence
\cs{large} is different from \cs{Large} (these control
sequences switch to large and very large fonts in the
\LaTeX\ macro package).  Control sequences end with the first
non-letter, even if it isn't a space.  For example,
\cs{parskip0pt} is the control sequence \cs{parskip}
followed by \verb|0pt|. This control sequence tells \TeX\ to
insert zero points of extra space between paragraphs.

Unless instructed otherwise (with control sequences),
\TeX\ builds rectangular paragraphs out of lines of words.
Changing fonts, building tables, and typesetting 
\ixx{mathematical equations}{mathematics}
are examples of situations in your document where \TeX\ needs 
extra information.  

The number of control sequences used in a \TeX\ document may seem
overwhelming at first.  Luckily, every
control sequence falls into one of several categories:

\begin{iplist}{.25in}
  \ipitem[Macro control sequences]
        \ixx{Macro control sequences}{control sequences!macro} associate a 
        name with an arbitrary string of text (including other control
        sequences).  
        They are interpreted by 
        replacing the control sequence with the text of its 
        definition.\ff{Actually, macro expansion differs from pure 
        textual replacement in 
        a number of technical ways, but they aren't important here.}
\goodbreak
        Macro control sequences are the root of \TeX's tremendous
        flexibility.  By defining control sequences with meaningful
        names, like \cs{chapter} and \cs{footnote}, \TeX\ can
        present a reasonably simple interface to the user.  By 
        redefining those control sequences, the typeset output can
        be modified without requiring you to retype large quantities of text.

        In Example~\ref{ex:csexample}, the macro control sequence 
        \cs{ora} is defined as a shortcut for typing ``O'Reilly \& 
        Associates.''  This is a simple example of how a macro control 
        sequence can be used.

  \ipitem[\ixx{Font control sequences}{control sequences!font}]
        In Example~\ref{ex:csexample}, the
        line \cs{font}\verb|\orafont=grlg10| creates a font 
        control sequence
        called \cs{orafont}.  When \cs{orafont} is used, \TeX\
        will begin typesetting in the font  \texttt{grlg10}.  The name of
        the font, \texttt{grlg10} in this case, refers to an external file 
        that
        contains font metric information.  Fonts are discussed in
        Chapter~\ref{chap:fonts}, {\it \nameref{chap:fonts}}.

  \ipitem[Registers]
        Registers\index{control sequences!use of registers by} are like
        variables in a programming language.  They hold a single
        value of a particular type.  Many types of
        values can be stored: numbers (also called ``count''
        values because they are simple counting numbers like 1, 2,
        17, or -5), dimensions (also called ``lengths''; they
        are distances like 3.5pt or 2in), boxes, glue, and token 
        lists (an internal representation of your document used by 
        \TeX).\ff[2]{Technically, several other kinds of values are
        stored this way as well, but they are less common and won't
        be discussed in this book at all.}

        If you are unfamiliar with computer programming, think of
        these registers as place holders.  When \TeX\ needs to save a
        piece of information, like how much space should be inserted
        between paragraphs, it stores the information in a register.  
        When the information is needed again, in this case when \TeX\
        has finished typesetting one paragraph and is about to start
        another, it can retrieve that information from the register.
        Registers are usually given names that at least hint at how
        they are used.  This helps people read and modify the rules
        that \TeX\ uses to typeset documents.

        In Example~\ref{ex:csexample}, \cs{parskip}, 
        \cs{baselineskip}, and \cs{parindent} are dimension 
        registers.  The \cs{pageno} control sequence is a 
        count register.  

        There are only 256 registers of each type. The type of information 
        (number, dimension, or token list)
        that a register can contain is defined when the control sequence
        is created.  Once a variable like
        \cs{parindent} is created to hold a dimension, it can never
        hold a number or a token list.\ff{Most control sequences can
        be redefined to hold different kinds of values, but they can never
        hold different kinds of values at the same time.  A dimension register
        can be redefined to hold tokens, for example, but then it can't hold
        dimensions anymore (unless it is redefined again).}

        Registers may seem unnecessary now that you know about macro
        control sequences, which can store arbitrary information.
        However, registers differ from macro control sequences not only in the
        types of values they can hold, but also in the types of
        operations that can be performed on them.  There is a \TeX\
        command called \cs{advance}, for example, that can 
        increment the value stored in a register by an arbitrary amount.
        You can't \cs{advance} a macro control
        sequence.  

  \ipitem[Built-in commands]
        A number of control\index{control sequences!built-in commands}
        sequences are built into \TeX.  These ``primitive''
        operations form the basis for all higher-level functionality.
        There are a wide variety of control sequences of this type.
        Everything that can be done in \TeX\ can be reduced to a
        sequence of primitive operations.

        There is no way to know, simply by inspection, if a control
        sequence is one of the built-in sequences or not.  Luckily, it
        doesn't matter very often; it really only
        matters when you are writing complex macros.

        The \cs{font} control sequence in Example~\ref{ex:csexample}
        is a built-in control sequence.  So is \cs{advance}, mentioned
        above.

%  \item Any others????\editorial{how much detail is too much?}
\end{iplist}

The number and kind of control sequences available depends upon the
macro package that you are using. (Macro packages are discussed fully
in Chapter~\ref{chap:macpack}, {\it \nameref{chap:macpack}}.)  For the
rest of this chapter, the default settings of \ixx{Plain \TeX}{Plain tex@Plain \TeX} are
assumed.\ff[2]{Plain \TeX\ is the name of a particular macro
package.  I selected it for the purpose of example in this chapter
because it is always installed with \TeX.  Most of what follows in
this chapter is true in other macro packages as well, but some of the
details are different.  See Chapter~\ref{chap:macpack} for more
information.}  There are other macro packages, like \LaTeX,
Lollipop, and \TeXinfo, which have different default values.

\subsection{Special Characters}
\index{special characters}

In addition to control sequences, \TeX\ reserves several characters
\index{special characters} for special purposes.  Most 
of them do not occur very frequently in
ordinary text, but you must be aware of them because there will be
very surprising consequences if you use them incorrectly.

Table~\ref{tab:activechars} shows all of the special characters in
Plain \TeX.\ff[3]{All of these special characters are configurable,
but most macro packages use the Plain \TeX\ defaults.}  Most of these 
characters are special in other macro
packages as well.  Font-specific characters are not reserved by \TeX,
but they don't produce the results you would expect when typeset in
Computer Modern because of the way \TeX\ expects fonts to be laid out.
Fonts are discussed in detail in Chapter~\ref{chap:fonts}, {\it
\nameref{chap:fonts}}.

\begin{xtable}{c|l}
  \caption{Special Characters in Plain \protect\TeX
    \label{tab:activechars}
    \index{characters!special!typesetting}}\\
  \bf Character & \bf Meaning \\[2pt]
  \hline
%\endfirsthead
%  \caption{Special Characters in Plain \protect\TeX\ (continued)}\\
%  \bf Character & \bf Meaning \\[2pt]
%  \hline
%\endhead
  \tstrut
  \verb|#| & Used for parameter definition in macros and tables \\
  \verb|$| & Toggles in and out of math mode \\
  \verb|%| & A comment (\TeX\ ignores everything to the end of the line) \\
  \verb|&| & The column separator in tables \\
  \verb|~| & The active space (an unbreakable space) \\
  \verb|_| & Marks a subscript (valid only in math mode) \\
  \verb|^| & Marks a superscript (valid only in math mode) \\
  \verb|\| & Begins a control sequence \\
  \verb|{| & Begins a group \\
  \verb|}| & Ends a group \\[2pt]
  \verb+|+ & Produces an em-dash (---) (font-specific)\\
  \verb|<| & Produces an upside down exclamation mark (\,!`\,) 
             (font-specific)\\
  \verb|>| & Produces an upside down question mark (\,?`\,) 
             (font-specific) \\
  \verb|"| & Incorrect for quoted text; use `{}` and '{}' instead 
             (font-specific)\\[2pt]
  \hline
\end{xtable}

It is best to avoid these characters until you are familiar with \TeX.
If you need to typeset\index{special characters!typesetting} one 
of these characters,
Table~\ref{tab:acttype} shows what to put in your document. You
should also avoid characters outside the standard printable
\idx{ASCII character set} (characters with ASCII values below 32 and
above 126).  \TeX\ can be configured to accept characters outside the
printable ASCII range, to support non-English languages, for example,
but it is not configured to do so ``out of the box.''
Chapter~\ref{chap:foreign}, {\it \nameref{chap:foreign}}, discusses
the issues of typesetting in different languages.

\begin{xtable}{c|l}
  \caption{How to Typeset Special Characters
    \label{tab:acttype}
    \index{special characters!typesetting}}\\
  \bf To Get  & \bf Put This in Your Document\\[2pt]
  \hline
\endfirsthead
  \caption[]{How to Typeset Special Characters (continued)}\\
  \bf To Get  & \bf Put This in Your Document\\[2pt]
  \hline
\endhead
  \tstrut
  \#                & \verb|\#| \\
  \$                & \verb|\$| \\
  \%                & \verb|\%| \\
  \&                & \verb|\&| \\
  ``                & \verb|``| \\
  ''                & \verb|''| \\
  \~{\ }            & \verb|\~{\ }| \\
  $\{$              & \verb|$\{$| \\
  $\}$              & \verb|$\}$| \\
  $<$               & \verb|$<$| \\
  $>$               & \verb|$>$| \\
  $|$               & \verb+$|$+ \\
  $\underbar{\hbox{\ }}$ & \verb|$\underbar{\hbox{\ }}$| \\
  $\hat{\hbox{\ }}$ & \verb|$\hat{\hbox{\ }}$| \\
  $\backslash$      & \verb|$\backslash$| \\[2pt]
  \hline
\end{xtable}

\begin{sidebar}
Some of the suggestions in Table~\ref{tab:acttype} will not always produce
exactly what you want.  The entry for ``\~{\ }'' really produces a tilde
accent, not a tilde character and the entries for ``$\{$'' through
``$\backslash$'' all get the actual characters from \TeX's math fonts.  The
Computer Modern text fonts don't include these characters so it is necessary
to get them from the math fonts.  However, if you are using PostScript
or other kinds of fonts, you may very well have curly braces, angle brackets,
underscores, etc. in the font.  You can access these characters directly 
with the \cs{char} primitive.  I strongly recommend that you always
define macros for this purpose, so that you can easily switch to some other
method if you change fonts.  Introducing \cs{char} primitives makes your
document less portable.  To use the \cs{char} primitive, simply put
the decimal ASCII value of the character that you want to print.  For example,
this book is typeset with PostScript fonts that include a backslash character
at position 92, so I defined \cs{bs} to print a backslash like this:

\begin{shortexample}
\def\bs{\char92\relax}
\end{shortexample}

Using \cs{relax} after the decimal value assures that \TeX\ won't
get confused if I put a backslash in front of other digits like
this \cs{bs300dpi}.
\end{sidebar}

The \ixx{braces}{curly braces} ``\{'' and ``\}'' are a very special
case.  \TeX\ uses curly braces to delimit arguments and make changes
(like switching fonts) that are local to a small section of the
document.  These are 
called \textit{\idx{grouping characters}}\index{characters!grouping} in \TeX\
jargon.  For example, to typeset a single word in boldface, you 
put \verb|{\bf word}| into your input file.  The \cs{bf}
control sequence switches to boldface type, and the curly braces
localize the effect to the single word {\bf word}.  As a result, it is
very important that you avoid braces (except when you use them as
delimiters) and that you carefully match all opening and closing
braces.  One of the most common errors in \TeX\ is to forget a closing
brace.

One last special character is the \idx{blank space}.  For the most part,
\TeX\ doesn't care how you space your lines of text.  Any space that
occurs is simply a word break to \TeX, and inserting multiple spaces
doesn't influence how \TeX\ typesets the line.  \TeX\ also considers
the end of a line an implicit space.  If you are trying to control
the layout of your input text and want to break a line without
introducing a space, place a comment character (\% in most macro
packages)\index{comments!in tex@in \TeX} at the very 
end of the line.  If the last character of a
line is the comment character, \TeX\ ignores the line break and all
the leading spaces on the following line.  This allows you to use
\idx{indentation} to make your input file more readable.
\goodbreak

For example, the following lines in your input file:

\begin{shortexample}
``This                      is some ex%
      ample text.''
\end{shortexample}

and this line:

\begin{shortexample}
``This is some example text.''
\end{shortexample}

both produce:

\begin{exindent}
  \small{``This is some example text.''}
\end{exindent}

in your typeset document.

\section{Text Formatting Versus Word Processing}
\label{sec:textformVSwordproc}

For many people, writing documents with a computer implies using a
word processor like \ixx{\product{WordPerfect}}{WordPerfect} or \ixx{\product{Microsoft Word}}{Microsoft Word}.\index{tex@\TeX!vs. word processing}\index{word processing!vs. tex text processing@vs. \TeX\ text processing}
The word processing program controls
every aspect of what you do: it's where you type your text, where you
see what it will look like, where you print, and where you do
everything else.  Some of these environments, the so-called 
WYSIWYG\index{WYSIWYG programs}
(what-you-see-is-what-you-get) programs, attempt to show you what the
printed document will actually look like {\em while\/} you edit
it.\ff{\TeX\ pundits, and other folks who have been frustrated by
the limitations of these environments, frequently refer to this as
WYSIAYG---what you see is {\em all\/} you get.}

If WYSIWYG environments are what you're used to, or what you expect, \TeX's approach may
seem very strange at first because \TeX\ is 
a {\em text formatter\/},\index{tex@\TeX!as  a text formatter}\index{text formatting}
not a word processor.  Instead of trying to show you what your
document will look like while you type, \TeX\ expects you to do all
the typing somewhere else, and then pass it a source file
containing all of your text plus control sequences that tell
\TeX\ how you'd like it printed.

In {\it The Psychology of Everyday Things}~\cite{dn:psyeveryday},
\ixx{Donald Norman}{Norman, Donald} describes these 
two modes of interaction as first person
and third person.  First person interaction provides the user with
the ability to directly manipulate the elements of a task, whether it's
flying an airplane or resizing text.  Third person interaction, on the
other hand, occurs where the user is expected to type commands to the
computer in an appropriate command language; the shell prompt is a good
example of third person interaction.

Is first person interaction really better?  Well, it depends.  Norman 
writes, ``Although
they [WYSIWYG environments] are often easy to use, fun, and entertaining, it is often difficult
to do a really good job with them.''  The
problem which arises is that the user is required to do the task, and he
or she may not be very good at it.  Third person systems are better when
the computer program can be trusted to do a better job of the task than
the user.

Is \TeX\ really better than a word processor?  Well, it depends on the
task and the person doing it.  \TeX\ probably isn't better for designing
one page flyers with lots of fonts and graphics (although I've done it).
But for longer documents, \TeX\ offers all of these advantages: 

\begin{itemize}
  \item \TeX\ has a precise understanding of the rules of 
        typesetting,\index{typesetting!rules} so you don't have to.
  \item Predefined styles allow experts to extend (or bend) the rules
        of typesetting without burdening the user.
  \item Journals and magazines can achieve consistency of appearance
        much more reliably because the consistency is in the style files.
  \item \TeX\ runs on cheap systems (old PCs with monochrome monitors and
        no graphics capability, for example).
  \item Although complex and difficult to learn, \TeX\ offers incredibly flexible table
        construction tools.
  \item Few, if any, word processors can provide running headers and footers
        as flexibly as \TeX.  Imagine the task of writing a dictionary: the
        left and right hand side headers change on each page, each time a 
        new entry is added.
  \item \TeX\ offers flexible bibliography layouts.
  \item \TeX\ is extensible.\index{extensibility of tex@extensibility of \TeX} Its behavior 
        can be modified by defining new commands and environments 
        without changing the actual program.
\end{itemize}

There are some other good reasons to separate document creation from
text formatting:

\begin{itemize}
  \item Documents are portable.\index{portability of tex@portability of \TeX}  Because the 
        source files are just plain
        text without any nonprintable characters, they can easily be
        copied from one system to another.
  \item \TeX\ is portable.  
        \TeX\ runs everywhere.  You can
        process your documents with \TeX\ on \Unix\ workstations; personal
        computers running MS-DOS, OS/2, and Windows; IBM mainframes
        running VM/CMS; workstations running VAX/VMS; Macintoshes; Amigas;
        Ataris; and just about every other computer with a reasonable
        amount of memory.  And the typeset output will be the same!
        This adds another dimension of portability to your documents.
  \item \TeX\ is free.  You can afford to have it on every system
        you use.  Several sources of \TeX\ software are listed in
        the preface of this book.  

  \item \TeX\ allows you to separate markup and output.
        Logical divisions in the text (chapters,
        sections, itemized lists, etc.) are identified by control sequences.
        An entirely different page layout can result from simply changing the
        definition of a few control sequences.

        This means that the look of your documents
        can be changed (to fit the style guidelines of a particular
        journal or publisher, for example) without changing the text
        of your documents at all.  

  \item Plain text files are easier to manipulate with other tools than
        specially encoded word processor files are.  This
        means that you can use standard utilities on your documents:
        revision control, grep, shell scripts, etc.  This is a less common
        practice in non-\Unix\ environments, but it is still convenient.

  \item You can continue to use your favorite editing tools.  The extent
        to which you find this advantageous is dependent, naturally, on the
        extent to which you have a favorite editing program.
        Nevertheless, this can be a considerable advantage.  For example, 
        users familiar with \program{emacs}\index{GNU Emacs} can 
        continue to rely on all of the
        features they are used to, including interactive spellchecking,
        access to online services like Webster's dictionary, customized
        editor macros, and convenient services like reading mail.

  \item You get better looking output.  \TeX\ gives you far more precise
        control
        over the placement of elements on the page than most word processing
        programs.  And \TeX\ is very intelligent about typesetting (paragraph
        breaking, \idx{kerning}, \idx{ligatures}, etc.).
\end{itemize}

\subsection{What About Desktop Publishing?}

Desktop publishing systems\index{tex@\TeX!vs. desktop publishing}\index{desktop publishing!vs. tex text  processing@vs. \TeX\ text  processing} like \ixx{\product{Ventura Publisher}}{Ventura Publisher} and
\ixx{\product{Aldus PageMaker}}{Aldus PageMaker} are noted for their ability to incorporate
multiple fonts and graphics into a document.  As word processors
become more sophisticated, the line between word processing and desktop
publishing is becoming blurry.

\setbox0=\hbox to\textwidth{\hfill\epsfig{figure=figs/lint.eps,width=2.1in}}
\parshape13
0pt \textwidth
0pt 3.9in
0pt 3.6in
0pt 3.2in
0pt 3.1in
0pt 3.1in
0pt 3.15in
0pt 3.05in
0pt 3.05in 
0pt 3in 
0pt 3.9in 
0pt 4.3in
0pt \textwidth
\leavevmode\hbox to 0pt{\vbox to 0pt{\vskip8pt\box0\vss}\hss}%
This book shows you many ways that \TeX\ can provide access to the same
sophisticated features.  \TeX\ can incorporate pictures and figures in
a number of ways
(just take a look at the way I've wrapped text around this kiwi),\ff{\TeX\
doesn't do this sort of thing automatically, but it isn't hard to 
do.  Why the kiwi?  It's on my business card$\ldots$}
and \TeX\ can use almost any font that another 
program can use---it can certainly use {\em all\/} of the popular types
of fonts.  Like typical word processors, desktop publishing programs
force you to
use a single application to create your entire document, and they lack
the flexibility required to combine just the pieces that you want.  All of the
advantages of text formatting over word processing also apply to desktop
publishing programs.  I'll grant, however, that WYSIWYG
environments are easier for first-time users.  But that doesn't make 
them better, it just makes them more popular.\par
\parshape0

\subsection{What About troff?}

\program{troff}\index{tex@\TeX!vs. troff}%
\index{troff!vs. tex text processing@vs. \TeX\ text processing}
is the ``other'' text formatting system.  If you've ever
tried to read a \Unix\ reference page without formatting it first,
you've seen \program{troff}.  For a long time it was distributed as part of all
\Unix\ systems.  Now it is more likely an extra-cost option.  
The \ixx{Free Software Foundation's }{Free Software Foundation (FSF)}\ixx{\program{groff}}{groff} processor is a free,
\program{troff}-compatible system.

On the surface, it is easier to compare \TeX\ and \troff\ than to
compare \TeX\ to the other
document preparation systems described in this chapter.  In reality,
the differences are subtle: \TeX\ and \troff\ have the same
general paradigm; they are equally powerful to a large extent,
and both have advantages and disadvantages.

\program{troff} is similar to \TeX\ in many ways.  Like \TeX, \troff\
processes a plain text file and produces a typeset document.  \TeX\
and \troff\ differ in the way that formatting information is inserted
into the text.  \TeX\ uses \idx{control sequences}, where \troff\ uses a
mixture of control sequences\ff{Although it has a very different
notion of what constitutes a control sequence.} and ``dot'' commands
(lines of text that begin with a period and contain typesetting
commands).

Although I am inclined to say that \troff\ documents are far more cryptic
than \TeX\ documents, I am certain that there are plenty of \troff\ users
who would disagree (strongly).  

Objectively, \TeX\ handles \ixx{mathematical  typesetting}{typesetting!mathematics} far better than
\troff\ and probably has better support 
for \ixx{multilingual documents}{typesetting!multilingual documents}\index{international!typesetting}.
The \ixx{\program{nroff}}{nroff} processor, which produces plain text output
from a \troff\ document, at one time provided a strong argument in favor of
\troff\ for typesetting documents required in both typeset and plain
text formats.  However, the \ixx{\TeXinfo}{texinfo@\TeXinfo} macro package for \TeX\
has largely defeated that argument.  In \troff's defense, \TeXinfo\ is
very, very different from other \TeX\ macro packages, so it really is
necessary to plan ahead and learn a very different set of macros to
typeset both plain text and typeset documents with \TeXinfo.
Chapter~\ref{chap:online}, {\it \nameref{chap:online}\/}, discusses this
issue further.

In my experience, there is more free support for \TeX\ than \troff.
\TeX\ is supported by a large community of users actively producing
new, useful document-preparation formats, styles, and tools.
In addition, \TeX\ is more widely available than \troff:
a \TeX\ port exists for almost every practical computer system, whereas
\troff\ is still mostly confined to \Unix\ systems (although the Free
Software Foundation's \program{groff} package has been ported to similar
systems like MS-DOS, Windows NT, and OS/2).

The following fragments show a side-by-side comparison of \TeX\
commands, on the left, and \troff\ commands, on the right:

\begin{shortexample}
\begin{figure}                    .(z                    
 \begin{center}                   .hl                    
   \hrule                         Text to be floated.    
   \vspace{8pt}                   .sp                    
   Text to be floated.            .ce                    
   \hrule                         .hl                    
   \caption{Example figure...}    Figure \*[fig]: Example figure...
   \vspace{8pt}                   .)z                    
 \end{center}
\end{figure}
\end{shortexample}

Both examples
produce a floating figure that looks like this:

\begin{figure}
  \begin{center}
    \hrule
    \vspace{8pt}
    Text to be floated.
    \label{fig:texexample}
    \vspace{8pt}
    \hrule
    \caption{Example figure produced by both \TeX\ and \program{troff}.}
  \end{center}
\end{figure}

\subsection{What About SGML?}

The \idx{Standard Generalized Markup Language (SGML)} is a document
description language.  SGML aims to separate the content of a document
from its presentation.  In other words, SGML identifies the features
of a document (chapter headings, paragraphs, etc.) without specifying
how they are to be presented.

This means that all SGML documents must interact with a
document formatter of some sort.  Many people are finding that \TeX\
is a natural choice when selecting a document formatter for their SGML
environment.  In fact, \ixx{\LaTeX}{latex@\LaTeX!use with SGML} 
%\fxxtnote{\LaTeX\ is a macro package for \TeX.
%Macro packages provide additional commands for \TeX\ and make typesetting
%some kinds of documents relatively easier than other macro packages.  They
%are discussed fully in Chapter~\ref{chap:macpack}, 
%{\it \nameref{chap:macpack}}.}  
already provides many SGML-like
commands because it was designed to separate markup from presentation.
  One of the specific goals of an effort (currently
underway) to develop a new version of \LaTeX\ is to make SGML and
\LaTeX\ work together easily, cleanly, and efficiently.  For more
information about the goals of this project and information about what
you can do to help, please read {\it The \LaTeX3 Project}~\cite{l3:project}.

\section{How \protect\TeX\ Works}
\label{sec:howtexworks}

A functioning \TeX\ system\index{tex@\TeX!overview of operations} in 
which you are producing documents of
medium size and complexity is really a collection of tools and files
that are related to each other in well defined (if somewhat subtle)
ways.

One of the fundamental goals of this book is to shed light on these
relationships and allow you to put together a \TeX\ system that quickly 
and easily does the jobs you need to accomplish.

\subsection{\protect\TeX{}ing a Simple Document}

This section briefly describes what you need to know about how \TeX\
processes a simple document (that is, one that does not contain complex
document elements like a table of contents, indexes, bibliographies, etc.).
Figure~\ref{fig:wholesimple} shows how the standard \TeX\ tools fit
together at the most basic level.
\goodbreak

\epsimage{fig:wholesimple}{A high-level view of \protect\TeX}

Figure~\ref{fig:wholething} expands on Figure~\ref{fig:wholesimple},
showing additional tools and files that you'll often need to use.

\epsimage{fig:wholething}{High-level view of \TeX\ including 
  more detail}

\subsubsection{Editing your document}

The most tangible and important part of your \TeX\ system is 
your \ixx{document}{documents!tex@\TeX}.
This is the file (or files) in which you write down what you want to typeset with
\TeX.  In addition to the actual text, you include 
control sequences to describe how you want the final text
to appear (size, font, justification, etc.).  The section
``\nameref{sec:texforbeginners}'' earlier in this chapter tells you briefly
what goes into your document file.

The most common way to create a document is with 
an \ixx{editor}{editors}\index{editing tex@editing \TeX}\index{tex@\TeX!editing}, which can provide you with a number of features to
make typing \TeX\
documents easier.  For example, an editor can help you insert
common \idx{control sequences} automatically, run \TeX\ automatically 
(from within
the editor), and keep you from making common mistakes (like typing a left
brace, but not the matching right one).  These features and how they
work in editors including GNU Emacs, \auctex, and Multi-Edit are
described in Chapter~\ref{chap:editing}, {\it\nameref{chap:editing}}.

\subsubsection{Running \protect\TeX}
\label{subsubsec:runningtex}

Once you have prepared your document file, it is time 
to run\index{running tex@running \TeX}\index{tex@\TeX!running} the \TeX\
program itself.  This may not be as easy as it sounds.  You need to determine
the name of the \TeX\ program at your site, to make sure
all of the files \TeX\ needs are available to it; you also need to specify the correct
command-line options.  Chapter~\ref{chap:running}, {\it\nameref{chap:running}},
describes everything you need to know.

\TeX\ may find errors in your document (places where \TeX\ doesn't
understand the instructions you used; not spelling or grammatical
errors, unfortunately ;-). Chapter~\ref{chap:running} also describes
the most common errors you're likely to make and gives advice for
interpreting error messages.

\newpage
If \TeX\ is successful in formatting your document (i.e., your 
document doesn't contain any errors), it produces a \ext{DVI}
(DeVice Independent)\index{DVI files} file.  The \ext{DVI} file 
is a device-independent
representation of the typeset output of your document.  \ext{DVI}
files are transitory.  Although there are a few programs that can
manipulate them (to rearrange the order of the pages in the output,
for example), most of the time you will immediately transform
them into something else---either printed output or previewed output
on the screen.  (See the following section ``\nameref{subsubsec:printpreview}.'')

\subsubsection{Using macros}

The \ixx{control sequences}{control sequences!macro} that you 
insert in your document are 
defined by a \ixx{macro package}{macro packages}.\ff{Well, 
actually, they're
\TeX\ primitives, are defined by a macro package,
defined in a file loaded by a macro package, or defined in
your document.}
Macro packages are collections of \TeX\
commands (macros) that extend \TeX.  Macro packages are frequently
stored in format files, specially compiled versions of the macro
package.  The \ixx{ini\TeX}{initex@ini\TeX} program interprets all of the control sequences
in a macro package to create a format file that \TeX\ reads when it runs.

\newpage
Many macro packages are particularly effective in implementing
particular document styles or supporting particular types of writing.
Two of the most common are \ixx{Plain \TeX}{Plain tex@Plain \TeX} and \ixx{\LaTeX}{latex@\LaTeX}.
Chapter~\ref{chap:macpack}, {\it\nameref{chap:macpack}}, describes Plain
\TeX, \LaTeX, and a number of other macro packages that extend the power
and ease of \TeX.

\subsubsection{Using fonts}

One of \TeX's strengths is its support for a myriad 
of predefined \ixx{fonts}{fonts!tex@\TeX}  and its
ability to let you create fonts of your own.  
In addition to your document and the format file, when \TeX\ runs it
needs font information as well.  This is provided in the form of a set
of \ext{TFM} (\TeX\ Font Metric)\index{TFM files} files 
that tell \TeX\ the size and
shape (roughly speaking, at least) of each character, as well as
some other information about how characters are related to each other.

Historically, the \ixx{\MF}{metafont@\MF} program was the way a \TeX\ user created fonts.
Like \TeX\ itself, \MF\ is about ten years old.  Ten years ago, it was
a unique program that was indispensible for creating the type of
output \TeX\ produces.  Today there are many competing font
technologies, all of them more common than \MF, and \MF's role is
diminishing.  Many people use \TeX\ today without ever using \MF\ at
all.  Nevertheless, \MF\ still has some importance, and we describe
how to run and use it in Chapter~\ref{chap:mf},
{\it\nameref{chap:mf}}.  Because the standard fonts that come
with \TeX\ are still the fonts produced by \MF, it will also be
mentioned elsewhere in this book.

If you are writing complex documents, you may need to learn a lot about
fonts and how to define and use them.  Chapter~\ref{chap:fonts},
\textit{\nameref{chap:fonts}},
tells you everything you need to know, including information about 
the New Font Selection Scheme,
a new way of describing and selecting fonts in \TeX.

\subsubsection{Previewing or printing \TeX\ documents}
\label{subsubsec:printpreview}

After you have produced a \ext{DVI} file, as described in the section
``\nameref{subsubsec:runningtex},'' later in this chapter, you run another program (generically
called a \dvidriver)\index{DVI drivers} to 
translate the \ext{DVI} file so you can either preview
or print your document.  Driver programs need your \ext{DVI} file and some
collection of fonts 
(usually \ext{PK} (packed)\index{PK fonts@\ext{PK} fonts}\index{fonts!PK}\index{PK files@\ext{PK} files} font 
files).\ff{Some
drivers may also benefit from loading the \ext{TFM} files used to create your
document.}  Many different kinds of fonts are described in
Chapter~\ref{chap:fonts}.

Chapter~\ref{chap:printing}, {\it\nameref{chap:printing}}, tells you
how to print\index{printing}\index{tex@\TeX!printing} your 
documents and deal with the problems you may
encounter using bitmapped or scalable fonts, printing pictures and figures,
and other printing issues.

Often you will want to look at your document before you actually print it.
Because \TeX\ is not a WYSIWYG system,\ff[2]{\Textures\ for the Mac
and \program{Scientific Word} offer WYSIWYG-like environments, but that's
not the point ;-)} you cannot do this until you have processed the 
\ext{DVI} file.  There are a number of 
good previewing products\index{previewing}\index{tex@\TeX!previewing},
including \ixx{\xdvi}{xdvi}, \ixx{\program{dvimswin}}{dvimswin}, 
and \ixx{\program{dviscr}}{dviscr}, that let you look at
your processed document on the screen before you decide whether to print it.
See Chapter~\ref{chap:preview}, {\it\nameref{chap:preview}}, for complete
information.

\subsection{\protect\TeX{}ing More Complex Documents}

This section briefly describes how \TeX\ processes a more complex
document (that is, one that includes elements like a
table of contents, indexes, bibliographies, etc.).

Many \TeX\ formats implement sophisticated cross-referencing schemes.  
\ixx{Cross references}{cross references} may sound 
rather esoteric, but they occur
frequently.  Tables of contents, figure and table numbers, indexes,
and bibliographic references are all flavors of cross referencing.

Cross references make your document more complex because they require more
information than is immediately available when \TeX\ initially
processes your document.
For example, if you refer to a figure which occurs later in the document,
\TeX\ has no way of knowing what figure number to insert into the text at the
point of the reference.  These are called \ixx{\textit{forward references}}{forward references}.

\TeX\ macro packages that support cross referencing overcome the difficulty of
forward references by requiring you to process your document more than once.
Each time your document is processed, the necessary reference information is
stored into a separate file.  If that file exists when you process your
document, the information saved {\em last time\/} is loaded so that it is
available {\em this time}.  The practical implication of this functionality is
that documents with cross references frequently have to be processed twice.
Occasionally, you may have to process a document three times.  This occurs
when the inserted reference causes \TeX\ to format a paragraph differently,
which in turn causes \TeX\ to change a page break.\ff{With extreme
cleverness or extreme bad luck you can create a document which will {\em
never\/} format correctly.}  Because most changes  are incremental
while revising
a document,
this is normally only an issue the first
time you process a document.

The following sections describe the \LaTeX\ methods for constructing a
table of contents, figure references, an index, and a bibliography.
\LaTeX\ is used in this example because it is a very common macro
package and is typical of the way macro packages provide these
features.  Similar mechanisms exist in most formats, except Plain \TeX.

Figure~\ref{fig:texcmplx} shows the relationships between many of the
components described in the following sections.  \LaTeX\ creates several
sorts of auxiliary files depending on the kind of cross references required
by your document and the style files you use.  These auxiliary files may
be modified (and others may be created) by other sorts of post-processing
programs (like \ixx{\program{MakeIndex}}{MakeIndex} for 
constructing indexes or \ixx{\BibTeX}{bibtex@\BibTeX}
for constructing bibliographies).   \LaTeX\ uses these auxiliary files,
if they exist, to update your document when it is processed again.

\epsimage{fig:texcmplx}{\TeX{}ing a More Complex Document}

\subsection{Building a Table of Contents}

A \idx{table of contents} is the simplest form of 
cross reference.  In \LaTeX, you
request a table of contents by 
inserting the \cs{tableofcontents} command wherever you want it to 
appear in your document.  If you request the
table of contents at the end of your document rather than the
beginning, your document can be printed with only one pass through \TeX.

\LaTeX\ uses a file with the same name as your document and the extension
\filename{.toc} to hold the table of contents entries.  You can control the
level of detail in your table of contents by setting 
the \cs{secnumdepth} counter.  A value of zero includes only chapters; one includes
chapters and sections; two includes chapters, sections, and subsections, 
and so on.

The \LaTeX\ 
commands \cs{listoftables} and
\cs{listoffigures} perform the same functions as
\cs{tableofcontents} for lists of tables and figures.  They use
external files with the 
extensions \filename{.lot}\index{lot files@.lot files} and
\filename{.lof}\index{lof files@.lof files}, respectively.  As 
with the table of contents, your
document can be correctly formatted in one pass if the
\cs{listoftables} and \cs{listoffigures} commands are placed at
the end of the document.

\subsection{Figure References}

\ixx{Figure references}{figures!references to} are a 
special case of \LaTeX's cross referencing mechanism.
The \LaTeX\ command \cs{label}\verb|{|\texttt{\textit{string}}\verb|}| 
creates a referent.
You refer to the label with the command
\cs{ref}\verb|{|\texttt{\textit{string}}\verb|}|.
In normal body text, the label refers to the current section or subsection.
In a figure or table environment, the label refers to that figure or table.

If your document contains no \idx{forward references} (if all
\cs{label} commands occur before the \cs{ref}'s that refer
to them) then it can be formatted in one pass.  Otherwise, \TeX\ will 
have to be run two or three times
to make all of the references correct.

%\LaTeX\ issues a warning message each time it encounters a reference to a
%label that has not been defined.  After formatting the entire document,
%\LaTeX\ prints the following message if any references in the
%document are incorrect.
%
%\begin{center}
%  \tiny%
%  \verb|Label(s) may have changed. Rerun to get cross-references right|
%\end{center}
%
%If this occurs, you should process your document again.  In some cases, it may
%be necessary to process your document three times in order to make this
%message go away.  Your document {\em has not} been typeset correctly until
%\LaTeX\ stops issuing this warning.

\subsection{Indexes and Glossaries}

\ixx{Indexes}{indexes} and \idx{glossaries} differ from the preceding 
forms of reference in that they
must be processed by a separate program.  In general, this is true
regardless of the macro package or format you use.
An external program is required because indexes and glossaries must be 
alphabetized, and in indexes, 
consecutive page numbers have to be converted into ranges, and so on.

%To construct an index in \LaTeX, you must do six things. 
%
%\begin{enumerate}
%  \item Use the \ext{makeidx} style file. 
%  \item Insert the command \cs{printindex} where you want the
%        index to appear in your document.  
%  \item Use the \cs{index} command to mark all index entries.
%  \item Process your document with \TeX.
%  \item Run the \program{MakeIndex} program to build the actual index.
%  \item Process your document again to include the index.
%\end{enumerate}
%
%The \cs{index} command accumulates page number references in a file
%with the extension \filename{.idx}.  The command \cs{index{}{\it
%string}\verb|}| creates an index entry for the word or phrase ``{\it
%string\/}'' with the number of the page on which it occurs.  
%Specially coded index entries can be used to produce cross references 
%(``Infinite Loop / see Loop, Infinite'', for example), multi-level
%entries, specially typeset entries, etc.  Consult {\it MakeIndex: An
%Index Processor for \LaTeX}~\cite{ll:makeindex} for more information
%about specially formatted index entries.
%\editorial{example of each...}
%
%After processing your document with \TeX, run \program{MakeIndex}
%(or \program{MakeIndx} on file systems with 8-character filename restrictions)
%to build
%the index.  \program{MakeIndex} reads the \ext{IDX} file and collects the
%index entries; then it sorts the entries, combines multiple entries that refer
%to a contiguous sequence of pages into a single entry with a range of pages,
%and builds nested entries before writing the index to a file with the
%extension \filename{.ind}.  The next time you process your document, the
%\ext{IND} file will be inserted where the \cs{printindex} command
%occurs.

\subsection{Bibliographies}

\LaTeX\ works in conjunction with another program, called \ixx{\BibTeX}{bibtex@\BibTeX}, 
to provide
a flexible, convenient way to construct \idx{bibliographies}.  The \cs{cite}
commands allows you to refer to other documents in much the same way that the
\cs{ref} command allows you to refer to other portions of the same
document.

You make a citation by placing the command 
\cs{cite}\verb|{|\textit{string}\verb|}|
where you wish the citation to occur.  The {\it string\/} is a
key that refers to the document in your bibliography database that
you wish to cite.  Example~\ref{ex:knuthbbl} is a
typical entry in a bibliography database.  It 
describes Knuth's\index{Knuth, Donald} classic 
book {\it The \TeX{}book}~\cite{kn:texbook}.  The key for this
entry is ``kn:texbook.''

\begin{example}{ex:knuthbbl}{A typical bibliography database entry}
@Book{kn:texbook,
  author    = "Donald E. Knuth",
  title     = "The {\TeX}book",
  publisher = "Addison-Wesley",
  year      = 1989,
  edition   = "Fifteenth",
  isbn      = "0-201-13447-0"
  note      = "Paperback ISBN: 0-201-13448-9"
}
\end{example}

Each entry in the database consists of a type (book, article, magazine, etc.),
a key, and a number of fields.  The number and names of the fields depend
on the type of entry.  The database is simply a plain ASCII file containing
any number of entries.  You can have multiple databases.

These are the commands you use, in addition to
\cs{cite}, to include a bibliography in your document:

\begin{shortexample}
% The bibliography goes here.
%
\bibliographystyle{plain}
\bibliography{textools,refbooks}
\end{shortexample}

The \cs{bibliographystyle} command tells \BibTeX\ how to format the
bibliography, and the \cs{bibliography} command identifies which
bibliographic databases contain the citations that you have made.  The
``plain'' style of bibliography is selected, and the
\filename{textools} and \filename{refbooks} files contain the
bibliographic information for the documents cited.  Document styles
can be used to alter the format of citations in your text.
The default extension for bibliographic 
styles is \filename{.bst}\index{bst files@.bst files}. The 
default extension 
for database files is \filename{.bib}\index{bib files@.bib files}. 
\goodbreak

\LaTeX\ places citations and bibliography information into 
the \filename{.aux} file\index{aux files@.aux files}.  \BibTeX\ 
reads the \filename{.aux}\index{aux files@.aux files} file 
and constructs a bibliography,
which it places into a file with the 
extension \filename{.bbl}\index{bbl files@.bbl files}, using the
entries you cited and the bibliography style you selected.

\subsection{Special Things}

Sometimes, producing a complex document requires the ability to interface
with objects outside of \TeX\ (pictures or figures created by high-end
graphics packages, special features of a particular printer, etc.).  To
support this kind of communication, \TeX\ provides a control sequence
called \cs{special}.  The arguments passed to the \cs{special}
command are written directly to the \ext{DVI} file for the \dvidriver.
It is the responsibility of the \dvidriver\ to handle them.  \dvidriver{}s
typically ignore \cs{special} commands that they do not recognize.

You will find \cs{special} commands of various kinds described throughout
this book, particularly when discussing color typesetting in 
Chapter~\ref{chap:macpack}, \textit{\nameref{chap:macpack}}, and graphics
in Chapter~\ref{chap:pictures}, \textit{\nameref{chap:pictures}}.
