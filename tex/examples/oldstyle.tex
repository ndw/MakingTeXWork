% Filename: oldstyle.tex

\def\author{Norman Walsh %
           (derived from vpfcmrm by Alan Jeffrey)}
\def\email{norm@ora.com}
\def\lastmodified{04 Dec 1993}
\def\versionnumber{0.6}
\def\description{Computer Modern with old style numerals}

% This TeX program generates osncmr10.vpl (the virtual 
% property list for a fake CMR font) from:
%
%    cmr10 and cmmi10
%

\input makevpl
\newcount\source
\newcount\target

% Then we make the vpl file.

\makevpl{osncmr10}

   \codingscheme TeX-text-with-old-style-numerals;

   \designsize 10pt;

   \loadfont\basefont=cmr10 at 10pt;
   \loadfont\numbfont=cmmi10 at 10pt;

% The font dimens come from the basefont.

   \basefont
   \fontdimens
      1 : \fontdimen1\font
      \for\source\in2\to7\do ,\fontdimen\source\font \od;

% Create the standard text ligatures: 
%   ff fi fl ffi ffl -- --- '? '!

   \ligtable "f":"i"=:\oct"034","f"=:\oct"033","l"=:\oct"035";
   \ligtable \oct"013": "i"=:\oct"016", "l"=:\oct"017";
   \ligtable \oct"140": \oct"140"=:\oct"134";
   \ligtable \oct"047": \oct"047"=:\oct"042";
   \ligtable \oct"055": \oct"055"=:\oct"173";
   \ligtable \oct"173": \oct"055"=:\oct"174";
   \ligtable \oct"041": \oct"140"=:\oct"074";
   \ligtable \oct"077": \oct"140"=:\oct"076";
   \ligtable \oct"033": "i"=:\oct"036", "l"=:\oct"037";

% Then we load in the characters.

% All of the characters '000--'377 are the same as the 
% basefont encoding except the numerals which come from 
% the numbfont...
% This VPL script assumes that the basefont is CMR and 
% contains only '177 characters.

   \basefont 
     \character
         '000='000
         \for\source\in'001\to'057\do ,\source=\source\od
         \for\source\in'072\to'177\do ,\source=\source\od;

% From numbfont we take the numerals

   \numbfont
      \character 
         '060='060
         \for\source\in'061\to'071\do ,\source=\source\od;

\endmakevpl
\end
